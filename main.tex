\documentclass[11pt,a4paper,hidelinks]{article}   % 11pt Schrift, A4-Format, Artikelklasse
\usepackage[
    left=4cm, 
    right=2cm, 
    top=2.5cm, 
    bottom=2.5cm
]{geometry}                                                 % Seitenränder

%:: Eigenschaften des Dokuments                 
\usepackage[utf8]{inputenc}                                 % Eingabecodierung UTF-8 (für Umlaute etc.)
\usepackage[scaled]{helvet}                                 % Serifenlose Schrift Helvetica, skaliert
\usepackage[T1]{fontenc}                                    % Korrekte Zeichencodierung für Trennung & Sonderzeichen
\usepackage[german]{babel}                                  % Deutsche Sprache, Trennung, Datum etc.

%:: Inhalt & Darstellung
\usepackage{enumitem}                                       % Anpassung von Aufzählungen
\usepackage{amsmath}                                        % Erweiterte Mathematik-Umgebungen
\usepackage{amssymb}                                        % Mathematische Symbole (ℝ, ⊕, etc.)
\usepackage{graphicx}                                       % Einfügen und Skalieren von Bildern
\usepackage{fancyhdr}                                       % Benutzerdefinierte Kopf- und Fußzeilen
\usepackage{tablefootnote}

%:: Wissenschaftlicher Apparat
\usepackage{hyperref}                                       % Hyperlinks für Referenzen, Inhaltsverzeichnis, etc.
\usepackage[toc,acronym,nomain,nonumberlist]{glossaries}    % Glossar- und Abkürzungsverzeichnis

\usepackage[
	backend=biber,
	style=authoryear
]{biblatex}                                             % Literaturverwaltung mit Biber und Autor-Jahr-Stil

%:: Präambel

\pagestyle{fancy}
\lhead{\nouppercase\leftmark}
\chead{ }
\rhead{\thepage}
\cfoot{ }

\renewcommand{\familydefault}{\sfdefault}

\setlength{\parindent}{0pt}

\bibliography{bibliography.bib}
\loadglsentries{./acronyms.tex} % Einbinden des Glossars fuer Akronyme
\usepackage{geometry}

\renewcommand{\maketitle}[5]{
	\begin{titlepage}
		\newgeometry{left=4cm, right=2cm, top=2cm, bottom=2cm}
		\begin{center}
			\begin{center}
				\includegraphics[width=0.5\linewidth]{logo.png}
			\end{center}
			\vspace{.5cm}
			\begin{Large}
				\textbf{FOM Hochschule für Oekonomie \& Management}
				\begin{small}
					\\ Hochschulzentrum Frankfurt am Main
				\end{small}
			\end{Large} \\
			\vspace{1.5cm}
			\begin{large}
				\textbf{Bachelor-Thesis} \\
				\begin{small}
					im Studiengang Wirtschaftsinformatik
				\end{small}
			\end{large} \\
			\vspace{1.0cm}
			\begin{small}
				zur Erlangung des Grades eines \\
				\vspace{.15cm}
				\begin{Large}
					Bachelor of Science (B.Sc.)
				\end{Large}
			\end{small} \\
			\vspace{1.0cm}
			\begin{small}
				über das Thema \\
				\vspace{.15cm}
				\begin{large}
					\textbf{#2}
				\end{large}
			\end{small} \\
			\vspace{1.0cm}
			von \\
			#1
		\end{center}
		\vspace*{\fill}
		\begin{tabular}{l @{ : } l}
			Erstgutachter & #3 \\
			Matrikelnummer & #4 \\
			Abgabedatum & #5 \\
		\end{tabular}
		\restoregeometry
	\end{titlepage}	
} 

\makeglossaries

\begin{document}
    \maketitle{Marvin Künzel}{Informationssicherheit in deutschen Unternehmen – Anforderungen und Umsetzungsoptionen im Kontext des KRITIS-Dachgesetzes und NIS2-Umsetzungsgesetzes}{Oliver Bach M.Sc.}{587486}{01-01-1970}
    \pagenumbering{roman}
    % \addtocounter{table}{-1}    % TODO: Dokumentieren
    \clearpage

    %:: Abschnitt: Verzeichnisse
    \newpage
    \tableofcontents
        % \addcontentsline{toc}{section}{Abbildungsverzeichnis}
	    % \listoffigures
	\newpage
    \addcontentsline{toc}{section}{Tabellenverzeichnis}
	\listoftables
	\newpage
    \printglossary[type=\acronymtype, title=Abkürzungsverzeichnis, toctitle=Abkürzungsverzeichnis]

    %:: Abschnitt: Einleitung
    \newpage
    \pagenumbering{arabic}
    \section{Einleitung}
    Im folgenden Kapitel sollen zunächst die Beweggründe für diese Arbeit widergespiegelt werden. Anschließend erfolgt eine Einordnung auf welchen Rahmen sich diese Arbeit bezieht um darauf folgend auf die spezifische Zielsetzung einzugehen. Abschließend wird der generelle Aufbau dieser Arbeit umrissen um die einzelnen Kapitel grob zusammenzufassen.
    \subsection{Motivation}\label{sec:Einleitung_Motivation}
        Der zunehmende Fachkräftemangel erschwert es \gls{kmu} angemessene Maßnahmen zur physischen Resilienz sowie der Informationssicherheit zu treffen. Mit den Richtlinien 2022/2555 und 2022/2557 verpflichtet die \gls{eu} Ihre Mitgliedstaaten auf nationaler Ebenen Gesetze zu erlassen, welche wiederum Unternehmen verpflichtet entsprechende Maßnahmen zu ergreifen. Sowohl kritische Einrichtungen sowie dessen Zulieferer sind somit verpflichtet neue Mindeststandards einzuhalten und zu implementieren. Zwar wurden die Richtlinien auf nationaler Ebene noch nicht vollständig in geltendes Recht überführt, mit dem \gls{nis2umsucg} sowie dem \gls{kritis-dachg} liegen jedoch konkrete Gesetzesentwürfe vor.
        
        Das stellt \gls{kmu} vor eine Herausforderung, da diese mit weniger Personellen und Finanziellen Ressourcen teilweise dieselben regulatorischen Anforderungen unterliegen, wie große Unternehmen. Dies verdeutlicht die Notwendigkeit von praxisnahen Empfehlungen wie diese regulatorischen Anforderungen umzusetzen sind, um \gls{kmu} entsprechend zu entlasten und deren Rechtskonformität und Wettbewerbsfähigkeit zu gewährleisten.
    \subsection{Abgrenzung}
        Wie bereits zuvor in Abschnitt \ref{sec:Einleitung_Motivation} erwähnt setzt diese Arbeit \gls{kmu} in den Fokus. Mit mehr als 50\% der Unternehmen in Deutschland ist dies die größte Gruppe hinsichtlich betroffener Unternehmen. In Tabelle \ref{tbl:definition-kmu} sind die entsprechenden Vorraussetzungen für die Klassifikation eines Unternehmens beschrieben. 
        \begin{table}[ht]
            \caption[Definition von kleinen und mittleren Unternehmen]{Definition von kleinen und mittleren Unternehmen\footnotemark}
            \label{tbl:definition-kmu}
            \resizebox{\textwidth}{!}{
                \begin{tabular}{llll}
                \hline
                Unternehmen & Anzahl Beschäftigte & Umsatz Mio. Euro pro Jahr & Bilanzsumme Mio. Euro pro Jahr \\ 
                \hline
                kleinst     & \(\leq\) 9            & \(\leq\) 2                  & \(\leq\) 2                       \\
                klein       & \(\leq\) 49           & \(\leq\) 10                 & \(\leq\) 10                      \\
                mittel      & \(\leq\) 249          & \(\leq\) 50                 & \(\leq\) 43                      \\ 
                \hline
                \end{tabular}
            }
        \end{table}\footnotetext{\footcite[Vgl.][Anhang, Titel 1, Artikel 2]{l124-36}} \\
        Hierbei stehen die Parameter nicht ausschließlich in einer logischen \(UND\) Beziehung zueinander, sondern sind vielmehr wie folgt zu Interpretieren.
            \[
            \text{Besch\"aftigte} \;\land\;
            \bigl( \text{Umsatz} \oplus \text{Bilanzsumme} \bigr)
            \]
        Das Ergebnis richtet sich somit ausschließlich an Unternehmen, die weniger als 250 Beschäftigte haben und entweder eine Bilanzsumme von unter 43 Millionen Euro oder einen Jahresumsatz von unter 50 Millionen Euro aufweisen.
    \subsection{Zielsetzung}
        \emph{Ergebnis der Arbeit soll ein Fragenkatalog mit Handlungsempfehlungen (Musterantworten) sein. Diesen sollen Unternehmen verwenden können um zum einen Ihre aktuelle Stellung einschätzen zu können. Ebenfalls sollen die Musterantworten den Unternehmen helfen existierende Lücken schneller zu schließen.}
    \subsection{Aufbau der Arbeit}
        \emph{Der Aufbau der Arbeit soll kurz beschrieben werden, sodass der Leser den Umriss der einzelnen Kapitel aber auch den Umfang ein- und abschätzen kann.}


    %:: Abschnitt: Methodisches Vorgehen
    \newpage
    \section{Methodisches Vorgehen}
        \emph{Kurze Zusammenfassung des Kapitel}
        \subsection{Systematische Literaturrecherche}
            \emph{Was ist die systematische Literaturrecherche?}
            \subsubsection{Vorgehen und Auswahlkriterien}
                \emph{Beschreibung des Ausgangspunktes und der daraus folgenden Suche weiterer Literatur. Im Fokus soll stehen wo gesucht wurde (Suchmaschinen, Organisationen etc.), warum dort gesucht wurde (Kontext hinsichtlich der Relevanz) und wie dort gesucht wurde (Abstrakte Beschreibung der relevanten Schlüsselworte, welche sich aus den Richtlinien implizit oder Explizit ergeben.)}
            \subsubsection{Herleitung des Fragenkatalogs}
                \emph{Beschreibung der Struktur des Fragenkatalogs (Basis sind die Richtlinien) sowie die Ableitung / Definition der genauen Fragen.}
        \subsection{Deduktiver Erkenntnisweg}
            \emph{Beschreiben was der deduktive Erkenntnisweg ist und warum dieser für die Ausgangslage der Arbeit relevant ist. Zusätzlich kann hier Bezug auf die hermeneutische Komponente genommen werden.}
        \subsection{Induktive Verifikation}
            \emph{Beschreiben was in dem Kontext der Arbeit die induktiv empirische Verifikation ist und warum diese erforderlich ist. Hier kann ein Bezug auf die interpretation von Gesetzen genommen werden (subjektives Verständnis) und inwieweit die Verifikation des Ergebnis eine Validierung ist.}
            \subsubsection{Fallstudie}
                \emph{Beschreibung was eine Fallstudie ist und was genau sowie in welcher Form dies im Rahmen der Arbeit Anwendung findet.}

    
    %:: Abschnitt: Theoretischer Rahmen
    \newpage
    \section{Theoretischer Rahmen}
        Im Rahmen des Kapitel des theoretischen Rahmen werden alle Grundlagen und definition getroffen, auf welchem diese Arbeit beruht. Dies ist die fundamentale Basis der Herleitung des Fragenkatalog beschrieben in Abschnitt \ref{sec:HerleitungDesFragenkatalog} und Ordnet darüberhinaus diese Arbeit in den aktuellen Forschungsstand ein.
        \subsection{Aktueller Forschungsstand}
            \emph{Die Richtlinien sind jeweils in noch kein national gültiges Recht überführt worden und gelten nur als Entwürfe. Demnach kann und gibt es noch keine konkret gültigen Publikationen. Abseits davon wurde sich bereits auf verschiedenen Ebenen mit dem Thema beschäftigt (Referenz zu relevanten Werken oder Organisationen (wie OPEN KRITIS)) und auch viele Aspekte sind in lange existierenden Standards (Referenz auf relevante Standards wie ISO27001, etc.) Aufgefasst. Konkrete Handlungsempfehlungen in der Form gibt es noch nicht ($\leftarrow$ diese Aussage muss geprüft werden.)}
        
        \subsection{Begriffsdefinitionen}
        Im Rahmen dieser Arbeit werden diverse Begriffe verwendet. Um ein einheitliches VVerständnis zu schaffen ist zunächst die definition grundlegend relevanter Begriffe erforderlich. Im folgenden Abschnitt werden alle Relevanten Begriff hinsichtlich Ihrer Bedeutung definiert.       
            \subsubsection{Kritische Anlagen und kritische Dienstleistungen}
            Eine kritische Anlage ist eine Anlage, welche für die Erbringung einer kritischen Dienstleistung erheblich ist. Eine kritische Dienstleistung wiederum ist eine Dienstleistung zu Versorgung der Allgemeinheit, dessen Ausfall oder Beeinträchtigung erhebliche Versorgungsengpässe oder die öffentlichen Sicherheit gefährden würde. Nach §56 Absatz 4 sind die kritischen Dienstleistungen nach §2 Nummer 24 für die genannten Sektoren durch das Bundesministerium des Innern und für Heimat in Einvernehmen mit diversen Ministerien durch Rechtsverordnungen zu bestimmen. Demnach gibt es aktuell noch keine spezifische definition von kritische Anlagen.\footnote{
                \footcite[§2 Nummer 22 und 24 sowie §56 Absatz 4][]{NIS2UmsuCG}
            }

            \subsubsection{Betreiber kritischer Anlagen}
            Eine natürliche oder juristische Person sowie unselbständige Organisationseinheit einer Gebietskörperschaft, welche bestimmenden Einfluss auf eine oder mehrere kritische Anlagen ausübt. Dies ist unter Berücksichtigung der rechtlichen, wirtschaftlichen und tatsächlichen Umstände. Im Sektor des Finanzwesen ist alleine die tatsächliche Sachherrschaft  ausschlaggebend.\footnote{
                \footcite[§28 Absatz 7][]{NIS2UmsuCG}
            }

            \subsubsection{Vertrauensdienst, Vertrauensdiensteanbieter und qualifizierter Vertrauensdiensteanbieter}
            Ein Vertrauensdiensteanbieter ist ein Anbieter, welcher Vertrauensdienste bereitstellt. Ein Vertrauensdienst ist im weitesten Sinne ein elektronischer Dienst, welcher einer der folgenden Kategorien zuzuordnen ist:
            \begin{itemize}
                \item Die Handhabung (Erstellung, Überprüfung und Validierung) von
                \begin{itemize}
                    \item elektronischen Signaturen
                    \item elektronischen Siegel
                    \item elektro­nischen Zeitstempel
                    \item Zertifikaten für die Website-Authentifizierung
                \end{itemize}
                \item Zustellung elektronischer Einschreiben
                \item Aufrechterhaltung von den Diensten betreffend der elektronischen Signaturen, Siegeln oder Zertifikaten
            \end{itemize}
            Qualifiziert ist der Vertrauensdienst, wenn er den Anforderungen der Verordnung (EU) Nr. 910/2014 genügt. Dies ist für die Bundesrepublik Deutschland im \gls{vdg} geregelt.

            Ein Anbieter ist dann als qualifizierter Vertrauensdiensteanbieter zu Betrachtet wenn dieser qualifiziert Vertrauensdienste anbietet und von der in dem Land zuständigen Aufsichtsstelle entsprechend eingestuft wurde. Für die Bundesrepublik Deutschland ist nach §2 des \gls{vdg} Entsprechend die Bundesnetzagentur und nachgelagert das \gls{bsi} zuständig.\footnote{
                \footcite[Vgl. Artikel 3, Nummer 16, 17, 19 und 20][]{EU910-2014}
                \footcite[Vgl. §2][]{VDG}
            }

            \subsubsection{Telekommunikationsdienste}
            Ein Telekommunikationsdienst ist im Allgemeinen ein Internetzugangdienst, interpersonelle Telekommunikationsdienste oder jeder Dienst, der ganz oder überwiegend in der Übertragung von Signalen über Telekommunikationsnetze besteht. Öffentlich ist dieser Telekommunikationsdienste wenn er einem unbestimmten Personenkreis zur verfügung steht.\footnote{
                \footcite[Vgl. §3 Nummer 60][]{TKG}
            }

            \subsubsection{Telekommunikationsnetz}
            Ein Telekommunikationsnetz ist allgemein die Gesamtheit aller Übertragungssysteme um Informationen auszutauschen. Ein öffentliches Telekommunikationsnetz wiederum ist ein Telekommunikationsnetz (nach Satz 1), welche die Erbringung von öffentlich zugänglicher Telekommunikationsdiensten im Sinne der Übertragung von Informationen von Netzabschlusspunkten dienen. Ein Netzabschlusspunkten wiederum ist der physische Punkt an welchem ein Endnutzer, also eine natürliche oder juristische Person, die einen öffentlich zugänglichen Telekommunikationsdienst für private oder geschäftliche Zwecke in Anspruch nimmt und weder öffentliche Telekommunikationsnetze betreibt noch öffentlich zugängliche Telekommunikationsdienste erbringt.\footnote{
                \footcite[Vgl. §3 Nummer 13, 32, 41, 42, 61 und 65][]{TKG}
            }

            \subsubsection{Kritische Komponente}
            Eine kritische Komponente definiert sich aus verschiedenen Bestandteilen. Zunächst muss fundamental der Begriff des elektronischen Kommunikationsnetzes definiert werden. Dies ist eine einzelne oder eine Gruppe aus aktiven oder passiven Ressourcen, welche, unabhängig des Mediums, zur Übertragung von Signalen zum Austausch von Informationen verwendet werden. Dies wiederum ist Bestandteil der Definition von Netz- und Informationssystemen. Fokussiert man sich zunächst auf den Begriff des Informationssystems kann man dies als Gruppe von Anwendungen, Diensten, informationstechnischen Anlagen oder anderen Komponenten für die Informationsverarbeitung definieren. Allgemein gültiger ist ein Informationssystemen eine Vorrichtung oder die Gruppe von Vorrichtungen, welche die automatische Datenverarbeitung unter Grundlage eines Programmes durchführt. Somit definiert sich das Netz- und Informationssystemen aus der Definition von elektronischen Kommunikationsnetzen und Informationssysteme, sowie den Daten, welche zum Zweck des Betriebes eines elektronisches Kommunikationsnetz oder Informationssystem gespeichert, verarbeitet, abgerufen oder übertragen werden. Eine kritischen Komponente wiederum ist ein \gls{ikt-produkt}. Dies ist ein ein Element oder eine Gruppe aus Elementen eines Netz- und Informationssystem. Kritisch wird die Komponente dadurch das diese in einer kritischen Anlagen eingesetzt und unter Grundlage des \gls{nis2umsucg} als kritische Komponente bestimmt wird oder eine kritische Funktion realisiert.\footnote{
                \footcite[Vgl. §2 Nummer 23][]{NIS2UmsuCG}
                \footcite[Vgl. Artikel 4, Nummer 1][]{EU2016-1148}
                \footcite[Vgl. Artikel 2, Buchstabe a][]{EU2002-21-EG}
                \footcite[Vgl. Artikel 2, Nummer 12][]{EU2019-881}
                \footcite[Vgl. S. 5][]{iso27000-2018}
            }

            \subsubsection{Risikomanagement}

            \subsubsection{Sicherheitsvorfall}
            \emph{auch erheblich....}
        
        \subsection{NIS-2-Umsetzungs- und Cybersicherheitsstärkungsgesetz (NIS2UmsuCG)}
            \subsubsection{Betroffene Organisationen und Subjekte}
            \subsubsection{Zuständige Aufsichtsbehörden}
            \subsubsection{Erforderliche Maßnahmen}
                Im Rahmen von \gls{nis2umsucg} sind diverse Maßnahmen beschrieben, welche betroffene Einrichtungen umsetzen müssen. Darüberhinaus gibt es mit der \emph{Durchführungsverordnung (EU) 2024/2690} definierte Anforderungen an Maßnahmen in Hinsicht auf die technischen und methodischen Anforderungen von \gls{dns}-Diensteanbieter, \gls{tld}-Namenregister, Anbieter von Cloud-Computing-Diensten, Anbieter von Rechenzent­rumsdiensten, Betreiber von \gls{cdn}, Anbieter verwalteter Dienste, Anbieter verwalteter Sicherheitsdienste, Anbieter von Online-Marktplätzen, Online-Suchmaschinen und Plattformen für Dienste sozialer Netzwerke und Vertrauensdiensteanbieter. Der Folgende Abschnitt Spiegelt sämtliche Anforderungen, welche aus dem \gls{nis2umsucg} sowie der Durchführungsverordnung hervorgehen. Ebenfalls werden alle von \gls{nis2umsucg} und der Durchführungsverordnung betroffenen Organisationen und Subjekte einheitlich als betroffene Einrichtungen bezeichnet. Zunächst wird der Fokus ausschließlich auf den Theoretischen Rahmen gesetzt, jedoch in allen dienlichen Fällen Bezug auf den entsprechenden Abschnitt der praktischen Maßnahmen, beschrieben in Abschnitt \ref{sec:ParktischeMaßnahmen}, genommen.\footnote{
                    \footcite{MISSING}
                }\medbreak
                
                Sämtliche beschriebene Anforderungen der \emph{Durchführungsverordnung (EU) 2024/2690} sind in aller Regel in einem zu definierenden Turnus zu Überprüfen und das Ergebnis zu dokumentieren, spätestens jedoch nach einem erheblichen Sicherheitsvorfall, einer erneuten Risikobewertung oder aber der wesentlichen  Änderungen  der  Betriebsabläufe.\medbreak

                % Nummer 1: ISMS ↓

                Teil der zu ergreifenden Maßnahmen ist ein \emph{Konzept für die Sicherheit von Netz- und Informationssystemen} sowie darin enthaltenen themenspezifischen Konzepte, welche sich auf die betroffene Einrichtung beziehen. Zentraler Bestandteil ist die Festlegung der Verantwortlichkeiten sowie Weisungsbefugnisse hinsichtlich der Sicherheit von Netz- und Informationssysteme. Diese Festlegung wird ebenfalls den Leitungsorganen mitgeteilt, sodass ein einheitliches Verständnis der definierten Verantwortlichkeiten und Weisungsbefugnisse herrscht. Sofern es möglich ist sollen wiedersprechende Verantwortlichkeiten vermieden werden. Zusätzlich muss im selben Rahmen eine Person definiert werden, welche den Leitungsorganen gegenüber als Ansprechpartner für Fragen bezüglich der Sicherheit von Netz- und Informationssystemen verantwortlich ist. Das Konzept an sich muss diverse Themengebiete abdecken. Ein Kernapsekt ist die Beschreibung des Ansatzes des Managements der Sicherheit in Netz- und Informationssystemen. In diesem Kontext bezieht sich Management auf die Verwaltung der Verfahren und Regeln für die Sicherheit in Netz- und Informationssystemen. Wichtig ist das die Geschäftsstrategie durch das Konzept ergänzt und für die Unternehmerischen Ziele geeignet sind. Ebenfalls ist ein Konzept genau wie die Einrichtung nicht statisch sondern dynamisch. Daher bedarf es einer Verpflichtung im Rahmen des Konzeptes dieses Kontinuierlich zu verbessern und erweitern. Ebenfalls bedarf es einer Verpflichtung die Ressourcen für die eingangs beschriebenen Rollen sowie damit verbundenen Finanzen, Verfahren, Instrumente und Technologien bereitzustellen. Diese Festgelegten Rollen sind ebenfalls Teil des Konzeptes. Zusätzlich sind sämtliche betroffenen aufzubewahrenden Unterlagen und die Dauer ihrer Aufbewahrung aufführen. Im Kontext der \gls{brd} sind hier unter anderem die \gls{ao}, das \gls{hgb} sowie die \gls{gobd} maßgeblich. Abschließend muss entsprechend beschrieben sein wie die Wirksamkeit hinsichtlich der Umsetzung des Konzeptes überwacht wird. Das Konzept an sich muss allen betroffenen Mitarbeitern sowie interessierten externen Beteiligten verfügbar gemacht werden. Ebenfalls ist dies durch diese Personengruppen anzuwenden. \footnote{
                    \footcite[Vgl. Anhang, Nummer 1][]{EU2024-2690}
                    \footcite[Referenz für Unternehmen die sich Wandeln][]{MISSING}
                }\medbreak

                % Nummer 12: Anlagen- und Wertemanagement | ITAM, Anlagenverzeichnis, Prozessmanagement Platform, DMS, CMDB, Monitoring
                % \emph{Anlagen- und Wertemanagement}
                Das \emph{Konzept für die Sicherheit von Netz- und Informationssystemen} billeted unter anderem ein zentrales Gerüst für das \emph{Anlagen- und Wertemanagement}. Erforderliche ist es ein vollständiges Verzeichnis über die Anlagen und Werte zu erstellt, welches ein aktuelles und kohärentes Inventar der Anlagen und Werte wiederspiegelt. Die Granularität hängt von den Bedrüfnissen der Einrichtung ab und umfasst zumindest die Liste der Betriebsabläufe und Dienste und ihre Beschreibung sowie eine Liste der Netz- und Informationssysteme und anderer zugehöriger Anlagen und Werte, die die Abläufe und die Dienste der betreffenden Einrichtungen unterstützen. Sämtliche Änderungen sind nachvollziehbar zu Dokumentieren. Alle Anlagen und Werte inklusive Informationen des Bereich Netz- und Informationssysteme müssen mittels eines System Klassifizierungsstufen für die erforderlichen Schutzniveau zugeordnet werden. Dieses System beruht auf Vertraulichkeits-, Integritäts-, Authentizitäts- und Verfügbarkeitsanforderungen um das Schutzniveau auf Basis der Sensibilität, ihrer Kritikalität, ihres Risikos und ihres Geschäftswerts zuzuordnen. Die          Verfügbarkeitsanforderungen leiten sich aus Notfallplan ab, welcher die Aufrechterhaltung und Wiederherstellung des Betriebs unter bestimmten Bedingungen gewährleisten soll. Ebenfalls Teil des \emph{Anlagen- und Wertemanagement} ist ein Konzept zur ordnungsgemäße Behandlung von Anlagen und Werten die allen betroffenen Personen bekannt sind und auf dem Konzept für die Sicherheit der Netz- und Informationssysteme beruht. Hierbei werden Aspekte des gesamten Lebenszyklus der Anlagen und Werte (Erwerb, Verwendung, Speicherung, Transport und Entsorgung) abgedeckt und die sichere Verwendung, Speicherung und der Transport sowie die Unwiederbringliche Löschung und Vernichtung definiert. Bei Beendigung des Beschäftigungsverhältnisses müssen die betreffenden Anlagen und Werte abgegeben, zurückgegeben oder gelöscht werden, und die Abgabe, Rückgabe und Löschung dokumentiert werden. Ist das nicht möglich muss sichergestellt sein das der Zugriff auf Netz- und Informationssysteme nicht mehr möglich ist. Gesondert wird ein Konzept für das Management von Wechseldatenträgern und wie diese an Orten, an dem die Wechseldatenträger mit den Netz- und Informationssystemen der betreffenden Einrichtungen verbunden sind, zu nutzen sind gefordert. Bestandteil hiervon sind die Sperrung von nicht notwendigen Wechseldatenträger sowie das Scannen auf Schadcode und Verhindern von Programmen, welche sich automatisch ausführen. Notwendige Wechseldatenträger sind zu Verschlüsseln und Maßnahme zur Kontrolle und dem Schutz einzuführen.\footnote{
                    \footcite[Vgl. Anhang, Nummer 12][]{EU2024-2690}
                }\medbreak

                % Nummer 2: Risikomanagement ↓    

                Das \emph{Konzept für die Sicherheit von Netz- und Informationssystemen} sowie das \emph{Anlagen- und Wertemanagement} sind Grundlage für das \emph{Konzept für das Risikomanagement}. Risiken müssen kontinuierlich Identifizierung, Analysiert, Bewertet, Bewältigt und überwacht werden. Ein Essenzieller Kernaspekt ist die Tatsache das sich das Risikomanagement für die Sicherheit von Netz- und Informationssystemen im Einvernehmen mit dem ganzheitlichen Risikomanagement der betroffenen Einrichtung befindet. Bestandteile des Risikomanagement sind der Risikomanagementprozess zur Identifizierung, Analyse, Bewertung, Bewältigung und Überwachung von Risiken hinsichtlich der Sicherheit von Netz- und Informationssystemen. Hierbei werden Risikomanagementmethodiken auf die ermittelten Risiken angewendet um eine Risikostrategie abzuleiten. Eine Zentrale Rolle spielt die Risikoüberwachung sowie die Risikokommunikation. Formal ist es erforderlich für den Risikomanagementprozess Verfahren zu definieren die Risikomanagementmethodiken sowie die Risikotoleranzschwelle im Einklang mit der Risikobereitschaft umfassen. Der zuvor beschriebene kontinuriliche Prozess der Identifizierung, Analyse, Bewertung und Behandlung von Risiken beinhaltet ebenfalls die Risikokriterien, welche ebenfalls für die Überwachung und eine kontinuriliche Verbesserung herangezogen werden. Neben den Verfahren zur Bewertung von Risiken und den entsprechend daraus folgenden Maßnahmen sind ebenfalls Verfahren zu definieren, welche Personen die Verantwortlichkeit für die Terminierte Umsetzung der Maßnahmen zuordnet. Zusätzlicher Bestandteil des Risikomanagement für die Sicherheit von Netz- und Informationssystemen ist die regelmäßige und unabhängige Überprüfung der \emph{Konzept für die Sicherheit von Netz- und Informationssystemen} inklusive geeigneter Systeme zur Berichterstattung der Ergebnisse für die Leitungsorganen. Diese sollen somit einen fundierten Überblick über den aktuellen Stand des Risikomanagements abrufen können.\footnote{
                    \footcite[Vgl. Anhang, Nummer 2][]{EU2024-2690}
                    \footcite[Risikomanagement in Unternehmen allgegenwärtig][]{MISSING}                
                }\medbreak

                % Nummer 7: Wirksamkeit von Risikomanagementmaßnahmen ↓

                Um die Ergebnisse des \emph{Konzept für das Risikomanagement} zu bewerten ist ein \emph{Konzepte und Verfahren zur Bewertung der Wirksamkeit von Risikomanagementmaßnahmen im Bereich der Cybersicherheit} erforderlich. Hierfür muss bestimmt werden welche Verfahren und Kontrollen zur Überwachung und Messung der Risikomanagementmaßnahmen angewendet werden. Auch die Methoden zur Überwachung, Messung, Analyse und Bewertung für die Gewährleistung gültiger Ergebnisse ist Teil des Konzept. Die Wirksamkeit muss regelmäßig geprüft werden, weshalb der Turnus sowie die Bedingung für eine Bewertung spezifiziert und die Zeitpunkt und Verantwortlichkeiten bis wann Ergebnisse Analysiert und bewertet werden müssen definiert wird.\footnote{
                    \footcite[Vgl. Anhang, Nummer 7][]{EU2024-2690}
                }\medbreak

                % Nummer 3: Protokollierung, Sicherheitsvorfälle & Monitoring ↓

                Ein Teilaspekt des Risikomanagement ist die prevention von Sicherheitsvorfällen. Nicht jeder Sicherheitsvorfall lässt sich präventiv vermeiden, weswegen Verfahren zur \emph{Bewältigung von Sicherheitsvorfällen} ein erforderlicher Teil der Maßnahmen ist. Vorhergehend ist das Konzept zur Bewältigung von Sicherheitsvorfällen. In diesem werden Rollen, Verantwortlichkeit und Verfahren für die zeitnahe Erkennung, Analyse, Eindämmung oder Reaktion auf Sicherheitsvorfälle sowie die Wiederherstellung nach einem Sicherheitsvorfall festgelegt. Wie ein Sicherheitsvorfall zu Melden und bei Bedarf zu eskalieren ist, ist ebenfalls Bestandteil des Konzeptes. Einem Sicherheitsvorfall vorangestellt ist ein Ereignis. Dieses muss zunächst auf Basis eines Kategorisierungs- und Klassifizierungssystems als Sicherheitsvorfall eingestuft werden. Diese beiden Aspekte sind entsprechend ebenfalls Bestandteil des Konzeptes. Damit die Vorgehen widerspruchsfrei sind werden sämtliche Dokumente, wie Verfahren, Anleitungen, Vorlagen und der gleichen ebenfalls als Teil des Konzeptes mit aufgefasst. Dieses Konzept muss entsprechend im Einklang mit dem Notfallplan für die Aufrechterhaltung und Wiederherstellung des Betriebs der Einrichtung stehen. Damit Ereignisse leichter erkannt werden können sind Verfahren und Instrumente für die Überwachung und Protokollierung von Aktivitäten in den Netz- und Informationssystem zu implementieren. Diese funktionieren soweit möglich automatisch und basieren auf einem einheitlichen Zeitgeber, sodass die Protokolle aus diversen Systemen in eine Korrelation gebracht werden können. Hierbei ist die Überwachung und Protokollierung nur auf Systeme anzuwenden, welche per Definition gemäß der Anlagen und Werte auf Basis der Risikobewertung ein entsprechendes Vorgehen erforderlich machen. Die angefertigten Protokolle werden entsprechend eines definierten Zeitraumes Archiviert. Die Protokolle umfassen die folgenden Ereignisse:\footnote{
                    \footcite[Vgl. Anhang, Nummer 3.1 - 3.2.3, 3.2.5 \& 3.2.6][]{EU2024-2690}
                    \footcite[Risikomanagement soll Sicherheitsvorfällen präventiv vermeiden][]{MISSING}
                    \footcite[Sicherheitsvorfälle kann man nicht immer vermeiden][]{MISSING}
                }
                \begin{itemize}
                    \item Logischer und physikalischer Zugang (Ein- und Ausgehend)
                    \item Ereignisprotokolle und Protokolle von Sicherheitslösungen wie beispielsweise Firewalls oder Endpunktsicherheitslösungen
                    \item Erstellen, Ändern und Löschen sowie Erweiterung der Rechte von Benutzerkonten
                    \item Zugriffe auf Systeme und Anwendungen sowie die Berechtigungsstufe (privilegierter Zugriff)
                    \item Ereignisse mit Bezug auf Authentifizierung
                    \item Sämtliche Aktivitäten in Bezug auf privilegierte Benutzerkonten (Verwaltungskonten)
                    \item Ereignisse des Umfelds der Systeme
                    \item Systemressourcen
                    \item Aktivierung, Beendigung und Pausieren der verschiedenen Protokolle
                \end{itemize} 
                Diese Protokolle werden regelmäßig auf Trends untersucht und es werden auf Basis geeigneter Schwellwerte Alarme ausgelöst. Ereignisse können auch durch Meldung Dritte, beispielsweise Personal, Kunden oder aber auch Anbieter, erfolgen. Hierzu ist ein entsprechender Mechanismus bereitzustellen und, bei Bedarf, die Betroffenen Personen zu unterrichten. Sowohl bei automatisiert (Protokollierung) erfassten als auch bei manuell (Meldung) erfassten Ereignissen sind Vorgehen erforderlich um zu ermittelten ob bei dem Ereignis ein Sicherheitsvorfall vorliegt. Für diese Feststellung ist ein Vorgehen zu definieren, welches die Bewertung auf Grundlage festgelegt Kriterien ermöglicht. Diese Kriterien bieten dann die Möglichkeit auf Basis von Meldungen oder Protokollen sowie die Korrelation und Analyse von Protokollen ein Ereignis zu Bewerten und bei Bedarf als Sicherheitsvorfall einzustufen. Ebenfalls Teil des Vorgehens sind die Durchführung einer Triage um die Gesamtheit der Eindämmung und Beseitigung der Sicherheitsvorfälle zu priorisieren. Auch eine erneute Bewertung von Ereignissen ist vorgesehen, etwa dann wenn neue Informationen vorliegen oder bereits vorliegende Informationen dies erforderlich machen. Eine vierteljährliche Bewertung ob wiederholte Ereignisse vorliegen ist der Abschließende Bestandteil der zu definierenden Vorgehen. Hiermit sollen kumuliert sich wiederholende Ereignisse betrachtet werden um diese ebenfalls als Sicherheitsvorfall einstufen zu können. Wird ein Ereignis als Sicherheitsvorfall klassifiziert ist dies Zeitnahe und gemäß Dokumentierter Verfahren in den drei Phasen der Eindämmung, Beseitigung und Wiederherstellung zu bearbeiten. Ein essenzieller Bestandteil ist die erstellung von Plänen und verfahren zur Kommunikation mit dem \gls{csirt} oder den zuständigen Behörden im Zusammenhang mit der Meldung von Sicherheitsvorfällen. Auch die in- und externen Kommunikationswege für Personal und Interessenträger außerhalb der Einrichtung ist ein Wichtiger Bestandteil. Jegliche Reaktion sowie die Tätigkeiten im Zusammenhang mit einem Sicherheitsvorfall sind zu Dokumentieren und diese Verfahren gilt es ebenfalls zu Überprüfen. Alle Sicherheitsvorfälle müssen Abschließend hinsichtlich ihrer Ursachen überprüft werden um diese in Zukunft zu vermeiden und bei Bedarf Schulungsmaßnahmen aus diesem Sicherheitsvorfall abzuleiten. Sämtliche Überprüfung, auch die Überprüfung ob ein Sicherheitsvorfall abschließend überprüft wurde, dienen dem Zwek der kontinuierlichen verbesserung. Dies beschränkt sich in dem Fall nicht ausschließlich auf das Konzept zur \emph{Bewältigung von Sicherheitsvorfällen}, sondern auch auf alle erforderlichen Maßnahmen wie beispielsweise das \emph{Risikomanagement}.\footnote{
                    \footcite[Vgl. Anhang, Nummer 3.2.4 \&1 3.3 - 3.6.2][]{EU2024-2690}
                    \footcite[Vgl. Artikel 4][]{EU2024-2690}
                }\medbreak

                % Nummer 5: Backup, BCM 

                Um den Betrieb während eines Sicherheitsvorfalls zu gewährleisten ist ein entsprechender Notfallplan erforderlich. Dieser ist Teil des \emph{Betriebskontinuitäts- und Krisenmanagement}. Hiermit soll nicht nur die Aufrechterhaltung sondern auch die Wiederherstellung des Betriebs sichergestellt werden. Konkret umfasst der Notfallplan auf Basis der Risikobewertung zunächst einmal den Zweck, Umfang und die Zielgruppe des Plans. Um im Notfall klar definierte Rollen und Verantwortlichkeit definiert zu wissen ist dies ebenfalls Bestandteil des Notfallplan. Neben der Reihenfolge der Wiederherstellung der Betriebsabläufe sind auch die hierfür erforderlichen Ressourcen, einschließlich Sicherungen und Redundanzen, Bestandteil des Plans. Damit Kommunikationskanäle klar definiert sind werden alle wichtige Kontaktangaben und (interne und externe) Kommunikationskanäle ebenfalls Dokumentiert. Zusätzlich sind die Informationen unter welchen Bedingungen der Notfallplan aktiviert- und wieder deaktiviert werden kann essenziell für dessen Anwendung. Diese Definitionen Bilden den Notfallplan für die jeweilige Einrichtung. Eine regelmäßige Erprobung ist ebenfalls Bestandteil des \emph{Betriebskontinuitäts- und Krisenmanagement}. Zur Gewährleistung einer Entsprechender Resilienz sind auf Basis der Auswirkung einer Störung auf die Betriebsabläufe die Kontinuitätsanforderungen für die Netz- und Informationssysteme abzuleiten. Aus den Kontinuitätsanforderungen kann man wiederum das anfertigen von Sicherungskopien Ableiten. Auch das erstellen von Sicherungspläne auf Basis der Risikobewertung und des Betriebskonti­nuitätsplans sind Teil der Maßnahmen des \emph{Betriebskontinuitäts- und Krisenmanagement}. In jedem fall enthält ein Sicherungspläne die Wiederherstellunggszeit sowie das vorgehen bei Wiederherstellung, Aufbewarhungsfristen, Integritätsprüfungen, phyische und logische Zugangskontrolle entsprechender Klassifizierunggsstufen der Anlagen und Werte sowie den Speicherort. Der Speicherort ist entsprechend so zu wählen das bei einem Ausfall der zu sichernden Systeme die Sicherungskopien nicht betroffen sein können. Ebenfalls ist bei den Aufbewarhungsfristen Regulatorische Anforderungen zu beachten, wie diese beispielsweise aus der \gls{ao} hervorgehen. Unabhängig der Kontinuitätsanforderungen sind der Risikobewertung entsprechende Netz- und Informationssysteme, Kommunikationskanäle sowie Anlagen und Werte redundant oder teilweise redundant auszulegen. Auch Personal muss hinsichtlich der Verantwortlichkeit, Weisungsbefugnis und Kompetenz entsprechend resilient sein. Für schwerwiegende Ereignisse sind Verfahren für das Krisenmanagement zu etablieren. Diese umfassen mindestens die Rollen und Verantwortlichkeiten des Personals und der Anbieter sowie die geeignete Kommunikationsmittel zwischen den betreffenden Einrichtungen  und den jeweils zuständigen Behörden. Ebenfalls sind die Maßnahmen zur Gewährleistung der Aufrechterhaltung der Sicherheit von Netz- und Informationssystemen in Krisensituationen fundamentaler Bestandteil des Krisenmanagement. Einhergehend mit dem \emph{Konzept für das Risikomanagement} sind die Verfahren zur Verwaltung von Informationen über Sicherheitsvorfälle, Schwachstellen, Bedrohungen oder mögliche Risikominderungsmaßnahmen, welche ebenfalls dem \emph{Betriebskontinuitäts- und Krisenmanagement} zuzuordnen sind.\footnote{
                    \footcite[Vgl. Anhang, Nummer 4][]{EU2024-2690}
                    \footcite[Vgl. Artikel 22, Absatz 1][]{EU2022-25555}
                }\medbreak

                % Sicherheit der Lieferkette

                Ein wesentlicher Bestandteil externer Einflüsse sind die Lieferketten. Dementsprechend ist es erforderlich ein dienliches Konzept für \emph{Sicherheit der Lieferkette} Aufzuweisen. Zusätzlich ist es essenziell die eigene Rolle und die Beziehung zu direkten Anbietern und Diensteanbietern zu evaluieren und sich dieser Bewusst zu werden.Kernforderung an das Konzept für \emph{Sicherheit der Lieferkette}  ist die Definition von Kriterien für die Auswahl von Anbietern und Diensteanbietern sowie die Auftragsvergabe. Hierbei ist es wichtig das die Lieferanten Cybersicherheitsverfahren einschließlich der Sicherheit ihrer Entwicklungsprozessen etabliert haben. Die allgemeine Qualität und Resilienz der angebotenen und entwickelten IKT-Produkte und -Dienste sowie Risikomanage­mentmaßnahmen im Bereich der Cybersicherheit sind ebenfalls ein wichtiger Bestandteil des Konzeptes. Zur Vermeidung von Abhängigkeiten soll zusätzlich Wert auf die Fähigkeit die Versorgungsquellen zu diversifizieren und Abhängigkeit von bestimmten Anbietern gelegt werden. Unter Rücksichtnahme von Risikobewertungen zur \emph{Sicherheit der Lieferkette}  aus der Zusammenarbeit von Kooperationsgruppe mit der Europäischen Kommission und der \gls{enisa} bildet dies die Anforderungen an das Konzept für \emph{Sicherheit der Lieferkette} . Für eine fortwährende Geschäftsbeziehung sehen die erforderlichen Maßnahmen vor eine auf Basis der Risikobewertung und im Rahmen der Leistungsverein­barungen mit Anbietern und Diensteanbietern in Form entsprechender Verträgen diverse Aspekte zu Regeln. Bestandteile davon sind die Cybersicherheitsanforderungen an die Anbieter oder Diensteanbieter sowie die Anforderungen an die Sicherheitsmaßnahmen beim Erwerb von IKT-Diensten oder IKT-Produkten. Ebenfalls unterliegen die Mitarbeitenden der Anbieter und Diensteanbieter Sensibilisierungs-, Qualifikations- und Ausbildungsanforderungen. Die Anbieter und Diensteanbieter verpflichten sich ebenfalls Sicherheitsvorfälle, die ein Risiko für die Sicherheit der Netz- und Informationssysteme darstellen, unverzüglich zu melden sowie Schwachstellen zu Beheben, welche ein Risiko für die Sicherheit der Netz- und Informationssysteme darstellen. Für die Unterauftragsvergabe durch den Anbieter und Diensteanbieter entstehen Cybersicherheitsanforderungen an Unterauftragnehmer, welche mit den Cybersicherheitsanforderungen an den Anbieter und Diensteanbieter vergleichbar sein können. Darüberhinaus werden sämtliche andere Pflichten der Anbieter und Diensteanbieter im Rahmen der Verträge zur Leistungsverein­barungen definiert. Diese zwei wesentlichen Komponenten sollen unmittelbar die \emph{Sicherheit der Lieferkette} gewährleisten. Geschäftsbeziehungen sind dynamisch und können sich wandeln. Daher müssen die Maßnahmen für die \emph{Sicherheit der Lieferkette} überwacht werden. Hierzu zählen, soweit Anwendbar, die durch den Anbietern und Diensteanbietern zu erstellenden Berichte über die Umsetzung der Leistungsvereinbarungen. Ebenfalls sollen Sicherheitsvorfälle im Zusammenhang mit IKT-Produkten und -Diensten von Anbietern und Diensteanbietern überprüft und Risiken, die sich aus Änderungen im Zusammenhang mit IKT-Produkten und -Diensten von Anbietern und Diensteanbietern ergeben, analysieren und falls erforderlich Maßnahmen ergreifen werden. Zusätzlich soll überprüft werden ob eine außerplanmäßiger Überprüfungen notwendig ist und sofern dieser Fall eintritt ist die Prüfung nachvollziehbar zu Dokumentieren. Für eine Übersicht ist ein Verzeichnis der direkten Anbieter und Diensteanbieter mit Kontaktstellen und eine Liste mit bereitgestellten IKT-Produkte, -Dienste und -Prozesse zu erstellen und pflegen.\footnote{
                    \footcite[Vgl. Anhang, Nummer 5][]{EU2024-2690}
                    \footcite[Geschäftsbeziehungen sind dynamisch][]{MISSING}
                }\medbreak

                % Nummer 6: ITAM / CMDB, Netzwersicherheit / Segmentierung, Change Management, MDM, Patch-Management, Vulnerability Management, Mail / Kommunikationssicherheit

                Kommen IKT-Produkten und -Diensten von Anbietern und Diensteanbietern oder durch Eigenentwicklung in den Einsatz sind Vor- aber auch Nachgelagert \emph{Sicherheitsmaßnahmen bei Erwerb, Entwicklung und Wartung von Netz- und Informationssystemen} erforderlich. Allgemein sind Verfahren für die Sicherheit unverzichtbare Netz- und Informationssysteme erforderlich. Somit soll es Sicherheitsanforderungen für die IKT-Dienste und -Produkte. Die Allgemeinen Sicherheitsanforderungen sollen durch Anforderungen an Sicherheitsaktualisierungen während der gesamten Lebensdauer der IKT-Dienste und -Produkte oder dem Ersatz nach Ablauf des Unterstützungszeitraums umfassen. Zusätzlich sind beschreibeungen zu Hard- und Software wie zu den umgesetzten Cybersicherheitsfunktionen erforderlich. Ebenfalls muss eine Zusicherung exisitieren, dass die definierten Sicherheitsanforderungen von dem IKT-Dienste oder -Produkte erfüllt werden. Dies muss demzufolge durch den Erichter des IKT-Dienste oder -Produkte erfolgen. Darüberhinaus müssen Methoden zur Validierung der Sicherheitsanforderungen existieren und dessen Ergebnisse Dokumentiert werden. Dies validiert oder falsifiziert die Zusicherung über die Einhaltung der Sicherheitsanforderungen. Unabhängig der Kritikalität eines Netz- und Informationssysteme sind sowohl für die beauftragte als auch eigenentwickelte Errichtung von Netz- und Informationssystems Verfahren zur Entwicklung, welches die Spezifikation, Konzeption, Entwicklung, Umsetzung und Tests umfassen, festzulegen. Hierfür muss eine Analyse der Sicherheitsanforderungen in der Spezifikations- und Entwurfsphase jedes Entwicklungs- oder Beschaffungsvorhabens einfließen. Allgemein müssen bestimmte Grundsätze für den Aufbau sicherer Systeme und für ein sicheres Programmieren aufgestellt und eingehalten werden. Auch an die Entwicklungsumgebungen müssen Sicherheitsanforderungen sowie Sicherheitstestverfahren im Entwicklungszyklus gestellt und angewendet werden. Sämtliche Daten für Validierungen der Netz- und Informationssystems in entsprechenden Tests sind Basierend den Anforderungen auszuwählen, zu schützen und zu verwalten wie zu bereinigen und anonymisieren. Für den Betrieb eines Netz- und Informationssystems sind den Anforderungen entsprechende Konfigurationen erforderlich. Maßnahmen um Konfigurationen von Hardware, Software, Diensten und Netzen, festzulegen, zu dokumentieren, umzusetzen und zu überwachen sind demnach zu treffen. Ein wichtiger Bestandteil hiervon sind die Verfahren und Instrumente zur Durchsetzung für neu installierte Systeme und Systeme über ihre gesamte Lebensdauer. Durch sich ändernde Anforderungen kann es erforderlich Netz- und Informationssystemen anzupassen. Daher sind Verfahren für das Änderungsmanagement, mit dem Ziel Änderungen an Netz- und Informationssystemen zu kontrollieren, einzuführen. Diese müssen im Einklang mit den Allgemeinen Änderungsmanagement der Einrichtung stehen. Dieses Änderungsmanagement umfasst Verfahren für Freigaben, Änderungen und Notfalländerungen an in Betrieb befindlicher Software und Hardware sowie ebenfalls die Änderungen der Konfiguration. Ein Abweichendes vorgehen, beispielsweise wegen einer Krisensituationen, muss inklusive Begründung Dokumentiert werden. Um regelmäßig den aktuellen Stand der Sicherheit einschätzen zu können sind, sowohl für Wartung als auch Implementation von Netz- und Informationssystemen, Konzepte und Verfahren für Sicherheitsprüfungen festzulegen. Hierbei ist zunächst die Notwendigkeit aber auch der Umfang und die Häufigkeit und die Art auf Basis der Risikobewertung festzulegen. Damit die Sicherheitsprüfungen repetitive durchgeführt werden können muss dies nach definierten Prüfmethoden erfolgen. Sämtliche Aspekte der Sicherheitsprüfung, demnach Art, Umfang, Zeitraum und die Ergebnisse, sind zu Dokumentieren. Ergibt die Sicherheitsprüfung eine Feststellung sind diese inklusive dessen Kritikalität und den Risikominderungsmaßnahmen zu Dokumentieren. Bei kritischen Feststellungen werden Risikominderungsmaßnahmen unmittelbar angewendet. Als Erweiterung der Sicherheitsprüfungen kann man die Verfahren und den Umgang hinsichtlich technische Schwachstellen in den Netz- und Informationssystemen sehen. Diese zu ergreifende Maßnahme soll Verfahren zur Informationen über Schwachstellen über geeignete Kanäle verfolgen, Schwachstellen-Scans durchführen und sicherstellen, dass die Behandlung von Schwachstellen im Einklang mit Änderungsmanagement, Sicherheitspatch-Management, Risikomanagement, Management von Sicherheitsvorfällen vereinbar ist. Sofern die Auswirkung einer Schachstelle es erforderlich machen werden der Plan zur Minderung oder warum keine Abhilfemaßnahmen getroffen wurden Dokumentiert und begründet. Ein Bestandteil der Behandlung von Schwachstellen kann das Sicherheitspatch-Management. Nicht nur für die Behandlung sondern auch für die Allgemeine Sicherheit der Netz- und Informationssysteme sind Verfahren hinsichtlich Sicherheitspatches, im Einklang mit Änderungs-,  Schwachstellen-  und Risikomanagementverfahren, zu definieren. Diese Umfassen Sicherheitspatches mit angemessenen Frist nach ihrer Verfügbarmachung anzuwenden, dass diese vor Produktivem Einsatz getestet werden und nur aus vertrauenswürdigen Quellen stammen. Die Integrität der Sicherheitspatches ist ebefnalls anzuwenden sowie Maßnahmen zu ergreifen und Restrisiken zu akzeptieren wenn ein Patch nicht verfügbar oder anwendbar ist. Ist ein Sicherheitspatch verfügbar und kann nicht angewendet werden, da die Nachteile die Vorteile für die Cybersicherheit überwiegen, muss dies mit entsprechender Begründung Dokumentiert werden. Die Umfangreichen Anforderungen der \emph{Sicherheitsmaßnahmen bei Erwerb, Entwicklung und Wartung von Netz- und Informationssystemen} beinhalten ebenfalls die Notwendigkeit geeigneter Maßnahmen zum Schutz der Netz- und Informationssysteme vor Cyberbedrohungen. Bestandteil hiervon sind:\footnote{
                    \footcite[Vgl. Anhang, Nummer 6 bis 6.6.2 \& 6.10][]{EU2024-2690}
                }
                \begin{itemize}
                    \item Dokumentation der Architektur des Netzes
                    \item Kontrollen festlegen und durchführen um die internen Systeme vor unbefugtem Zugriff zu schützen
                    \item Netzkommunikation verhindert wenn diese für den Betrieb nicht erforderlich ist
                    \item Kontrolle des Fernzugriff auf Netz- und Informationssysteme festlegen (ebenfalls für Dienstleister)
                    \item Nicht benötigte Verbindungen und Dienste verbieten beziehungsweise deaktivieren
                    \item Im angemessenen Rahmen den Zugang zu Netz- und Informationssystemen nur mit genehmigten Geräte erlauben
                    \item Verbindungen von externen (beispielsweise Dienstleister) nur nach einem Genehmigungsverfahren und nur für einen definierten Zeitraum gewährleisten
                    \item Kommunikation zwischen Systemen nur über vertrauenswürdige Kanäle (logische, kryptografische oder physikalische Trennung von anderen Kommunikationskanälen) ermöglcihen
                    \item Durchführungsplan für das sichere, angemessene und schrittweise Einführung der neuesten Generation von Kommunikationsprotokollen
                    \item Durchführungsplan für die Einführung international vereinbarter und interoperabler moderner E-Mail-Kommunikationsnormen
                    \item Verfahren für die Sicherheit des \gls{dns} sowie für die Sicherheit und Hygiene des Internet-Routings bei für das Netz bestimmten Datenverkehr
                \end{itemize} Daraus kann man weiterführend die Anforderung der Segmentierung von Systeme in Netze oder Zonen, sowie die eigenen von den System Dritter zu trennen, ableiten. Diese Maßnahmen umfassen die funktionale, logische und physische Beziehung zwischen den eigenen und der Systeme dritter. Der Zugang zu einem Netz oder einer Zone auf Grundlage einer Bewertung der Sicherheitsanforderung gewährt. Systeme, die für den Betrieb oder für die Sicherheit unverzichtbar sind in gesondert gesicherten Zonen unterzubringen. Für sowohl ein- als auch ausgehenden Datenverkehr ist eine demilitarisierte Zone zu errichten. Allgemein ist die Kommunikation zwischen und innerhalb der Zonen auf das Notwendige zu beschränken. Ebenfalls ist eine strikt Trennung von operativen zu Netzen, welchem dem Zweck der Verwaltung dienen, notwendig. Netz- und Informationssysteme, genutzt als Produktionssysteme oder Systemen genutzt für die Entwicklung, Tests und Sicherung müssen ebenfalls voneinander Segmentiert werden. Software, welche auf den Netz- und Informationssysteme betrieben wird, erfordert ebenfalls gesonderten Schutzt. Withcit ist das die Netz- und Informationssysteme vor Schad- und nicht genehmigter Software geschützt werden. Hierbei sollen Maßnahmen, welche sowohl die Erkennung als auch die Vermeidung der Verwendung von Schad- und nicht genehmigter Software definiert und umgesetzt werden. Sofern angemessen ist eine Erkennungs- und Reaktions­software auf den Netz- und Informationssysteme zu betreiben.\footnote{
                    \footcite[Vgl. Anhang, Nummer 6.7, 6.8 \& 6.9][]{EU2024-2690}
                }\medbreak
                
                % Nummer 11: Zugriffskontrolle | PIAM, Audit Logging, Jumphost, Gruppen und Berechtigungen (ACL/ACE), Passwort Management, Zentrales Identity Management
                % \emph{Zugriffskontrolle}
                Die Mensch-Maschine- oder Maschine-Maschine-Interaktion muss Maßnahmen der \emph{Zugriffskontrolle} unterliegen. Hierfür wird ein Konzepte für die logische und physische Kontrolle des Zugangs zu Netz- und Informationssystemen auf Basis von geschäftlichen Anforderungen sowie Anforderungen an die Sicherheit von Netz- und Informationssystemen gefordert. Diese Konzept betrifft den Bereich des Zugang zu Netz- und Informationssystemen und bezieht sich auf das Personal, die Besuchern und externen Einrichtungen. Diese müssen in angemessen weise authentifiziert und durch einschlägige Personen genehmigt werden bevor der Zugang zu den Netz- und Informationssystemen gewährt wird. Umfang und Dauer sind soweit möglich zu Beschränken. Allgemein sind die Zugangs- und Zugriffsrechte für die Netz- und Informationssysteme zu Dokumentiert sowie in einem Register zu führen und die Grundsätze in From von \emph{Need-to-know}\footnote{Informationen sind nur für Subjekte zugänglich, welche die Informationen benötigen.}, \emph{Need-to-use}\footnote{Informationen sind nur für Subjekte zugänglich, welche die Informationen verwenden müssen.} und der \emph{Aufgabentrennung}\footnote{Informationen werden durch verschiedene Identitäten einem Benutzer Kontextabhängig zugänglich gemacht} umzusetzen. Bei Beendigung oder Änderung des Beschäftigungsverhältnisses müssen diese entsprechend angepasst werden. Anpassungen und andere Verwaltungstätigkeiten hinsichtlich der Zugangs- und Zugriffsrechte sind zu Protokollieren. Nach dem Grundsatz der \emph{Aufgabentrennung} sind dedizierte Systemverwaltungskonten für Systemverwaltungsvorgänge (Installation, Konfiguration, Verwaltung oder Wartung) und privilegierte Konten mit entsprechenden Grundsätze der Verwaltung erforderlich. Die Nutzung ist mittels starker Verfahren zur Identifizierung und Authentifizierung abzusichern sowie die Verwendung vorher zu genehmigen. Ein Systemverwaltungskonten darf ausschließlich mit Systemverwaltungssystemen verwendet werden. Ein Systemverwaltungssystemen, welches zur Verwaltung von Systemen verwendet wird, ist ausschließlich für diesen Zweck vorgesehen und von Anwendungssoftware zu trennen. Diese sind durch Authentifizierung und Verschlüsselung geschützt. Ein Wichtiger Aspekt von Zugangs- und Zugriffsrechte sind die Identitäten, mit welchen diese in Verbindung stehen. Die Identitäten eines Netz- und Informationssysteme und der Nutzer wird über dessen gesamten Lebenszyklus verwaltet. Hierfür gibt es eindeutige Kennungen, 1-zu-1 Beziehung von Benutzerkennung zu Nutzer (logische zu natürlicher Person), die Überwachung der Kennung und die Protokollierung der Verwaltung der Kennungen. Ist eine direkte Beziehung von Nutzer zu Kennung aus geschäftlichen oder operativen Gründen nicht möglich muss dies vorher genehmigt werden um zulässig verwendbar zu sein. Das Verfahren zur Genehmigung von geteilten Kennungen ist zu Dokumentieren sowie diese Kennungen im Rahmen des Risikomanagement im Bereich der Cybersicherheit gesondert zu betrachtet. Die von den Kennungen verwendeten Authentifizierungsverfahren und -techniken müssen für der Klassifizierung der Anlage bzw. des Werts angemessen sein. Auch das Personal muss angemessenen mit Authentifizierungsinformationen umgehen, was die Verwaltung geheimer Authentifizierungsinformationen und die Zuweisung an Nutzer unter Gewährleistung der Vertraulichkeit der Informationen beinhaltet.Die Änderung der Authentifizierungsdaten zu Beginn, in festgelegten Zeitabständen und bei beeinträchtigten der Authentifizierungsdaten ist erforderlich. Ebenfalls ist das Sperren der Kennungen bei gewisser Anzahl an Fehlgeschlagener Anmeldeversuche sowie das beenden inaktiver Sitzungen erforderlich. Mit gesonderte Authentifizierungsdaten sind Systemverwaltungskonten und privilegierte Konten zu schützen und nach Möglichkeit und wenn es angemessen ist Multifaktor-Authentifizierung zu verwenden. \footnote{
                    \footcite[Vgl. Anhang, Nummer 11][]{EU2024-2690}
                }\medbreak

                % Nummer 9: Kryptografie Schlüsselmanagement (Key Management), PKI
                % \emph{Kryptografie} 
                Authentifizierung steht immer in einem Kontext mit \emph{Kryptografie}. Gemäß der Risikobewertung und der Anlagen- und Werteklassifizierung ist mittels eines Konzept und Verfahren die angemessene und wirksame Nutzung von Kryptografie zur Gewährleistung der Vertraulichkeit, Authentizität und Integrität der Daten zu erstellen und anzuwenden. Das Konzept umfasst Art, Stärke und Qualität der kryptografischen Maßnahmen im Angemessenen Umfang hinsichtlich der Einstufung der Anlagen und Werte nach dem Krypto-Agilitätsansatz. Die Protokolle oder Protokollfamilien, kryptografische Algorithmen, Kryptierungsstärke, kryptografische Lösungen und Nutzungsverfahren müssen vor Verwendung genehmigt werden. Zusätzlich Umfasst das Konzept das Schlüsselmanagement inklusive der Methoden für:
                \begin{itemize}
                    \item Generierung von Schlüssel für kryptografische Systeme und Anwendungen
                    \item Ausstellen von \gls{pki}-Zertifikaten
                    \item Verteilung und Aktivierung von Schlüsseln
                    \item Speicherung der Schlüssel sowie erhalte für Autorisierte Personen
                    \item Änderung oder Aktualisierung von Schlüsseln
                    \item Prozess der Änderung oder Aktualisierung von Schlüsseln
                    \item Umgang mit beeinträchtigten Schlüsseln
                    \item Wiederruf von Schlüsseln inklusive dem zurückziehen und deaktivieren
                    \item Sicherung und Archivierung von Schlüsseln
                    \item Vernichtung von Schlüsseln
                    \item Protokollierung und Prüfung von Verwaltungstätigkeiten im Zusammenhang mit Schlüsseln
                    \item Aktivierungs- und Deaktivierungsfristen für die Beschränkung des Nutzungszeitraumes gemäß eigen erstellter Vorgaben
                \end{itemize}Sämtliche kryptografischen Maßnahmen beziehen sich auf sowohl gespeichert oder gerade übermittelte Daten.\footnote{
                    \footcite[Vgl. Anhang, Nummer 9][]{EU2024-2690}
                }\medbreak
                
                % Nummer 13: Sicherheit des Umfelds und physische Sicherheit 
                % \emph{Sicherheit des Umfelds und physische Sicherheit}
                Neben all den logischen Komponenten der Netz- und Informationssystemen ist auch die \emph{Sicherheit des Umfelds und physische Sicherheit} ein wichtiger Aspekt des \gls{nis2umsucg}. Diese soll Verluste, Schäden oder Beeinträchtigungen von Netz- und Informationssystemen oder Unterbrechungen ihres Betriebs durch Störungen in den Versorgungsleistungen verhindern. Hierfür soll die Einrichtung Maßnahmen ergreifen, welche vor Stromausfall und anderen Störungen schützen sowie Telekommunikationsdienste vor Abhörung und Beschädigung bewahren. Anhand bestimmter Mindest- und Höchstkontrollwerte ergibt sich ob ein Ereignisse die Versorgungsleistungen von Strom und Telekommunikationsdienste betreffend den zuständigen in- oder externen Stellen zu melden ist. Nach Ermessen und sofern angemessen sind Verträge für Notversorgung der Versorgungsleistungen abzuschließen. Die Wirksamkeit, Überwachung, Wartung und Erprobung der Netz- und Informationssysteme die sich auf von der Einrichtung angebotene dienste beziehen, insbesondere Strom, Temperatur- und Feuchtigkeitsregelung, Telekommunikation und Internetverbindung, ist kontinuierlich zu Erproben. Allgemein müssen  Physische Bedrohungen und Bedrohungen des Umfelds verhindert oder verringert werden indem Schutzmaßnahmen konzipiert und umgesetzt Mindest- und Höchstkontrollwerte bestimmt werden. Ergänzend sind Umgebungsparameter zu überwachen und den zuständigen internen oder externen stellen zu melden, wenn die Mindest- oder Höchstkontrollwerteliegen unter- oder überschreiten werden. Mittels Sicherheitsperimeter soll Bereiche der Netz- und Informationssysteme und andere zugehörige Anlagen, gesichert durch Zutrittskontrollen und Zugangspunkte, geschützt werden. Diese werden auf Basis Risikobewertung festgelegt. Ebenfalls gilt es Maßnamen zur physischen Sicherheit von Büros, Räumen und Betriebsstätten zu konzipieren und den unbefugten physikalischen Zugriff zu Überwachen.\footnote{
                    \footcite[Vgl. Anhang, Nummer 13][]{EU2024-2690}
                }\medbreak

                % Nummer 8: Grundlegende Verfahren im Bereich der Cyberhygiene und Schulungen im Bereich der Cybersicherheit 
                %\emph{Grundlegende Verfahren im Bereich der Cyberhygiene und Schulungen im Bereich der Cybersicherheit}
                Im Rahmen der \emph{Durchführungsverordnung (EU) 2024/2690} sind Personen an diversen stellen betroffen. \emph{Grundlegende Verfahren im Bereich der Cyberhygiene und Schulungen im Bereich der Cybersicherheit} für Mitarbeiter, Leitungsorgane und direkte Anbieter und Dienstanbieter ist ein weiter wichtiger Bestandteil der Anforderungen an die \emph{Durchführungsverordnung (EU) 2024/2690}. Mitarbeiter, Leitungsorgane und direkte Anbieter und Dienstanbieter müssen sich der Risiken bewusst und über die Bedeutung der Cybersicherheit informiert werden. Mitarbeiter, Leitungsorgane und direkte Anbieter und Dienstanbieter müssen sich der Risiken bewusst und über die Bedeutung der Cybersicherheit informiert werden. Um dies praktisch umzusetzen wird den betroffenen Personen ein repetitive Sensibilisierungsprogramm angeboten das:
                \begin{itemize}
                    \item Im Einklang mit dem Konzept und den Verfahren für die Sicherheit von Netz- und Informationssystemen steht
                    \item Einschlägige Cyberbedrohungen beinhaltet
                    \item Kontaktstellen und Ressourcen für zusätzliche Informationen beinhaltet
                    \item Kontaktstellen für Beratung zu Cybersicherheitsfragen beinhaltet
                    \item Verfahren im Bereich der Cyberhygiene für Nutzer beinhaltet
                \end{itemize} Zusätzlich müssen sicherheitsrelevanten Rollen die zusätzlich Sensibilisierung erfordern identifiziert werden. Ergänzend dem Sensibilisierungsprogramm ist ein Schulungsprogramm einzuführen. Der Schulungsbedarf für bestimmte Rollen und Positionen auf Basis von Kriterien er ermittelt und beinhaltet zumindest Anweisungen für die sichere Konfiguration und den sicheren Betrieb der Netz- und Informationssysteme, einschließlich mobiler Geräte sowie die Unterrichtung über bekannte Cyberbedrohungen und Schulung in Bezug auf das Verhalten bei sicherheitsrelevanten Ereignissen. Schulungsprogramm sowie der Schulungsbedarf werden regelmäßige aktualisiert.\footnote{
                    \footcite[Vgl. Anhang, Nummer 8][]{EU2024-2690}
                }\medbreak

                % Nummer 10: Sicherheit des Personals
                %\emph{Sicherheit des Personals} 
                \emph{Grundlegende Verfahren im Bereich der Cyberhygiene und Schulungen im Bereich der Cybersicherheit} stehen im Einklang mit der \emph{Sicherheit des Personals}. Für die \emph{Sicherheit des Personals} ist es wichtig das sowohl Mitarbeiter als auch direkten Anbieter und Diensteanbieter ihre Verantwortlichkeiten im Bereich der Sicherheit verstehen und sich zu ihrer Einhaltung verpflichten. Zu Gewährleistung gibt es Mechanismen die sicherstellen das die betroffenen Personen die Standardverfahren der Cyberhygiene verstehen und befolgen. Privilegierte oder administrative Rollen sich ihrer Verantwortlichkeiten und Weisungsbefugnisse bewusst und handeln entsprechend. Im Bezug auf die Sicherheit von Netz- und Informationssystemen verstehen und handeln die Leitungsorgane entsprechend ihre Rolle, Verantwortlichkeit und Weisungsbefugnisse. Bei der Personalakquise ist zu Gewährleisten das die Person für die Position qualifiziert ist. Dies kann mittels diverser verfahren, wie beispielsweise die Überprüfung der   Referenzen, Überprüfungsverfahren, Validierung von Zeugnissen oder schriftliche Prüfungen, erfolgen. Allgemein sind Kriterien Festzulegen, anhand welcher Rollen, Verantwortlichkeiten und Weisungsbefugnisse ermittelt werden, bei welchen es eine Zuverlässigkeitsüberprüfungen bedarf. Diese Rollen, Verantwortlichkeiten und Weisungsbefugnisse dürfen nur von Personen nach einer Zuverlässigkeitsüberprüfungen wahrgenommen werden. Unabhängig des Beschäftigungsverhältnisses, also auch nach Beendigung dieses, ist sicherzustellen das die Sicherheit von Netz- und Informationssystemen gewahrt ist. Generell sind Disziplinarverfahren für den Umgang mit Verstößen zu implementieren. Dies setzt die \emph{Konzept für die Sicherheit von Netz- und Informationssystemen} in den Fokus. Die Disziplinarverfahren werden bekannt gemacht und aufrechterhalten.\footnote{
                    \footcite[Vgl. Anhang, Nummer 8][]{EU2024-2690}
                }\medbreak

                In Summe beschreiben die zuvor genannten Konzepte, Verfahren, Maßnahmen und Anforderungen den im Kontext von \gls{nis2umsucg} und damit auch der \emph{Durchführungsverordnung (EU) 2024/2690} umzusetzenden Rahmen. Konkrete Handlungsempfehlungen sind in Abschnitt \ref{sec:ParktischeMaßnahmen} beschrieben und ergänzen diese theoretische Grundlage.
            \subsubsection{Registrierungs- und Meldewesen}
            \subsubsection{Risiken und Sanktionen}

        \subsection{KRITIS-Dachgesetz (KRITIS-DachG)}
        Das \gls{kritis-dachg} soll die physische Resilienz von Betreibern kritischer Einrichtungen Regeln. Hierbei gibt das \gls{kritis-dachg} Resilienzziele, Orientierung und eine Übersicht von beispielhaften Maßnahmen vor. Entgegen dem \gls{nis2umsucg} liegt hier der Fokus auf die nicht-IT-bezogene Maßnahmen zur Stärkung der Resilienz dieser Einrichtungen.
            \subsubsection{Betroffene Organisationen und Subjekte}
            \subsubsection{Zuständige Aufsichtsbehörden}
            \subsubsection{Erforderliche Maßnahmen}
            \subsubsection{Registrierungs- und Meldewesen}
            \subsubsection{Risiken und Sanktionen}



    %:: Abschnitt: Herleitung des Fragenkatalogs
    \newpage
    \section{Praktische Maßnahmen}\label{sec:ParktischeMaßnahmen}
        \subsection{Organisationsstrukturen}
            \emph{In adäquaten Unterabschnitten sollen relevante und für die Arbeit nützliche Organisationsstrukturen näher erläutert werden. Hierbei soll der Fokus unter anderem darauf liegen wie man die Leistung von Mitarbeitern bewerten kann, wenn diese im Rahmen der Informationssicherheit zusätzliche Aufgaben wahrnehmen, welche nicht unmittelbar Ihrer Haupttätigkeit entsprechen.}
        \subsection{Geschäftsprozessmodellierung}
            \emph{In adäquaten Unterabschnitten sollen relevante Aspekte der Geschäftsprozessmodellierung beschrieben werden. Hierbei soll ein Fokus auf die BPMN2.0 gelegt werden, da diese Prozesse sehr spezifisch beschreiben kann und direkt oder indirekt in Business Process Engine verwendet werden können.}
        \subsection{Technische Konzepte}
            \emph{In adäquaten Unterabschnitten sollen relevante Aspekte technischer Konzepte näher erläutert werden. Essenzielle Systeme, wie beispielsweise ein Information Security Management System (ISMS), sollen entsprechend beschrieben werden. Weitere technische Maßnahmen (Netzwerksegmentierung, Einbruchserkennung (IDS), Endpoint Management, Patch Management, uvm.) sollen hier ebenfalls beschrieben werden, sofern hinreichend für das Ergebnis der Arbeit. Dies soll an die Geschäftsprozesse anknüpfen, da Dokumentation sowohl organisatorisch (manuell) oder auch technisch (automatisch) sowie hybrid sein kann.}
    \section{Herleitung des Fragenkatalogs}\label{sec:HerleitungDesFragenkatalog}
        \emph{In diesem Kapitel wird mit adäquaten Unterabschnitten, welche sich im Rahmen der Recherche ergeben werden, der Fragenkatalog an sich und dessen Herleitung, auf Basis der diversen Werke, beschrieben}

    %:: Abschnitt: Rahmen und Ergebnis der Fallstudie
    \newpage
    \section{Rahmen und Ergebnis der Fallstudie}
        \emph{Hierbei wir beschrieben in welchem Umfeld die Fallstudie durchgeführt wird, sowie dessen Ergebnis analysiert. Das genaue Ergebnis befindet sich dann im Anhang.}

    %:: Abschnitt: Schlussbetrachtung
    \newpage
    \section{Schlussbetrachtung}
        \emph{Die Schlussbetrachtung soll alle abschließend wichtigen Punkte aufgreifen.}
        \subsection{Limitation}
        \emph{Im Rahmen der Limitation soll eingegrenzt werden was die Arbeit Limitiert hat. Unter anderem ist das Ergebnis auf den IST Stand der aktuellen Gesetzesentwürfe beschränkt.}
        \subsection{Ausblick}
        \emph{Aus der Arbeit können sich weitere Forschungsfragen ergeben. Nicht zuletzt der weitere Forschungsbedarf bzw. die erneute Validierung wenn aus den Entwürfen beschlossene Gesetze geworden sind.}
        \subsection{Fazit}
        \emph{Abschließend soll im Rahmen der Arbeit noch ein selbstkritisches Fazit gezogen werden.}

    %:: Abschnitt: Literatur
    \newpage
    \printbibliography
\end{document}