\documentclass[11pt,a4paper,hidelinks]{article}   % 11pt Schrift, A4-Format, Artikelklasse
\usepackage[
    left=4cm, 
    right=2cm, 
    top=2.5cm, 
    bottom=2.5cm
]{geometry}                                                 % Seitenränder

%:: Eigenschaften des Dokuments                 
\usepackage[utf8]{inputenc}                                 % Eingabecodierung UTF-8 (für Umlaute etc.)
\usepackage[scaled]{helvet}                                 % Serifenlose Schrift Helvetica, skaliert
\usepackage[T1]{fontenc}                                    % Korrekte Zeichencodierung für Trennung & Sonderzeichen
\usepackage[german]{babel}                                  % Deutsche Sprache, Trennung, Datum etc.

%:: Inhalt & Darstellung
\usepackage{enumitem}                                       % Anpassung von Aufzählungen
\usepackage{amsmath}                                        % Erweiterte Mathematik-Umgebungen
\usepackage{amssymb}                                        % Mathematische Symbole (ℝ, ⊕, etc.)
\usepackage{graphicx}                                       % Einfügen und Skalieren von Bildern
\usepackage{fancyhdr}                                       % Benutzerdefinierte Kopf- und Fußzeilen
\usepackage{tablefootnote}

%:: Wissenschaftlicher Apparat
\usepackage{hyperref}                                       % Hyperlinks für Referenzen, Inhaltsverzeichnis, etc.
\usepackage[toc,acronym,nomain,nonumberlist]{glossaries}    % Glossar- und Abkürzungsverzeichnis

\usepackage[
	backend=biber,
	style=authoryear
]{biblatex}                                             % Literaturverwaltung mit Biber und Autor-Jahr-Stil

%:: Präambel

\pagestyle{fancy}
\lhead{\nouppercase\leftmark}
\chead{ }
\rhead{\thepage}
\cfoot{ }

\renewcommand{\familydefault}{\sfdefault}

\setlength{\parindent}{0pt}
\setlength{\parskip}{6pt}

\bibliography{bibliography.bib}
\loadglsentries{./acronyms.tex} % Einbinden des Glossars fuer Akronyme
\usepackage{geometry}

\renewcommand{\maketitle}[5]{
	\begin{titlepage}
		\newgeometry{left=4cm, right=2cm, top=2cm, bottom=2cm}
		\begin{center}
			\begin{center}
				\includegraphics[width=0.5\linewidth]{logo.png}
			\end{center}
			\vspace{.5cm}
			\begin{Large}
				\textbf{FOM Hochschule für Oekonomie \& Management}
				\begin{small}
					\\ Hochschulzentrum Frankfurt am Main
				\end{small}
			\end{Large} \\
			\vspace{1.5cm}
			\begin{large}
				\textbf{Bachelor-Thesis} \\
				\begin{small}
					im Studiengang Wirtschaftsinformatik
				\end{small}
			\end{large} \\
			\vspace{1.0cm}
			\begin{small}
				zur Erlangung des Grades eines \\
				\vspace{.15cm}
				\begin{Large}
					Bachelor of Science (B.Sc.)
				\end{Large}
			\end{small} \\
			\vspace{1.0cm}
			\begin{small}
				über das Thema \\
				\vspace{.15cm}
				\begin{large}
					\textbf{#2}
				\end{large}
			\end{small} \\
			\vspace{1.0cm}
			von \\
			#1
		\end{center}
		\vspace*{\fill}
		\begin{tabular}{l @{ : } l}
			Erstgutachter & #3 \\
			Matrikelnummer & #4 \\
			Abgabedatum & #5 \\
		\end{tabular}
		\restoregeometry
	\end{titlepage}	
} 

\makeglossaries

\begin{document}
    \maketitle{Marvin Künzel}{Informationssicherheit in deutschen Unternehmen – Anforderungen und Umsetzungsoptionen im Kontext des KRITIS-Dachgesetzes und NIS2-Umsetzungsgesetzes}{Oliver Bach M.Sc.}{587486}{11. September 2025}
    \pagenumbering{roman}
    % \addtocounter{table}{-1}    % TODO: Dokumentieren
    \clearpage

    %:: Abschnitt: Verzeichnisse
    \newpage
    \tableofcontents
        % \addcontentsline{toc}{section}{Abbildungsverzeichnis}
	    % \listoffigures
	\newpage
    \addcontentsline{toc}{section}{Tabellenverzeichnis}
	\listoftables
	\newpage
    \printglossary[type=\acronymtype, title=Abkürzungsverzeichnis, toctitle=Abkürzungsverzeichnis]

    %:: Abschnitt: Einleitung
    \newpage
    \pagenumbering{arabic}
    \section{Einleitung}
    Im folgenden Kapitel sollen zunächst die Beweggründe für diese Arbeit widergespiegelt werden. Anschließend erfolgt eine Einordnung auf welchen Rahmen sich diese Arbeit bezieht um darauf folgend auf die spezifische Zielsetzung einzugehen. Abschließend wird der generelle Aufbau dieser Arbeit umrissen um die einzelnen Kapitel grob zusammenzufassen.
    \subsection{Motivation}\label{sec:Einleitung_Motivation}
        Der zunehmende Fachkräftemangel erschwert es \gls{kmu} angemessene Maßnahmen zur physischen Resilienz sowie der Informationssicherheit zu treffen. Mit den Richtlinien 2022/2555 und 2022/2557 verpflichtet die \gls{eu} Ihre Mitgliedstaaten auf nationaler Ebenen Gesetze zu erlassen, welche wiederum Unternehmen verpflichtet entsprechende Maßnahmen zu ergreifen. Sowohl kritische Einrichtungen sowie dessen Zulieferer sind somit verpflichtet neue Mindeststandards einzuhalten und zu implementieren. Zwar wurden die Richtlinien auf nationaler Ebene noch nicht vollständig in geltendes Recht überführt, mit dem \gls{nis2umsucg} sowie dem \gls{kritis-dachg} liegen jedoch konkrete Gesetzesentwürfe vor. Das stellt \gls{kmu} vor eine Herausforderung, da diese mit weniger Personellen und Finanziellen Ressourcen teilweise dieselben regulatorischen Anforderungen unterliegen, wie große Unternehmen. Dies verdeutlicht die Notwendigkeit von praxisnahen Empfehlungen wie diese regulatorischen Anforderungen umzusetzen sind, um \gls{kmu} entsprechend zu entlasten und deren Rechtskonformität und Wettbewerbsfähigkeit zu gewährleisten.\footnote{
            \footcite[][]{MISSING}
        }
    \subsection{Abgrenzung}
        Wie bereits zuvor in Abschnitt \ref{sec:Einleitung_Motivation} erwähnt setzt diese Arbeit \gls{kmu} in den Fokus. Mit mehr als 50\% der Unternehmen in Deutschland ist dies die größte Gruppe hinsichtlich betroffener Unternehmen. In Tabelle \ref{tbl:definition-kmu} sind die entsprechenden Vorraussetzungen für die Klassifikation eines Unternehmens beschrieben. 
        \begin{table}[ht]
            \caption[Definition von kleinen und mittleren Unternehmen]{Definition von kleinen und mittleren Unternehmen\footnotemark}
            \label{tbl:definition-kmu}
            \resizebox{\textwidth}{!}{
                \begin{tabular}{llll}
                \hline
                Unternehmen & Anzahl Beschäftigte & Umsatz Mio. Euro pro Jahr & Bilanzsumme Mio. Euro pro Jahr \\ 
                \hline
                kleinst     & \(\leq\) 9            & \(\leq\) 2                  & \(\leq\) 2                       \\
                klein       & \(\leq\) 49           & \(\leq\) 10                 & \(\leq\) 10                      \\
                mittel      & \(\leq\) 249          & \(\leq\) 50                 & \(\leq\) 43                      \\ 
                \hline
                \end{tabular}
            }
        \end{table}\footnotetext{\footcite[Vgl.][Anhang, Titel 1, Artikel 2]{l124-36}} \\
        Hierbei stehen die Parameter nicht ausschließlich in einer logischen \(UND\) Beziehung zueinander, sondern sind vielmehr wie folgt zu Interpretieren.
            \[
            \text{Besch\"aftigte} \;\land\;
            \bigl( \text{Umsatz} \oplus \text{Bilanzsumme} \bigr)
            \]
        Das Ergebnis richtet sich somit ausschließlich an Unternehmen, die weniger als 250 Beschäftigte haben und entweder eine Bilanzsumme von unter 43 Millionen Euro oder einen Jahresumsatz von unter 50 Millionen Euro aufweisen.
    \subsection{Zielsetzung}
        Im Rahmen dieser Arbeit werden das \gls{nis2umsucg} sowie das \gls{kritis-dachg} betrachtet und auf die für Betreiber wesentlichen Punkte zusammengefasst. Dies beinhaltet unter welchen Vorraussetzungen eine Organisation oder ein Subjekt von einem der Gesetze\footnote{Aktuell handelt es sich bei sowohl \gls{nis2umsucg} auch \gls{kritis-dachg} um Entwürfe zu Gesetzen. Der aktuelle Umstand ist nähergehend in Unterabschnitt \ref{subsec:AktuellerForschungsstand} beschrieben.} betroffen ist, welche Behörden für die Aufsicht der erforderlichen Maßnahen Zuständig sind sowie die zu ergreifenden Maßnahmen an sich. Ebenfalls ist das Registrierungs- und Meldewesen sowie die Risiken und Sanktionen für betroffene Organisationen und Subjekte Bestandteil dieser Arbeit. Aus diesem Theoretischen Rahmen werden praktische Handlungsempfehlungen abgeleitet um exemplarisch aufzuzeigen wie die erforderlichen Maßnahmen erfüllt werden können. Abschließend wird aus diesem Ergebnis ein Fragenkatalog abgeleitet, sodass Unternehmen auf Basis dessen den eigenen Stand besser einschätzen können. Zur Validierung wird dieser Fragenkatalog im Rahmen einer Fallstudie bei einem mittelständigen Unternehmen angewendet.
    \subsection{Aufbau der Arbeit}
        Strukturell soll diese Arbeit mit dem \emph{Methodisches Vorgehen} einleiten. Hier wird darauf eingegangen welche Methoden und Paradigma im Rahmen dieser Arbeit angewendet wurde. Um ein eindeutiges Verständnis für den praktischen Rahmen, beginnend mit den \emph{praktischen Maßnahmen}, zu erhalten werden diverse Begrifflichkeit und Grundlagen im \emph{theoretischen Rahmen} erläutert sowie eingeordnet. Beginnend in diese Abschnitt ist jedoch zunächst eine Einordnung des \emph{Stand der fachlichen Auseinandersetzung und des Gesetzgebungsverfahrens zu \gls{nis2umsucg} und \gls{kritis-dachg}}. Teil des \emph{theoretischen Rahmen} ist ebenfalls Abschnitt \ref{sec:NIS2UmsuCG}, welcher ausschließlich das \gls{nis2umsucg} behandelt. Aufgrund des Umfangs wurde dies in ein gesonderten Abschnitt aufgefasst. Daran schließt praktische Rahmen mit Maßnahmen, der erstellung eine daraus Folgenden Fragenkatalogs sowie die Anwendung dessen in einer Fallstudie. Abschließend wird auf Limitationen dieser Arbeit eingegangen sowie ein Ausblick gegeben. Beendet wird diese Arbeit mit einem Fazit um dem Rahmen der Arbeit und dessen Ergebnisse kritisch zu betrachten. 


    %:: Abschnitt: Methodisches Vorgehen
    \newpage
    \section{Methodisches Vorgehen}
        Methodisch beruht diese Arbeit auf einer systematischen Literaturrecherche, welcher die Basis für ein deduktiven Erkenntnisweg ist, welcher induktiv im Rahmen eines Fragenkatalogs verifiziert wird. In diesem Abschnitt wird das Vorgehen nähergehend beschrieben sowie eine Einordnung zur Herleitung des Fragenkatalogs getätigt.
        \subsection{Systematische Literaturrecherche}
            \emph{Was ist die systematische Literaturrecherche?}
            \subsubsection{Vorgehen und Auswahlkriterien}
                \emph{Beschreibung des Ausgangspunktes und der daraus folgenden Suche weiterer Literatur. Im Fokus soll stehen wo gesucht wurde (Suchmaschinen, Organisationen etc.), warum dort gesucht wurde (Kontext hinsichtlich der Relevanz) und wie dort gesucht wurde (Abstrakte Beschreibung der relevanten Schlüsselworte, welche sich aus den Richtlinien implizit oder Explizit ergeben.)}
            \subsubsection{Herleitung des Fragenkatalogs}
                \emph{Beschreibung der Struktur des Fragenkatalogs (Basis sind die Richtlinien) sowie die Ableitung / Definition der genauen Fragen.}
        \subsection{Deduktiver Erkenntnisweg}
            \emph{Beschreiben was der deduktive Erkenntnisweg ist und warum dieser für die Ausgangslage der Arbeit relevant ist. Zusätzlich kann hier Bezug auf die hermeneutische Komponente genommen werden.}
        \subsection{Induktive Verifikation}
            \emph{Beschreiben was in dem Kontext der Arbeit die induktiv empirische Verifikation ist und warum diese erforderlich ist. Hier kann ein Bezug auf die interpretation von Gesetzen genommen werden (subjektives Verständnis) und inwieweit die Verifikation des Ergebnis eine Validierung ist.}
            \subsubsection{Fallstudie}
                \emph{Beschreibung was eine Fallstudie ist und was genau sowie in welcher Form dies im Rahmen der Arbeit Anwendung findet.}

    
    %:: Abschnitt: Theoretischer Rahmen
    \newpage
    \section{Theoretischer Rahmen}
        Im Rahmen des Kapitel des theoretischen Rahmen werden alle Grundlagen und definition getroffen, auf welchem diese Arbeit beruht. Dies ist die fundamentale Basis der Herleitung des Fragenkatalog beschrieben in Abschnitt \ref{sec:HerleitungDesFragenkatalog} und Ordnet darüberhinaus diese Arbeit in den aktuellen Forschungsstand ein.
        \subsection{Stand der fachlichen Auseinandersetzung und des Gesetzgebungsverfahrens zu NIS2UmsuCG und KRITIS-Dachgesetz}\label{subsec:AktuellerForschungsstand}
        % NOTE: https://www.bundesrat.de/DE/aufgaben/gesetzgebung/zust-einspr/zust-einspr-node.html
        Durch die Auflösung des 20. Deutschen Bundestages wurden die Gesetzgebungsverfahren sowohl für \gls{nis2umsucg} als auch \gls{kritis-dachg} im 21. Deutschen Bundestag neu begonnen.\footnote{\footcite[Vgl. §125,][]{BTGO}\footcite[Vgl. S. 1][]{brd:c54cf9}\footcite[Vgl. S. 52][]{brd:4a7cbe}} Dies ist der Diskontinuität\footnote{„\emph{Nach diesem Grundsatz verfallen am Ende der Wahlperiode im Bundestag alle noch nicht abschließend behandelten Vorlagen (vgl. § 125 Geschäftsordnung des Bundestages). Die bisherigen Abgeordneten verlieren ihr Mandat und Fraktionen oder Ausschüsse müssen neu gebildet werden. Für den Bundesrat als kontinuierlich tagendes Organ gilt dieser Grundsatz nicht.}“\footcite[Vgl. S. 1,][]{brd:33503f}} der Gesetzgebungsverfahren nach Auflösung eines Bundestages der \gls{brd} geschuldet. Im Rahmen des Gesetzgebungsverfahren der \gls{brd} wird zunächst ein Referentenentwurf erarbeitet, welcher durch die Bundesregierung als Kabinettsentwurf im Rahmen von drei Lesungen im Deutschen Bundestag als Beschlossenes Getz verabschiedet wird.\footcite[Vgl. S. 1][]{bmi:a4771a} Im Falle von \gls{nis2umsucg} und \gls{kritis-dachg} handelt es sich um Zustimmungsgesetze, weswegen der Bundesrat dem Beschluss des Bundestag zustimmen muss. Sofern der Bundesrat keine Änderungsvorschläge anbringt ist es anschließend ein Beschlossenes Gesetz.\footnote{
            \footcite[Vgl. Artikel 104a, Absatz 4,][]{GG}
            \footcite[Vgl. Artikel 84, Absatz 1 GG,][]{GG}
            \footcite[Vgl. S. 1][]{bmi:a4771a}
        }\medbreak
        \gls{nis2} muss seit 17. Oktober 2024 als geltendes nationales Recht umgesetzt sein\footcite[Vgl. Artikel 21, Absatz 5,][]{EU2022-2555} und ist aktuell in Form von \gls{nis2umsucg} ein Referentenentwurf vom 25. Juni 2025\footcite[Vgl. S. 2,][]{bmi:98905b}. Im Dezember 2024 ist bereits eine zwei monatige Frist der Europäischen Kommission verstrichen zu reagieren und die erforderlichen Maßnahmen ergreifen.\footcite[Vgl. S. 15,][]{eu:9274cb} Das damit Verbundene Vertragsverletzungsverfahren der Europäischen Kommission läuft dementsprechend noch und ist an dem Punkt das die Europäische Kommission den Europäische Gerichtshof ersuchen kann.\footnote{\footcite[Vgl. S. 2,][]{eu:4a7cbe}\footcite[Vgl. S. 2,][]{eu:376b82}\footcite[Vgl. S. 15,][]{eu:9274cb}} Das \gls{kritis-dachg} wurde bereits als Kabinettsentwurf in der 203. Sitzung des 20. Bundestages in Form einer ersten Lesung besprochen, jedoch befindet sich das Gesetzgebungsverfahren wie zuvor beschrieben wieder am Anfang.\footcite[Vgl. S. 26270 - 26280][]{brd:33503f} Somit sind weder \gls{kritis-dachg} noch \gls{nis2umsucg} Rechtskräftige Gesetze.\medbreak
        In der Privatwirtschaft sind \gls{nis2umsucg} und \gls{kritis-dachg} mittlerweile allgegenwärtig. Dienstleister und Hersteller bieten für beide Themengebiete bereits Informationen aber auch Dienstleistungen, wie Beratungsleistung, an.\footnote{

        } Zu \gls{nis2umsucg} hat das \gls{bsi} bereits diverse Allgemeine Informationen aber auch spezifische hinsichtlich der Melde- und Registrierungspflicht aber auch zu der Betroffenheit von Unternehmen veröffentlicht. \gls{nis2umsucg} als auch \gls{kritis-dachg} haben mit OpenKRITIS eine fortlaufenden Informationsressourcen. Entgegen dieser Arbeit beschränkt sich OpenKRITIS auf das Schaffen von klaren Strukturen zur Umsetzung von Cybersecurity, Governance und Prüfungen. Bezug auf Konkrete Umsetzungen wird hier nicht genommen.
        % -----
        
        \subsection{Begriffsdefinitionen}
        Im Rahmen dieser Arbeit werden diverse Begriffe verwendet. Um ein einheitliches VVerständnis zu schaffen ist zunächst die definition grundlegend relevanter Begriffe erforderlich. Im folgenden Abschnitt werden alle Relevanten Begriff hinsichtlich Ihrer Bedeutung definiert.       
            \subsubsection{Kritische Anlagen und kritische Dienstleistungen}
            Eine kritische Anlage ist eine Anlage, welche für die Erbringung einer kritischen Dienstleistung erheblich ist. Eine kritische Dienstleistung wiederum ist eine Dienstleistung zu Versorgung der Allgemeinheit, dessen Ausfall oder Beeinträchtigung erhebliche Versorgungsengpässe oder die öffentlichen Sicherheit gefährden würde. Nach §56 Absatz 4 sind die kritischen Dienstleistungen nach §2 Nummer 24 für die genannten Sektoren durch das Bundesministerium des Innern und für Heimat in Einvernehmen mit diversen Ministerien durch Rechtsverordnungen zu bestimmen. Demnach gibt es aktuell noch keine spezifische definition von kritische Anlagen.\footnote{
                \footcite[§2 Nummer 22 und 24 sowie §56 Absatz 4][]{NIS2UmsuCG}
            }

            \subsubsection{Betreiber kritischer Anlagen}\label{def:BetreiberKritischerAnlage}
            Eine natürliche oder juristische Person sowie unselbständige Organisationseinheit einer Gebietskörperschaft, welche bestimmenden Einfluss auf eine oder mehrere kritische Anlagen ausübt. Dies ist unter Berücksichtigung der rechtlichen, wirtschaftlichen und tatsächlichen Umstände. Im Sektor des Finanzwesen ist alleine die tatsächliche Sachherrschaft  ausschlaggebend.\footnote{
                \footcite[§28 Absatz 7][]{NIS2UmsuCG}
            }

            \subsubsection{Vertrauensdienst, Vertrauensdiensteanbieter und qualifizierter Vertrauensdiensteanbieter}
            Ein Vertrauensdiensteanbieter ist ein Anbieter, welcher Vertrauensdienste bereitstellt. Ein Vertrauensdienst ist im weitesten Sinne ein elektronischer Dienst, welcher einer der folgenden Kategorien zuzuordnen ist:
            \begin{itemize}
                \item Die Handhabung (Erstellung, Überprüfung und Validierung) von
                \begin{itemize}
                    \item elektronischen Signaturen
                    \item elektronischen Siegel
                    \item elektro­nischen Zeitstempel
                    \item Zertifikaten für die Website-Authentifizierung
                \end{itemize}
                \item Zustellung elektronischer Einschreiben
                \item Aufrechterhaltung von den Diensten betreffend der elektronischen Signaturen, Siegeln oder Zertifikaten
            \end{itemize}
            Qualifiziert ist der Vertrauensdienst, wenn er den Anforderungen der Verordnung (EU) Nr. 910/2014 genügt. Dies ist für die Bundesrepublik Deutschland im \gls{vdg} geregelt.

            Ein Anbieter ist dann als qualifizierter Vertrauensdiensteanbieter zu Betrachtet wenn dieser qualifiziert Vertrauensdienste anbietet und von der in dem Land zuständigen Aufsichtsstelle entsprechend eingestuft wurde. Für die Bundesrepublik Deutschland ist nach §2 des \gls{vdg} Entsprechend die Bundesnetzagentur und nachgelagert das \gls{bsi} zuständig.\footnote{
                \footcite[Vgl. Artikel 3, Nummer 16, 17, 19 und 20][]{EU910-2014}
                \footcite[Vgl. §2][]{VDG}
            }

            \subsubsection{Telekommunikationsdienste}
            Ein Telekommunikationsdienst ist im Allgemeinen ein Internetzugangdienst, interpersonelle Telekommunikationsdienste oder jeder Dienst, der ganz oder überwiegend in der Übertragung von Signalen über Telekommunikationsnetze besteht. Öffentlich ist dieser Telekommunikationsdienste wenn er einem unbestimmten Personenkreis zur verfügung steht.\footnote{
                \footcite[Vgl. §3 Nummer 60][]{TKG}
            }

            \subsubsection{Telekommunikationsnetz}
            Ein Telekommunikationsnetz ist allgemein die Gesamtheit aller Übertragungssysteme um Informationen auszutauschen. Ein öffentliches Telekommunikationsnetz wiederum ist ein Telekommunikationsnetz (nach Satz 1), welche die Erbringung von öffentlich zugänglicher Telekommunikationsdiensten im Sinne der Übertragung von Informationen von Netzabschlusspunkten dienen. Ein Netzabschlusspunkten wiederum ist der physische Punkt an welchem ein Endnutzer, also eine natürliche oder juristische Person, die einen öffentlich zugänglichen Telekommunikationsdienst für private oder geschäftliche Zwecke in Anspruch nimmt und weder öffentliche Telekommunikationsnetze betreibt noch öffentlich zugängliche Telekommunikationsdienste erbringt.\footnote{
                \footcite[Vgl. §3 Nummer 13, 32, 41, 42, 61 und 65][]{TKG}
            }

            \subsubsection{Kritische Komponente}
            Eine kritische Komponente definiert sich aus verschiedenen Bestandteilen. Zunächst muss fundamental der Begriff des elektronischen Kommunikationsnetzes definiert werden. Dies ist eine einzelne oder eine Gruppe aus aktiven oder passiven Ressourcen, welche, unabhängig des Mediums, zur Übertragung von Signalen zum Austausch von Informationen verwendet werden. Dies wiederum ist Bestandteil der Definition von Netz- und Informationssystemen. Fokussiert man sich zunächst auf den Begriff des Informationssystems kann man dies als Gruppe von Anwendungen, Diensten, informationstechnischen Anlagen oder anderen Komponenten für die Informationsverarbeitung definieren. Allgemein gültiger ist ein Informationssystemen eine Vorrichtung oder die Gruppe von Vorrichtungen, welche die automatische Datenverarbeitung unter Grundlage eines Programmes durchführt. Somit definiert sich das Netz- und Informationssystemen aus der Definition von elektronischen Kommunikationsnetzen und Informationssysteme, sowie den Daten, welche zum Zweck des Betriebes eines elektronisches Kommunikationsnetz oder Informationssystem gespeichert, verarbeitet, abgerufen oder übertragen werden. Eine kritischen Komponente wiederum ist ein \gls{ikt-produkt}. Dies ist ein ein Element oder eine Gruppe aus Elementen eines Netz- und Informationssystem. Kritisch wird die Komponente dadurch das diese in einer kritischen Anlagen eingesetzt und unter Grundlage des \gls{nis2umsucg} als kritische Komponente bestimmt wird oder eine kritische Funktion realisiert.\footnote{
                \footcite[Vgl. §2 Nummer 23][]{NIS2UmsuCG}
                \footcite[Vgl. Artikel 4, Nummer 1][]{EU2016-1148}
                \footcite[Vgl. Artikel 2, Buchstabe a][]{EU2002-21-EG}
                \footcite[Vgl. Artikel 2, Nummer 12][]{EU2019-881}
                \footcite[Vgl. S. 5][]{iso27000-2018}
            }

            \subsubsection{Risikomanagement}

            \subsubsection{Sicherheitsvorfall}
            \emph{auch erheblich....}

            \subsubsection{Rechtsverordnungen}

            \subsubsection{maritime Infrastruktur }

            \subsection{NIS-2-Umsetzungs- und Cybersicherheitsstärkungsgesetz}\label{subsec:NIS2UmsuCG}
            Abhängigkeiten physischer und digitaler Infrastruktur bieten ein Risikopotential für den Wohlstand, das Wachstum, die Versorgungssicherheit und das Markt­funktionieren von Unternehmen in der \gls{eu} und \gls{brd}. Eine Zunehmende Bedrohungslage durch Ransomware, Schwachstellen­ausnutzung, falsch konfigurierte Informationssysteme, Supply-Chain-Angriffe und viele andere Bedrohungsarten lassen das Risikopotential zu einem Risiko wachsen. Geopolitische Faktoren sowie unzureichende Steuerungs­instrumente auf Europäischer und Nationaler Ebene bergen ein Risiko für die Mitglieder der \gls{eu}. All diese Punkte sollen im Rahmen der \emph{\gls{nis2}} gelöst werden. Hauptziel ist ein Einheitlich hohes Cyber­sicherheits­niveau innerhalb der \gls{eu} zu gewährleisten. Mit dem \gls{nis2umsucg} wird die \gls{nis2} als nationales Gesetz der \gls{brd} implementiert und soll den Ordnungs­rahmen ausweiten und die Befugnis des \gls{bsi} stärken sowie die gesetzliche Verankerung eines verbindlichen Informations­sicherheits­managements bieten.\footnote{
                \footcite[Vgl. S 1 - 2][]{NIS2UmsuCG}
                \footcite[Vgl. Absatz 1, 3, 5][]{EU2022-2555}
                \footcite[Vgl. S. 14 - 15][]{bsi-lage-itsicherheit-2023}
                \footcite[S. 9, 12 - 13 \& 37][postnote]{enisa-thread-landscape}
            } In Abschnitt \ref{sec:NIS2UmsuCG} sind die \emph{betroffene Organisationen und Subjekte}, die \emph{zuständige Aufsichtsbehörden} sowie die \emph{erforderliche Maßnahmen}, das \emph{Registrierungs- und Meldewesen} und die \emph{Risiken und Sanktionen} detailliert beschrieben.

            \subsection{KRITIS-Dachgesetz (KRITIS-DachG)}
                Das \gls{kritis-dachg} soll die physische Resilienz von Betreibern kritischer Einrichtungen Regeln. Hierbei gibt das \gls{kritis-dachg} Resilienzziele, Orientierung und eine Übersicht von beispielhaften Maßnahmen vor. Ebenfalls ist dies die nationale Rechtsprechung zur Erfüllung der \emph{Richtlinie \gls{eu} 2022/2557}. Entgegen dem \gls{nis2umsucg} liegt hier der Fokus auf die nicht-IT-bezogene Maßnahmen zur Stärkung der Resilienz dieser Einrichtungen.
                \subsubsection{Betroffene Organisationen und Subjekte}
                Das \gls{kritis-dachg} findet Anwendung für alle Betreiber kritischer Anlagen in den Sektoren Energie, Transport und Verkehr, Finanzwesen, Leistungen der Sozialversicherung sowie Grundsicherung für Arbeitsuchende, Gesundheitswesen, Wasser, Ernährung, Informationstechnik und Telekommunikation, Weltraum und Siedlungsabfallentsorgung. Im Rahmen der \emph{Delegierten Verordnung \gls{eu} 2023/2450} wurde eine Ergänzung zu der Richtlinie der \gls{eu}, auf welcher das \gls{kritis-dachg} beruht, geschaffen. Dies ist eine nicht erschöpfende Liste aus wesentlichen Diensten der zuvor genannten Sektoren und unterteilt diese in weitere Untersektoren. Zusätzlich kann das \gls{bmi} durch Rechtsverordnungen kritische Dienstleistungen für die zuvor genannten Sektoren bestimmen. Dies geschieht im einvernehmen mit dem \gls{bmwk}, dem \gls{bmf}, dem \gls{bmj}, dem \gls{bmas}, dem \gls{bmvg}, dem \gls{bmleh}, dem \gls{bmg}, dem \gls{bmv}, dem \gls{bmbf} und dem \gls{bmukn}. Einrichtungen, die nicht unter das \gls{kritis-dachg} fallen, können dennoch als von erheblicher Bedeutung gelten, wenn die Betriebsfähigkeit kritischer Anlagen von ihnen abhängt .\footnote{
                    \footcite[Vgl. §4, Absatz 1,][]{KRITIS-DachG}
                    \footcite[Vgl. §4, Absatz 3 \& 4,][]{KRITIS-DachG}
                    \footcite[Vgl. Artikel 2, Nummer 1,][]{EU2023-2450}
                }
                \subsubsection{Zuständige Aufsichtsbehörden}
                Das \gls{bbk} ist die Zuständige Zentrale Anlaufstelle für die grenzüberschreitend Kommunikation mit anderen Mitgliedstaat im Zusammenhang mit der \emph{Richtlinie \gls{eu} 2022/2557}. Inlands werden diverse andere Organe als zuständige Organe für die betroffenen Einrichtungen benannt. Hier sind zunächst für alle betroffenen Einrichtungen das \gls{bbk} und \gls{bsi} hinsichtlich der Gemeinsam errichteten Meldestelle für Vorfälle relevant. Dies ist dieselbe Meldestelle wie im Rahmen des \gls{nis2umsucg} definiert. Für unter anderem die Erbringung der Nachweise über die Einhaltung der Resilienzpflichten kann das \gls{bmi} durch Rechtsverordnungen zuständige Behörden festlegen. Sofern weder durch \gls{kritis-dachg} noch durch eine Rechtsverordnung eine Zuständige Stelle bestimmt wurde ist es Aufgabe der Bundesländer jeweilig Zuständige Landesbehörden zu bestimmen. Im Rahmen des \gls{kritis-dachg} wird bereits das \gls{bnetza} für die Strom-, Energie und Wasserstoffversorgung sowie für die Betreiber von öffentlicher Telekommunikationsnetzen oder öffentlich zugänglichen Telekommunikationsdiensten als Zuständige Behörde identifiziert. Das \gls{bsi} wiederum ist für die Sprach- und Datenübertragung sowie Datenspeicherung und -verarbeitung, sofern diese nicht den Betrieb der öffentlichen Telekommunikationsnetzen oder Telekommunikationsdiensten dienen, zuständig. Für sämtliche Bodeninfrastrukturen von weltraumgestützter Dienste im Sektor Weltraum ist das \gls{bafa} die Zuständige Einrichtung. Oft beziehen sich die Zuständigkeiten nur auf Bundeseigene Objekte. So ist das \gls{eba} nur für die bundeseigenen Eisenbahnverkehrsunternehmen oder Eisenbahninfrastrukturunternehmen zuständig. Hier ebenfalls beschränkt ist die Zuständigkeit des \gls{fba} auf die Verkehrssteuerungs- und Leitsysteme sowie intelligente Verkehrssysteme auf Bundesautobahnen und -straßen in Bundesverwaltung. Auch die \gls{gwds} ist nur für bundeseigenen Wasserstraßeninfrastruktur zuständig. Für die Wasserstands- und Gezeitenvorhersage des Bundes wiederum ist das \gls{bsh} zuständig. Für die Allgemeine Wettervorhersage ist der Deutsche Wetterdienst verantwortlich, wobei sich dies auf die kritischen Dienstleistungen der in seiner Zuständigkeit befindlichen Wettervorhersagen bezieht. Ebenfalls sind weitreichendere Zuständigkeiten definiert, wie beispielsweise das \gls{bmwk} für die generelle Mineralölversorgung. Drittunternehmen, welche die kritischen Dienstleistung im Finanzsektor erbringen, werden durch die \gls{bafin} beaufsichtig, sofern diese nach Artikels 46 der \emph{Verordnung (EU) 2022/2554} die zuständige Einrichtung ist. Komponenten des Sozialsaats sind ebenfalls über das \gls{kritis-dachg} geregelt. So ist die Zuständigkeit für die Erbringung von Leistungen der Sozialversicherung gemäß den Leistungsträger nach dem Sozialgesetzbuch die zuständige Aufsichtsbehörden. Grundsicherung für Arbeitsuchende sowie Leistungen des Rechts der Arbeitsförderung werden durch die \gls{ba} abgedeckt.\footnote{
                    \footcite[Vgl. §2,][]{KRITIS-DachG}
                    \footcite[Vgl. §3, Absatz 1,][]{KRITIS-DachG}
                    \footcite[Vgl. §4, Absatz 1,][]{KRITIS-DachG}
                    \footcite[Vgl. §46, Absatz 1,][]{KRITIS-DachG}
                    \footcite[Vgl. §18, Absatz 1,][]{KRITIS-DachG}
                    \footcite[Vgl. §16, Absatz 1,][]{KRITIS-DachG}
                }
                \subsubsection{Erforderliche Maßnahmen}
                Betreiber kritischer Anlagen müssen Risikoanalyse und Risikobewertung durchführen. Grundlage dafür muss die nationalen Risikoanalysen und Risikobewertungen sowie anderer vertrauenswürdiger Informationsquellen sein. Durchzuführen ist diese immer im Bedarfsfall oder aber mindestens alle vier Jahre. Die Risikoanalysen und Risikobewertungen muss die folgenden Aspekte betrachten:
                \begin{itemize}
                    \item Naturbedingte, technische oder menschlich verursachte Risiken die geeignet sein können, die Verfügbarkeit zu beeinträchtigten
                    \item Sektorenübergreifende und grenzüberschreitende Risiken
                    \item Extremereignisse durch Unfälle, Naturgefahren und gesundheitliche Notlagen
                    \item Hybride Bedrohungen, sicherheitsgefährdende oder andere feindliche Bedrohungen, einschließlich terroristischer Straftaten
                    \item Abhängigkeit zu anderen Betreibern kritischer Anlagen, auch in anderen Sektoren und Mitgliedstaaten
                    \item Ausmaß der Abhängigkeiten anderer Sektoren von der kritischen Dienstleistung
                \end{itemize}
                Handelt es sich um eine maritime Infrastruktur muss dessen Besonderheit in der Risikoanalysen und Risikobewertungen berücksichtigt werden. Das \gls{bmi} kann darüberhinaus weitere Vorgaben, einschließlich Vorlagen und Muster für die Risikoanalysen und Risikobewertungen, bestimmen.\footnote{
                    \footcite[Vgl. §11, Absatz 2,][]{KRITIS-DachG}
                    \footcite[Vgl. §12,][]{KRITIS-DachG}
                } Verpflichtend ist es Maßnahmen zur Gewährleistung seiner Resilienz zu treffen. Damit sollen das Auftreten von Vorfällen verhindert sowie ein angemessener physischer Schutz gewährleistet werden. Ebenfalls soll somit auf Vorfälle reagiert, diese abgewehrt und die negativen Auswirkungen begrenzt werden. Unter anderem soll somit auch eine zügige Wiederherstellung der kritischen Dienstleistungen gewährleistet werden. Für die Gewährleistung der Resilienz ist eine Notversorgung zu etablieren, Objektschutz zu gewährleisten sowie die Umgebung der Liegenschaften zu überwachen. Mit Einsatz von Dedektionsgeräten und Zugangskontrollen sowie entsprechenden Abläufen im Alarmfall trägt man der Resilienz bei. Ebenfalls sind Maßnahmen zur Aufrechterhaltung des Betriebs, wozu unter anderem die Ermittlung alternativer Lieferketten zählen, erforlderich.Die Resilienzanforderungen stellen sich auch an die Mitarbeitenden und Personals externer Dienstleister. Diese müssen ein angemessenes Sicherheitsmanagement erfahren und mittels Informationsmaterialien, Schulungen und Übungen befähigt werden dies durchzuführen. Sämtliche Aspekte sind in einen Resilenzplan aufzufassen, damit aus diesem alle Notwendigen Maßnahmen hervorgehen. Basis für diesen Resilienzplan sind die Risikoanalyse und Risikobewertung der betroffenen Einrichtung. Alle diese Maßnahmen hinsichtlich ihrer Umsetzung müssen alle drei Jahre in From von Sicherheitsaudits, Prüfungen oder Zertifizierungen nachgewiesen werden. Die Geschäftsleitungen ist in allen Fällen dazu verpflichtet die Umsetzung der zu ergreifenden Resilenzmaßnahmen durch geeignete Organisationsmaßnahmen zu gewährleisten.\footnote{
                    \footcite[Vgl. §13,][]{KRITIS-DachG}
                    \footcite[Vgl. §14, Absatz 2,][]{KRITIS-DachG}
                    \footcite[Vgl. §16,][]{KRITIS-DachG}
                    \footcite[Vgl. §20, Absatz 1][]{KRITIS-DachG}
                    \footcite[Vgl. §39, Absatz 1,][]{NIS2UmsuCG}
                }
                \subsubsection{Registrierungs- und Meldewesen}
                Für betroffenen Einrichtungen besteht eine Registrierungspflicht binnen drei Monaten nachdem die kritische Anlage erstmals als solche eingestuft wurde. Die Registrierung ist bei der vom \gls{bbk} und \gls{bsi} gemeinsam eingerichtetn Registrierungsmöglichkeit einzureichen, jedoch frühsten bis einschließlich 17. Juli 2026. Die Registrierung umfasst Name sowie Rechtsform und, wenn vorhanden, Handelsregisternummer, die aktuellen  Kontaktdaten einschließlich E-Mail-Adresse, öffentlichen IP-Adressbereiche und Telefonnummer, der Sektor der Einrichtung sowie die Brache, sofern einschlägig bekannt und die kritische Dienstleistung, für deren Erbringung die Anlage erheblich ist. Zusätzlich sind Standort und Versorgungsgebiet sowie  Werte zum Versorgungsgrad und, sofern einschlägig, die Kategorie der kritischen Anlage anzugeben. Eine Kontaktstelle, über die der Betreiber kritischer Anlagen erreichbar ist, ist ebenfalls Bestandteil der Registrierung. Nachdem eine Einrichtung sich Registriert hat werden dieser durch das \gls{bbk} zuständige Behörde mitgeteilt und die zu ergreifenden Maßnahmen gelten frühestens nach neun Monaten erstmals.\footnote{
                    \footcite[Vgl. §5, Absatz 1,][]{KRITIS-DachG}
                    \footcite[Vgl. §8,][]{KRITIS-DachG}
                }
                Meldungen sind ebenfalls an die gemeinsam vom \gls{bbk} und \gls{bsi} eingerichtet Meldestelle zu richten. Vorfälle sind spätestens nach 24 Stunden der ersten Kenntnisnahme anzuzeigen. Ein ausführlicher Bericht ist binnen einem Monat an die Meldestelle zu richten. Die Meldungen müssen alle zu dem Zeitpunkt vorliegenden Informationen beinhalten, sodass die Art, Ursache und mögliche, auch grenzüberschreitende, Auswirkungen und Folgen des Vorfalls ermittel- und nachvollziehbar sind. Unter anderem Zählen dazu die bisherige sowie voraussichtliche Dauer des Vorfalls, das betroffene geografische Gebiet und die Anzahl der Betroffenen (absolut) sowie die Größe des Anteils des ganzen (relativ).\footnote{
                    \footcite[Vgl. §18,][]{KRITIS-DachG}
                    \footcite[Vgl. §32, Absatz 1,][]{NIS2UmsuCG}
                }

                \subsubsection{Risiken und Sanktionen}
                Aus \gls{kritis-dachg} geht für die Geschäftsleitung ein unmittelbares Risiko hervor. Setzen diese die erforderlichen Maßnahmen nicht um haftet diese entsprechend der Rechtsform der Einrichtung anwendbaren Regeln des Gesellschaftsrechts. Somit haftet ein Gesellschafter einer \gls{gmbh} potentiell gegenüber ihrer Gesellschaft, da in diesem Fall die schuldhaftender Pflichtverstoß vorliegt. Ein Verstoß gegen das \gls{kritis-dachg} liegt dann vor, wenn die Registrierungspflicht nicht ordnungsgemäße eingehalten oder die Nachweispflicht nicht oder nicht richtig, nicht vollständig oder nicht rechtzeitig erfolgt. Die Haftung bei einem verstoß schränkt sich auf Geldstrafen in dem Bereich 50 000 Euro bis 500 000 Euro.\footnote{
                    \footcite[Vgl. §20, Absatz 2][]{KRITIS-DachG}
                    \footcite[Vgl. §24, Absatz 2][]{KRITIS-DachG}
                    \footcite[Vgl. §13, Absatz 2][]{GmbHG}
                    \footcite[Vgl. §43, Absatz 2][]{GmbHG}
                }

    \section{NIS-2-Umsetzungs- und Cybersicherheitsstärkungsgesetz}\label{sec:NIS2UmsuCG} % NOTE: Done on 24.07.2025 
        In dem Folgenden Abschnitt sind wichtige Punkte des \gls{nis2umsucg} beschrieben. Hierzu zählen die \emph{betroffene Organisationen und Subjekte}, die entsprechend \emph{zuständigen Aufsichtsbehörden} sowie die zu \emph{ergreifenden Maßnahmen} und das \emph{Registrierungs- und Meldewesen} sowie die aus dem Gesetzt entstehenden \emph{Risiken und Aktionen}. Sämtliche Maßnahmen, beschrieben in den folgenden Unterabschnitten, sind der zuständigen Behörde alle drei Jahre durch Sicherheitsaudits, Prüfungen oder Zertifizierungen nachzuweisen.\footnote{
            \footcite[Vgl. §39 Absatz 1,][]{NIS2UmsuCG}
        }
        \subsection{Betroffene Organisationen und Subjekte}
        Organisationen und Subjekte die nach \gls{nis2umsucg} als wichtige Einrichtung oder besonders wichtige Einrichtung klassifiziert, oder als Betreiber einer Anlage eingestuft werden, sind von \gls{nis2umsucg} betroffen. Darüberhinaus sind \gls{dns}-Diensteanbieter, \gls{tld}-Namenregister, Anbieter von Cloud-Computing-Diensten, Anbieter von Rechenzent­rumsdiensten, Betreiber von \gls{cdn}, Anbieter verwalteter Dienste, Anbieter verwalteter Sicherheitsdienste, Anbieter von Online-Marktplätzen, Online-Suchmaschinen und Plattformen für Dienste sozialer Netzwerke und Vertrauensdiensteanbieter verpflichtet die \emph{Durchführungsverordnung (EU) 2024/2690} umzusetzen.\footnote{
            \footcite[Vgl. §30 Absatz 1, 2 \& 3,][]{NIS2UmsuCG}
            \footcite[Vgl. Absatz 1,][]{EU2024-2690}
        }\medbreak
        Besonders wichtige Einrichtungen sind jene, die eine kritische Anlage betreiben. Die Vorraussetzungen für eine kritische Anlage sind im Unterabschnitt \ref{def:BetreiberKritischerAnlage} nähergehend beschrieben. Ist die Einrichtung ein qualifizierte Vertrauensdiensteanbieter, \gls{tld} Name Registries oder \gls{dns}-Diensteanbieter gilt diese unmittelbar als besonders wichtige Einrichtung. Alle Anbieter von öffentlich zugänglicher Telekommunikationsdienste oder Einrichtungen, welche ein öffentliches Telekommunikationsnetz betreiben, sind besonders wichtige Einrichtungen, wenn diese mindestens fünfzig Mitarbeiter beschäftigen oder jeweils einen Jahresumsatz sowie eine Jahresbilanzsumme von über 10 Millionen Euro aufweisen. Zusätzlich sind alle weiteren natürliche oder juristische Personen sowie rechtlich unselbstständige Organisationseinheiten einer Gebietskörperschaft, welche entgeltlich Waren oder Dienstleistungen Anbieten, sowie mindestens 250 Beschäftigten oder einem Jahresumsatz von 43 Millionen Euro und 50 Millionen Euro Jahresbilanzsumme, sofern diese in Anlage 1 des \gls{nis2umsucg} aufgeführt sind.\footnote{
            \footcite[Vgl. §28 Absatz 1,][]{NIS2UmsuCG}
        }\medbreak
        Als wichtige Einrichtung gilt man als Vertrauensdiensteanbieter oder Anbieter von öffentlich zugänglicher Telekommunikationsdienste sowie Betreiber öffentlicher Telekommunikationsdienste. Zusätzlich zu Anlage 1 des \gls{nis2umsucg} ist Anlage 2 für alle weiteren natürliche oder juristische Personen sowie rechtlich unselbstständige Organisationseinheiten einer Gebietskörperschaft zu betrachten. Hier besteht nun die Anforderung das mindestens 50 Mitarbeiter beschäftigt werden und einen Jahresumsatz sowie eine Jahresbilanzsumme von über 10 Millionen Euro vorliegt.\footnote{
            \footcite[Vgl. §28 Absatz 2,][]{NIS2UmsuCG}
        }\medbreak
        Für die Bestimmung der Anzahl der Mitarbeiter sowie Jahresumsatz und Jahresbilanzsumme von natürliche und juristische Personen sind alle Partner- oder verbundenen Unternehmen zu betrachten. Hiervon sind alle Partner- oder verbundenen Unternehmen ausgeschlossen dessen Betrieb der informationstechnischen Systeme, Komponenten und Prozesse unabhängig der im Fokus stehenden natürliche und juristische Personen ist. Ebenfalls sind bei Betrachtung der Geschäftstätigkeiten in Hinblick auf Anlage 1 und Anlage 2 des \gls{nis2umsucg} sämtliche Tätigkeiten zu vernachlässigen, welche in der Gesamtheit der Geschäftstätigkeit der natürliche oder juristische Personen zu vernachlässigen sind.\footnote{
            \footcite[Vgl. §28 Absatz 4 \& 3,][]{NIS2UmsuCG}
        }\medbreak
        Darüberhinaus sind direkte direkten Anbietern und Diensteanbieter sowie Erzeuger und Anbieter von IKT-Produkten und -Diensten unmittelbar betroffen. Dies bedingt die Anforderung an die Sicherheit der Lieferketten sowie Anforderungen zu Erwerb, Entwicklung und Wartung von Netz- und Informationssystemen. Ebenfalls sind im \gls{nis2umsucg} Restriktionen und Anforderungen zum Einsatz von kritischen Komponenten genannt. Einrichtungen die ein gewisses maß an Informationssicherheit nicht erfüllen oder nachweisen können sind potentiell als Lieferanten für von \gls{nis2} und dem \emph{Durchführungsverordnung (EU) 2024/2690} ausgeschlossen.\footnote{
            \footcite[Vgl. §30 Absatz 6,][]{NIS2UmsuCG} % NOTE: Besonders wichtige Einrichtungen und wichtige Einrichtung dürfen durch Rechtsverordnung nach §56 Absatz 3 bestimmte IKT-Produkte, IKT-Dienste und IKT-Prozesse nur verwenden, wenn diese über eine Cybersicherheitszertifizierung gemäß europäischer Schemata nach Artikel 49 der Verordnung (EU) 2019/881 verfügen. 
            \footcite[Vgl. §56 Absatz 3,][]{NIS2UmsuCG}
            \footcite[Vgl. Nummer 5.1.1. \& 5.1.2.,][, Anhang]{EU2024-2690}
            \footcite[Vgl. Nummer 5.1.4. \& 5.1.5.,][, Anhang]{EU2024-2690}
            \footcite[Vgl. Nummer 6.1.1. \& 6.1.2.,][, Anhang]{EU2024-2690}
            \footcite[Vgl. Nummer 6.1.1. \& 6.1.2.,][, Anhang]{EU2024-2690}
            \footcite[Vgl. Nummer 6.2.2 \& 6.2.3.,][, Anhang]{EU2024-2690}
        }

        \subsection{Zuständige Aufsichtsbehörden}
        Allgemein ist das \gls{bsi} die zuständige Aufsichtsbehörde für Einhaltung der Vorschriften in Bezug auf die Bundesverwaltungen. Für wichtige und besonders wichtige Einrichtung, welche in der \gls{brd} niedergelassen sind, ist das \gls{bsi} ebenfalls zuständig. Befindet sich eine kritische Anlage auf dem Hoheitsgebiet der \gls{brd} ist das \gls{bsi} die zuständige Aufsichtsbehörde für den Betreiber dieser kritischen Anlage.\footnote{
            \footcite[Vgl. §59,][]{NIS2UmsuCG}
        }\medbreak
        Von der \emph{Durchführungsverordnung (EU) 2024/2690} betroffene Einrichtungen sowie Domain-Name-Registry-Dienstleister werden vom \gls{bsi} nur dann beaufsichtigt wenn dessen Hauptnierderlassung innerhalb der \gls{brd} und der \gls{eu} ist. Ist dies der Fall so ist das \gls{bsi} für die gesamte Einrichtung innerhalb der \gls{eu} Zentral zuständig. Die Hauptniederlassung bestimmt sich im Zusammenhang mit den Maßnahmen zum Cybersicherheitsrisikomanagement. Der Mitgliedstaaten der \gls{eu}, in welchem hauptsächlich die Entscheidungen für das Cybersicherheitsrisikomanagement getroffen werden, ist als Hauptniederlassung anzusehen. Werden diese Entscheidungen außerhalb der \gls{eu} getroffen gilt als Hauptniederlassung der Mitgliedstaat in dem die Cybersicherheitsrisikomanagement durchgeführt werden. Ist die Hauptniederlassung mittels dieses Vorgehens nicht ermittelbar, so ist die Hauptniederlassung die Niederlassung mit der höchsten Beschäftigten Zahl innerhalb der \gls{eu}.\footnote{
            \footcite[Vgl. §60, Absatz 1 \& 2 ][]{NIS2UmsuCG}
        }
        Einrichtungen die Dienste innerhalb der \gls{eu} Anbieten aber keinerlei Niederlassung haben müssen einen Vertreter benennen. Dieser muss sich deinem Mitgliedstaat der \gls{eu} befinden, in welchem die Dienste Angeboten werden. Ist dies die \gls{brd} so ist das \gls{bsi} zuständig. Wird kein Vertreter benannt so kann das \gls{bsi} sich für die betreffende Einrichtung als zuständig erklären. Der in der \gls{eu} benannte Vertreter bleibt von rechtliche Schritte, welche gegen die Einrichtung eingeleitet werden, unberührt.\footnote{
            \footcite[Vgl. §60, Absatz 3 \& 4 ][]{NIS2UmsuCG}
        }\medbreak

        \subsection{Erforderliche Maßnahmen}
        % TODO: Konkreten / Spezifische Kapitel ergänzen
        % Nummer 6 <- NIS2UmsuCG §38, Umsetzungs-, Überwachungs- und Schulungspflicht für Geschäftsleitungen besonders wichtiger Einrichtungen und wichtiger Einrichtungen 
        % Nummer 8 <- NIS2UmsuCG §39, Nachweispflichten für Betreiber kritischer Anlagen
            Aus \gls{nis2umsucg} sowie der \emph{Durchführungsverordnung (EU) 2024/2690} gehen diverse Anforderungen hervor, welche auf Basis europäischen und internationalen Normen beruhen. Die erforderlichen Maßnahmen sind in den folgenden Unterabschnitten beschrieben und in ihrer Gänze ausgeführt. Die \emph{Durchführungsverordnung (EU) 2024/2690} sieht zusätzlich vor das Unternehmen die Möglichkeit haben Ausgleichsmaßnahmen zu ergreifen, die geeignet sind diese Anforderungen zu erfüllen, wenn wegen ihrer Größe die Umsetzung nicht Verhältnismäßig ist. Ebenfalls ist es möglich bestimmte Anforderungen nicht zu erfüllen, wenn diese als nicht angemessen, anwendbar oder durchführbar erscheinen. Abweichungen sind folglich zu dokumentieren.\footnote{
                \footcite[Vgl. Absatz 3,][]{EU2024-2690}
                \footcite[Vgl. Absatz 5,][]{EU2024-2690}
                \footcite[Vgl. Absatz 6,][]{EU2024-2690}
            } Sämtliche beschriebene Anforderungen der \emph{Durchführungsverordnung (EU) 2024/2690} sind in aller Regel in einem zu definierenden Turnus zu Überprüfen und das Ergebnis zu dokumentieren, spätestens jedoch nach einem erheblichen Sicherheitsvorfall, einer erneuten Risikobewertung oder aber der wesentlichen  Änderungen  der  Betriebsabläufe.\footnote{
                \footcite[Vgl. 1.1.2., 1.2.6., 2.1.4. , 2.2.3., 2.3.4., 3.1.3., 3.2.7., 3.5.5., 3.6.3., 4.1.4. , 4.2.6. , 4.3.4., 5.1.6., 6.1.3., 6.2.4., 6.3.3., 6.4.4., 6.5.3., 6.7.3. , 6.8.3., 6.10.4. , 7.3., 8.2.5. , 9.3., 10.1.3., 10.2.3., 10.4.2., 11.1.3. , 11.2.3., 11.3.3. , 11.5.4., 11.6.4. , 12.1.3., 12.2.3. , 12.3.3. , 12.4.3. , 13.1.3. , 13.2.3. und 13.3.3.,][, Anhang]{EU2024-2690}
            }
            

            % Nummer 1: ISMS ↓
            \subsubsection{Konzept für die Sicherheit von Netz- und Informationssystemen}
            Teil der zu ergreifenden Maßnahmen ist ein \emph{Konzept für die Sicherheit von Netz- und Informationssystemen} sowie darin enthaltenen themenspezifischen Konzepte, welche sich auf die betroffene Einrichtung beziehen. Zentraler Bestandteil ist die Festlegung der Verantwortlichkeiten sowie Weisungsbefugnisse hinsichtlich der Sicherheit von Netz- und Informationssysteme. Diese Festlegung wird ebenfalls den Leitungsorganen mitgeteilt, sodass ein einheitliches Verständnis der definierten Verantwortlichkeiten und Weisungsbefugnisse herrscht. Sofern es möglich ist sollen wiedersprechende Verantwortlichkeiten vermieden werden. Zusätzlich muss im selben Rahmen eine Person definiert werden, welche den Leitungsorganen gegenüber als Ansprechpartner für Fragen bezüglich der Sicherheit von Netz- und Informationssystemen verantwortlich ist. Das Konzept an sich muss diverse Themengebiete abdecken. Ein Kernapsekt ist die Beschreibung des Ansatzes des Managements der Sicherheit in Netz- und Informationssystemen. In diesem Kontext bezieht sich Management auf die Verwaltung der Verfahren und Regeln für die Sicherheit in Netz- und Informationssystemen. Wichtig ist das die Geschäftsstrategie durch das Konzept ergänzt und für die Unternehmerischen Ziele geeignet sind. Ebenfalls ist ein Konzept genau wie die Einrichtung nicht statisch sondern dynamisch. Daher bedarf es einer Verpflichtung im Rahmen des Konzeptes dieses Kontinuierlich zu verbessern und erweitern. Ebenfalls bedarf es einer Verpflichtung die Ressourcen für die eingangs beschriebenen Rollen sowie damit verbundenen Finanzen, Verfahren, Instrumente und Technologien bereitzustellen. Diese Festgelegten Rollen sind ebenfalls Teil des Konzeptes. Zusätzlich sind sämtliche betroffenen aufzubewahrenden Unterlagen und die Dauer ihrer Aufbewahrung aufführen. Im Kontext der \gls{brd} sind hier unter anderem die \gls{ao}, das \gls{hgb} sowie die \gls{gobd} maßgeblich. Abschließend muss entsprechend beschrieben sein wie die Wirksamkeit hinsichtlich der Umsetzung des Konzeptes überwacht wird. Das Konzept an sich muss allen betroffenen Mitarbeitern sowie interessierten externen Beteiligten verfügbar gemacht werden. Ebenfalls ist dies durch diese Personengruppen anzuwenden. \footnote{
                \footcite[Vgl. Nummer 1,][, Anhang]{EU2024-2690}
                \footcite[Referenz für Unternehmen die sich Wandeln][]{MISSING}
            }

            % Nummer 12: Anlagen- und Wertemanagement | ITAM, Anlagenverzeichnis, Prozessmanagement Platform, DMS, CMDB, Monitoring
            % \emph{Anlagen- und Wertemanagement}
            \subsubsection{Anlagen- und Wertemanagement}
            Das \emph{Konzept für die Sicherheit von Netz- und Informationssystemen} billeted unter anderem ein zentrales Gerüst für das \emph{Anlagen- und Wertemanagement}. Erforderliche ist es ein vollständiges Verzeichnis über die Anlagen und Werte zu erstellt, welches ein aktuelles und kohärentes Inventar der Anlagen und Werte wiederspiegelt. Die Granularität hängt von den Bedrüfnissen der Einrichtung ab und umfasst zumindest die Liste der Betriebsabläufe und Dienste und ihre Beschreibung sowie eine Liste der Netz- und Informationssysteme und anderer zugehöriger Anlagen und Werte, die die Abläufe und die Dienste der betreffenden Einrichtungen unterstützen. Sämtliche Änderungen sind nachvollziehbar zu Dokumentieren. Alle Anlagen und Werte inklusive Informationen des Bereich Netz- und Informationssysteme müssen mittels eines System Klassifizierungsstufen für die erforderlichen Schutzniveau zugeordnet werden. Dieses System beruht auf Vertraulichkeits-, Integritäts-, Authentizitäts- und Verfügbarkeitsanforderungen um das Schutzniveau auf Basis der Sensibilität, ihrer Kritikalität, ihres Risikos und ihres Geschäftswerts zuzuordnen. Die Verfügbarkeitsanforderungen leiten sich aus Notfallplan ab, welcher die Aufrechterhaltung und Wiederherstellung des Betriebs unter bestimmten Bedingungen gewährleisten soll. Ebenfalls Teil des \emph{Anlagen- und Wertemanagement} ist ein Konzept zur ordnungsgemäße Behandlung von Anlagen und Werten die allen betroffenen Personen bekannt sind und auf dem Konzept für die Sicherheit der Netz- und Informationssysteme beruht. Hierbei werden Aspekte des gesamten Lebenszyklus der Anlagen und Werte (Erwerb, Verwendung, Speicherung, Transport und Entsorgung) abgedeckt und die sichere Verwendung, Speicherung und der Transport sowie die Unwiederbringliche Löschung und Vernichtung definiert. Bei Beendigung des Beschäftigungsverhältnisses müssen die betreffenden Anlagen und Werte abgegeben, zurückgegeben oder gelöscht werden, und die Abgabe, Rückgabe und Löschung dokumentiert werden. Ist das nicht möglich muss sichergestellt sein das der Zugriff auf Netz- und Informationssysteme nicht mehr möglich ist. Gesondert wird ein Konzept für das Management von Wechseldatenträgern und wie diese an Orten, an dem die Wechseldatenträger mit den Netz- und Informationssystemen der betreffenden Einrichtungen verbunden sind, zu nutzen sind gefordert. Bestandteil hiervon sind die Sperrung von nicht notwendigen Wechseldatenträger sowie das Scannen auf Schadcode und Verhindern von Programmen, welche sich automatisch ausführen. Notwendige Wechseldatenträger sind zu Verschlüsseln und Maßnahme zur Kontrolle und dem Schutz einzuführen.\footnote{
                \footcite[Vgl.Nummer 12,][, Anhang]{EU2024-2690}
            }

            % Nummer 2: Risikomanagement ↓    
            \subsubsection{Konzept für das Risikomanagement}
            Das \emph{Konzept für die Sicherheit von Netz- und Informationssystemen} sowie das \emph{Anlagen- und Wertemanagement} sind Grundlage für das \emph{Konzept für das Risikomanagement}. Risiken müssen kontinuierlich Identifizierung, Analysiert, Bewertet, Bewältigt und überwacht werden. Ein Essenzieller Kernaspekt ist die Tatsache das sich das Risikomanagement für die Sicherheit von Netz- und Informationssystemen im Einvernehmen mit dem ganzheitlichen Risikomanagement der betroffenen Einrichtung befindet. Bestandteile des Risikomanagement sind der Risikomanagementprozess zur Identifizierung, Analyse, Bewertung, Bewältigung und Überwachung von Risiken hinsichtlich der Sicherheit von Netz- und Informationssystemen. Hierbei werden Risikomanagementmethodiken auf die ermittelten Risiken angewendet um eine Risikostrategie abzuleiten. Eine Zentrale Rolle spielt die Risikoüberwachung sowie die Risikokommunikation. Formal ist es erforderlich für den Risikomanagementprozess Verfahren zu definieren die Risikomanagementmethodiken sowie die Risikotoleranzschwelle im Einklang mit der Risikobereitschaft umfassen. Der zuvor beschriebene kontinuriliche Prozess der Identifizierung, Analyse, Bewertung und Behandlung von Risiken beinhaltet ebenfalls die Risikokriterien, welche ebenfalls für die Überwachung und eine kontinuriliche Verbesserung herangezogen werden. Neben den Verfahren zur Bewertung von Risiken und den entsprechend daraus folgenden Maßnahmen sind ebenfalls Verfahren zu definieren, welche Personen die Verantwortlichkeit für die Terminierte Umsetzung der Maßnahmen zuordnet. Zusätzlicher Bestandteil des Risikomanagement für die Sicherheit von Netz- und Informationssystemen ist die regelmäßige und unabhängige Überprüfung der \emph{Konzept für die Sicherheit von Netz- und Informationssystemen} inklusive geeigneter Systeme zur Berichterstattung der Ergebnisse für die Leitungsorganen. Diese sollen somit einen fundierten Überblick über den aktuellen Stand des Risikomanagements abrufen können.\footnote{
                \footcite[Vgl. Nummer 2,][, Anhang]{EU2024-2690}
                \footcite[Risikomanagement in Unternehmen allgegenwärtig][]{MISSING}  
                \footcite[Vgl. §30 Absatz 1 und Absatz 2, Nummer 1,][]{NIS2UmsuCG}              
            }

            % Nummer 7: Wirksamkeit von Risikomanagementmaßnahmen ↓
            \subsubsection{Konzepte und Verfahren zur Bewertung der Wirksamkeit von Risikomanagementmaßnahmen im Bereich der Cybersicherheit}
            Um die Ergebnisse des \emph{Konzept für das Risikomanagement} zu bewerten ist ein \emph{Konzepte und Verfahren zur Bewertung der Wirksamkeit von Risikomanagementmaßnahmen im Bereich der Cybersicherheit} erforderlich. Hierfür muss bestimmt werden welche Verfahren und Kontrollen zur Überwachung und Messung der Risikomanagementmaßnahmen angewendet werden. Auch die Methoden zur Überwachung, Messung, Analyse und Bewertung für die Gewährleistung gültiger Ergebnisse ist Teil des Konzept. Die Wirksamkeit muss regelmäßig geprüft werden, weshalb der Turnus sowie die Bedingung für eine Bewertung spezifiziert und die Zeitpunkt und Verantwortlichkeiten bis wann Ergebnisse Analysiert und bewertet werden müssen definiert wird.\footnote{
                \footcite[Vgl. Nummer 7,][, Anhang]{EU2024-2690}
                \footcite[Vgl. §30 Absatz 2, Nummer 6,][]{NIS2UmsuCG}
            }

            % Nummer 3: Protokollierung, Sicherheitsvorfälle & Monitoring ↓
            \subsubsection{Bewältigung von Sicherheitsvorfällen}
            Ein Teilaspekt des Risikomanagement ist die prevention von Sicherheitsvorfällen. Nicht jeder Sicherheitsvorfall lässt sich präventiv vermeiden, weswegen Verfahren zur \emph{Bewältigung von Sicherheitsvorfällen} ein erforderlicher Teil der Maßnahmen ist. Vorhergehend ist das Konzept zur Bewältigung von Sicherheitsvorfällen. In diesem werden Rollen, Verantwortlichkeit und Verfahren für die zeitnahe Erkennung, Analyse, Eindämmung oder Reaktion auf Sicherheitsvorfälle sowie die Wiederherstellung nach einem Sicherheitsvorfall festgelegt. Wie ein Sicherheitsvorfall zu Melden und bei Bedarf zu eskalieren ist, ist ebenfalls Bestandteil des Konzeptes. Einem Sicherheitsvorfall vorangestellt ist ein Ereignis. Dieses muss zunächst auf Basis eines Kategorisierungs- und Klassifizierungssystems als Sicherheitsvorfall eingestuft werden. Diese beiden Aspekte sind entsprechend ebenfalls Bestandteil des Konzeptes. Damit die Vorgehen widerspruchsfrei sind werden sämtliche Dokumente, wie Verfahren, Anleitungen, Vorlagen und der gleichen ebenfalls als Teil des Konzeptes mit aufgefasst. Dieses Konzept muss entsprechend im Einklang mit dem Notfallplan für die Aufrechterhaltung und Wiederherstellung des Betriebs der Einrichtung stehen. Damit Ereignisse leichter erkannt werden können sind Verfahren und Instrumente für die Überwachung und Protokollierung von Aktivitäten in den Netz- und Informationssystem zu implementieren. Diese funktionieren soweit möglich automatisch und basieren auf einem einheitlichen Zeitgeber, sodass die Protokolle aus diversen Systemen in eine Korrelation gebracht werden können. Hierbei ist die Überwachung und Protokollierung nur auf Systeme anzuwenden, welche per Definition gemäß der Anlagen und Werte auf Basis der Risikobewertung ein entsprechendes Vorgehen erforderlich machen. Die angefertigten Protokolle werden entsprechend eines definierten Zeitraumes Archiviert. Die Protokolle umfassen die folgenden Ereignisse:\footnote{
                \footcite[Vgl. Nummer 3.1 - 3.2.3, 3.2.5 \& 3.2.6,][, Anhang]{EU2024-2690}
                \footcite[Risikomanagement soll Sicherheitsvorfällen präventiv vermeiden][]{MISSING}
                \footcite[Sicherheitsvorfälle kann man nicht immer vermeiden][]{MISSING}
                \footcite[Vgl. §30 Absatz 2, Nummer 2,][]{NIS2UmsuCG} % Sicherheitsvorfälle
            }
            \begin{itemize}
                \item Logischer und physikalischer Zugang (Ein- und Ausgehend)
                \item Ereignisprotokolle und Protokolle von Sicherheitslösungen wie beispielsweise Firewalls oder Endpunktsicherheitslösungen
                \item Erstellen, Ändern und Löschen sowie Erweiterung der Rechte von Benutzerkonten
                \item Zugriffe auf Systeme und Anwendungen sowie die Berechtigungsstufe (privilegierter Zugriff)
                \item Ereignisse mit Bezug auf Authentifizierung
                \item Sämtliche Aktivitäten in Bezug auf privilegierte Benutzerkonten (Verwaltungskonten)
                \item Ereignisse des Umfelds der Systeme
                \item Systemressourcen
                \item Aktivierung, Beendigung und Pausieren der verschiedenen Protokolle
            \end{itemize} 
            Diese Protokolle werden regelmäßig auf Trends untersucht und es werden auf Basis geeigneter Schwellwerte Alarme ausgelöst. Ereignisse können auch durch Meldung Dritte, beispielsweise Personal, Kunden oder aber auch Anbieter, erfolgen. Hierzu ist ein entsprechender Mechanismus bereitzustellen und, bei Bedarf, die Betroffenen Personen zu unterrichten. Sowohl bei automatisiert (Protokollierung) erfassten als auch bei manuell (Meldung) erfassten Ereignissen sind Vorgehen erforderlich um zu ermittelten ob bei dem Ereignis ein Sicherheitsvorfall vorliegt. Für diese Feststellung ist ein Vorgehen zu definieren, welches die Bewertung auf Grundlage festgelegt Kriterien ermöglicht. Diese Kriterien bieten dann die Möglichkeit auf Basis von Meldungen oder Protokollen sowie die Korrelation und Analyse von Protokollen ein Ereignis zu Bewerten und bei Bedarf als Sicherheitsvorfall einzustufen. Ebenfalls Teil des Vorgehens sind die Durchführung einer Triage um die Gesamtheit der Eindämmung und Beseitigung der Sicherheitsvorfälle zu priorisieren. Auch eine erneute Bewertung von Ereignissen ist vorgesehen, etwa dann wenn neue Informationen vorliegen oder bereits vorliegende Informationen dies erforderlich machen. Eine vierteljährliche Bewertung ob wiederholte Ereignisse vorliegen ist der Abschließende Bestandteil der zu definierenden Vorgehen. Hiermit sollen kumuliert sich wiederholende Ereignisse betrachtet werden um diese ebenfalls als Sicherheitsvorfall einstufen zu können. Wird ein Ereignis als Sicherheitsvorfall klassifiziert ist dies Zeitnahe und gemäß Dokumentierter Verfahren in den drei Phasen der Eindämmung, Beseitigung und Wiederherstellung zu bearbeiten. Ein essenzieller Bestandteil ist die erstellung von Plänen und verfahren zur Kommunikation mit dem \gls{csirt} oder den zuständigen Behörden im Zusammenhang mit der Meldung von Sicherheitsvorfällen. Auch die in- und externen Kommunikationswege für Personal und Interessenträger außerhalb der Einrichtung ist ein Wichtiger Bestandteil. Jegliche Reaktion sowie die Tätigkeiten im Zusammenhang mit einem Sicherheitsvorfall sind zu Dokumentieren und diese Verfahren gilt es ebenfalls zu Überprüfen. Alle Sicherheitsvorfälle müssen Abschließend hinsichtlich ihrer Ursachen überprüft werden um diese in Zukunft zu vermeiden und bei Bedarf Schulungsmaßnahmen aus diesem Sicherheitsvorfall abzuleiten. Sämtliche Überprüfung, auch die Überprüfung ob ein Sicherheitsvorfall abschließend überprüft wurde, dienen dem Zwek der kontinuierlichen verbesserung. Dies beschränkt sich in dem Fall nicht ausschließlich auf das Konzept zur \emph{Bewältigung von Sicherheitsvorfällen}, sondern auch auf alle erforderlichen Maßnahmen wie beispielsweise das \emph{Risikomanagement}.\footnote{
                \footcite[Vgl. Nummer 3.2.4 \&1 3.3 - 3.6.2,][, Anhang]{EU2024-2690}
                \footcite[Vgl. Artikel 4,][]{EU2024-2690}
            }

            % Nummer 5: Backup, BCM 
            \subsubsection{Betriebskontinuitäts- und Krisenmanagement}
            Um den Betrieb während eines Sicherheitsvorfalls zu gewährleisten ist ein entsprechender Notfallplan erforderlich. Dieser ist Teil des \emph{Betriebskontinuitäts- und Krisenmanagement}. Hiermit soll nicht nur die Aufrechterhaltung sondern auch die Wiederherstellung des Betriebs sichergestellt werden. Konkret umfasst der Notfallplan auf Basis der Risikobewertung zunächst einmal den Zweck, Umfang und die Zielgruppe des Plans. Um im Notfall klar definierte Rollen und Verantwortlichkeit definiert zu wissen ist dies ebenfalls Bestandteil des Notfallplan. Neben der Reihenfolge der Wiederherstellung der Betriebsabläufe sind auch die hierfür erforderlichen Ressourcen, einschließlich Sicherungen und Redundanzen, Bestandteil des Plans. Damit Kommunikationskanäle klar definiert sind werden alle wichtige Kontaktangaben und (interne und externe) Kommunikationskanäle ebenfalls Dokumentiert. Zusätzlich sind die Informationen unter welchen Bedingungen der Notfallplan aktiviert- und wieder deaktiviert werden kann essenziell für dessen Anwendung. Diese Definitionen Bilden den Notfallplan für die jeweilige Einrichtung. Eine regelmäßige Erprobung ist ebenfalls Bestandteil des \emph{Betriebskontinuitäts- und Krisenmanagement}. Zur Gewährleistung einer Entsprechender Resilienz sind auf Basis der Auswirkung einer Störung auf die Betriebsabläufe die Kontinuitätsanforderungen für die Netz- und Informationssysteme abzuleiten. Aus den Kontinuitätsanforderungen kann man wiederum das anfertigen von Sicherungskopien Ableiten. Auch das erstellen von Sicherungspläne auf Basis der Risikobewertung und des Betriebskonti­nuitätsplans sind Teil der Maßnahmen des \emph{Betriebskontinuitäts- und Krisenmanagement}. In jedem fall enthält ein Sicherungspläne die Wiederherstellunggszeit sowie das vorgehen bei Wiederherstellung, Aufbewarhungsfristen, Integritätsprüfungen, phyische und logische Zugangskontrolle entsprechender Klassifizierunggsstufen der Anlagen und Werte sowie den Speicherort. Der Speicherort ist entsprechend so zu wählen das bei einem Ausfall der zu sichernden Systeme die Sicherungskopien nicht betroffen sein können. Ebenfalls ist bei den Aufbewarhungsfristen Regulatorische Anforderungen zu beachten, wie diese beispielsweise aus der \gls{ao} hervorgehen. Unabhängig der Kontinuitätsanforderungen sind der Risikobewertung entsprechende Netz- und Informationssysteme, Kommunikationskanäle sowie Anlagen und Werte redundant oder teilweise redundant auszulegen. Auch Personal muss hinsichtlich der Verantwortlichkeit, Weisungsbefugnis und Kompetenz entsprechend resilient sein. Für schwerwiegende Ereignisse sind Verfahren für das Krisenmanagement zu etablieren. Diese umfassen mindestens die Rollen und Verantwortlichkeiten des Personals und der Anbieter sowie die geeignete Kommunikationsmittel zwischen den betreffenden Einrichtungen  und den jeweils zuständigen Behörden. Ebenfalls sind die Maßnahmen zur Gewährleistung der Aufrechterhaltung der Sicherheit von Netz- und Informationssystemen in Krisensituationen fundamentaler Bestandteil des Krisenmanagement. Einhergehend mit dem \emph{Konzept für das Risikomanagement} sind die Verfahren zur Verwaltung von Informationen über Sicherheitsvorfälle, Schwachstellen, Bedrohungen oder mögliche Risikominderungsmaßnahmen, welche ebenfalls dem \emph{Betriebskontinuitäts- und Krisenmanagement} zuzuordnen sind.\footnote{
                \footcite[Vgl. Nummer 4,][, Anhang]{EU2024-2690}
                \footcite[Vgl. Artikel 22, Absatz 1,][]{EU2022-2555}
                \footcite[Vgl. §30 Absatz 2, Nummer 3,][]{NIS2UmsuCG}
            }

            % Sicherheit der Lieferkette
            \subsubsection{Sicherheit der Lieferkette}
            Ein wesentlicher Bestandteil externer Einflüsse sind die Lieferketten. Dementsprechend ist es erforderlich ein dienliches Konzept für \emph{Sicherheit der Lieferkette} Aufzuweisen. Zusätzlich ist es essenziell die eigene Rolle und die Beziehung zu direkten Anbietern und Diensteanbietern zu evaluieren und sich dieser Bewusst zu werden.Kernforderung an das Konzept für \emph{Sicherheit der Lieferkette}  ist die Definition von Kriterien für die Auswahl von Anbietern und Diensteanbietern sowie die Auftragsvergabe. Hierbei ist es wichtig das die Lieferanten Cybersicherheitsverfahren einschließlich der Sicherheit ihrer Entwicklungsprozessen etabliert haben. Die allgemeine Qualität und Resilienz der angebotenen und entwickelten IKT-Produkte und -Dienste sowie Risikomanage­mentmaßnahmen im Bereich der Cybersicherheit sind ebenfalls ein wichtiger Bestandteil des Konzeptes. Zur Vermeidung von Abhängigkeiten soll zusätzlich Wert auf die Fähigkeit die Versorgungsquellen zu diversifizieren und Abhängigkeit von bestimmten Anbietern gelegt werden. Unter Rücksichtnahme von Risikobewertungen zur \emph{Sicherheit der Lieferkette}  aus der Zusammenarbeit von Kooperationsgruppe mit der Europäischen Kommission und der \gls{enisa} bildet dies die Anforderungen an das Konzept für \emph{Sicherheit der Lieferkette} . Für eine fortwährende Geschäftsbeziehung sehen die erforderlichen Maßnahmen vor eine auf Basis der Risikobewertung und im Rahmen der Leistungsverein­barungen mit Anbietern und Diensteanbietern in Form entsprechender Verträgen diverse Aspekte zu Regeln. Bestandteile davon sind die Cybersicherheitsanforderungen an die Anbieter oder Diensteanbieter sowie die Anforderungen an die Sicherheitsmaßnahmen beim Erwerb von IKT-Diensten oder IKT-Produkten. Ebenfalls unterliegen die Mitarbeitenden der Anbieter und Diensteanbieter Sensibilisierungs-, Qualifikations- und Ausbildungsanforderungen. Die Anbieter und Diensteanbieter verpflichten sich ebenfalls Sicherheitsvorfälle, die ein Risiko für die Sicherheit der Netz- und Informationssysteme darstellen, unverzüglich zu melden sowie Schwachstellen zu Beheben, welche ein Risiko für die Sicherheit der Netz- und Informationssysteme darstellen. Für die Unterauftragsvergabe durch den Anbieter und Diensteanbieter entstehen Cybersicherheitsanforderungen an Unterauftragnehmer, welche mit den Cybersicherheitsanforderungen an den Anbieter und Diensteanbieter vergleichbar sein können. Darüberhinaus werden sämtliche andere Pflichten der Anbieter und Diensteanbieter im Rahmen der Verträge zur Leistungsverein­barungen definiert. Diese zwei wesentlichen Komponenten sollen unmittelbar die \emph{Sicherheit der Lieferkette} gewährleisten. Geschäftsbeziehungen sind dynamisch und können sich wandeln. Daher müssen die Maßnahmen für die \emph{Sicherheit der Lieferkette} überwacht werden. Hierzu zählen, soweit Anwendbar, die durch den Anbietern und Diensteanbietern zu erstellenden Berichte über die Umsetzung der Leistungsvereinbarungen. Ebenfalls sollen Sicherheitsvorfälle im Zusammenhang mit IKT-Produkten und -Diensten von Anbietern und Diensteanbietern überprüft und Risiken, die sich aus Änderungen im Zusammenhang mit IKT-Produkten und -Diensten von Anbietern und Diensteanbietern ergeben, analysieren und falls erforderlich Maßnahmen ergreifen werden. Zusätzlich soll überprüft werden ob eine außerplanmäßiger Überprüfungen notwendig ist und sofern dieser Fall eintritt ist die Prüfung nachvollziehbar zu Dokumentieren. Für eine Übersicht ist ein Verzeichnis der direkten Anbieter und Diensteanbieter mit Kontaktstellen und eine Liste mit bereitgestellten IKT-Produkte, -Dienste und -Prozesse zu erstellen und pflegen.\footnote{
                \footcite[Vgl. Anhang, Nummer 5,][]{EU2024-2690}
                \footcite[Geschäftsbeziehungen sind dynamisch][]{MISSING}
                \footcite[Vgl. §30 Absatz 2, Nummer 4,][]{NIS2UmsuCG} 
            }

            % Nummer 6: ITAM / CMDB, Netzwersicherheit / Segmentierung, Change Management, MDM, Patch-Management, Vulnerability Management, Mail / Kommunikationssicherheit
            \subsubsection{Sicherheitsmaßnahmen bei Erwerb, Entwicklung und Wartung von Netz- und Informationssystemen}
            Kommen IKT-Produkten und -Diensten von Anbietern und Diensteanbietern oder durch Eigenentwicklung in den Einsatz sind Vor- aber auch Nachgelagert \emph{Sicherheitsmaßnahmen bei Erwerb, Entwicklung und Wartung von Netz- und Informationssystemen} erforderlich. Allgemein sind Verfahren für die Sicherheit unverzichtbare Netz- und Informationssysteme erforderlich. Somit soll es Sicherheitsanforderungen für die IKT-Dienste und -Produkte. Die Allgemeinen Sicherheitsanforderungen sollen durch Anforderungen an Sicherheitsaktualisierungen während der gesamten Lebensdauer der IKT-Dienste und -Produkte oder dem Ersatz nach Ablauf des Unterstützungszeitraums umfassen. Zusätzlich sind beschreibeungen zu Hard- und Software wie zu den umgesetzten Cybersicherheitsfunktionen erforderlich. Ebenfalls muss eine Zusicherung exisitieren, dass die definierten Sicherheitsanforderungen von dem IKT-Dienste oder -Produkte erfüllt werden. Dies muss demzufolge durch den Erichter des IKT-Dienste oder -Produkte erfolgen. Darüberhinaus müssen Methoden zur Validierung der Sicherheitsanforderungen existieren und dessen Ergebnisse Dokumentiert werden. Dies validiert oder falsifiziert die Zusicherung über die Einhaltung der Sicherheitsanforderungen. Unabhängig der Kritikalität eines Netz- und Informationssysteme sind sowohl für die beauftragte als auch eigenentwickelte Errichtung von Netz- und Informationssystems Verfahren zur Entwicklung, welches die Spezifikation, Konzeption, Entwicklung, Umsetzung und Tests umfassen, festzulegen. Hierfür muss eine Analyse der Sicherheitsanforderungen in der Spezifikations- und Entwurfsphase jedes Entwicklungs- oder Beschaffungsvorhabens einfließen. Allgemein müssen bestimmte Grundsätze für den Aufbau sicherer Systeme und für ein sicheres Programmieren aufgestellt und eingehalten werden. Auch an die Entwicklungsumgebungen müssen Sicherheitsanforderungen sowie Sicherheitstestverfahren im Entwicklungszyklus gestellt und angewendet werden. Sämtliche Daten für Validierungen der Netz- und Informationssystems in entsprechenden Tests sind Basierend den Anforderungen auszuwählen, zu schützen und zu verwalten wie zu bereinigen und anonymisieren. Für den Betrieb eines Netz- und Informationssystems sind den Anforderungen entsprechende Konfigurationen erforderlich. Maßnahmen um Konfigurationen von Hardware, Software, Diensten und Netzen, festzulegen, zu dokumentieren, umzusetzen und zu überwachen sind demnach zu treffen. Ein wichtiger Bestandteil hiervon sind die Verfahren und Instrumente zur Durchsetzung für neu installierte Systeme und Systeme über ihre gesamte Lebensdauer. Durch sich ändernde Anforderungen kann es erforderlich Netz- und Informationssystemen anzupassen. Daher sind Verfahren für das Änderungsmanagement, mit dem Ziel Änderungen an Netz- und Informationssystemen zu kontrollieren, einzuführen. Diese müssen im Einklang mit den Allgemeinen Änderungsmanagement der Einrichtung stehen. Dieses Änderungsmanagement umfasst Verfahren für Freigaben, Änderungen und Notfalländerungen an in Betrieb befindlicher Software und Hardware sowie ebenfalls die Änderungen der Konfiguration. Ein Abweichendes vorgehen, beispielsweise wegen einer Krisensituationen, muss inklusive Begründung Dokumentiert werden. Um regelmäßig den aktuellen Stand der Sicherheit einschätzen zu können sind, sowohl für Wartung als auch Implementation von Netz- und Informationssystemen, Konzepte und Verfahren für Sicherheitsprüfungen festzulegen. Hierbei ist zunächst die Notwendigkeit aber auch der Umfang und die Häufigkeit und die Art auf Basis der Risikobewertung festzulegen. Damit die Sicherheitsprüfungen repetitive durchgeführt werden können muss dies nach definierten Prüfmethoden erfolgen. Sämtliche Aspekte der Sicherheitsprüfung, demnach Art, Umfang, Zeitraum und die Ergebnisse, sind zu Dokumentieren. Ergibt die Sicherheitsprüfung eine Feststellung sind diese inklusive dessen Kritikalität und den Risikominderungsmaßnahmen zu Dokumentieren. Bei kritischen Feststellungen werden Risikominderungsmaßnahmen unmittelbar angewendet. Als Erweiterung der Sicherheitsprüfungen kann man die Verfahren und den Umgang hinsichtlich technische Schwachstellen in den Netz- und Informationssystemen sehen. Diese zu ergreifende Maßnahme soll Verfahren zur Informationen über Schwachstellen über geeignete Kanäle verfolgen, Schwachstellen-Scans durchführen und sicherstellen, dass die Behandlung von Schwachstellen im Einklang mit Änderungsmanagement, Sicherheitspatch-Management, Risikomanagement, Management von Sicherheitsvorfällen vereinbar ist. Sofern die Auswirkung einer Schachstelle es erforderlich machen werden der Plan zur Minderung oder warum keine Abhilfemaßnahmen getroffen wurden Dokumentiert und begründet. Ein Bestandteil der Behandlung von Schwachstellen kann das Sicherheitspatch-Management. Nicht nur für die Behandlung sondern auch für die Allgemeine Sicherheit der Netz- und Informationssysteme sind Verfahren hinsichtlich Sicherheitspatches, im Einklang mit Änderungs-,  Schwachstellen-  und Risikomanagementverfahren, zu definieren. Diese Umfassen Sicherheitspatches mit angemessenen Frist nach ihrer Verfügbarmachung anzuwenden, dass diese vor Produktivem Einsatz getestet werden und nur aus vertrauenswürdigen Quellen stammen. Die Integrität der Sicherheitspatches ist ebefnalls anzuwenden sowie Maßnahmen zu ergreifen und Restrisiken zu akzeptieren wenn ein Patch nicht verfügbar oder anwendbar ist. Ist ein Sicherheitspatch verfügbar und kann nicht angewendet werden, da die Nachteile die Vorteile für die Cybersicherheit überwiegen, muss dies mit entsprechender Begründung Dokumentiert werden. Die Umfangreichen Anforderungen der \emph{Sicherheitsmaßnahmen bei Erwerb, Entwicklung und Wartung von Netz- und Informationssystemen} beinhalten ebenfalls die Notwendigkeit geeigneter Maßnahmen zum Schutz der Netz- und Informationssysteme vor Cyberbedrohungen. Bestandteil hiervon sind:\footnote{
                \footcite[Vgl. Nummer 6 bis 6.6.2 \& 6.10,][, Anhang]{EU2024-2690}
                \footcite[Vgl. §3, Absatz 1, Nummer 2,][]{NIS2UmsuCG} % Vulnerability Management
                \footcite[Vgl. §13, Absatz 1,][]{NIS2UmsuCG} % Vulnerability Management
                \footcite[Vgl. §30 Absatz 2, Nummer 5,][]{NIS2UmsuCG} % Erwerb, Entwicklung und Wartung 
            }
            \begin{itemize}
                \item Dokumentation der Architektur des Netzes
                \item Kontrollen festlegen und durchführen um die internen Systeme vor unbefugtem Zugriff zu schützen
                \item Netzkommunikation verhindert wenn diese für den Betrieb nicht erforderlich ist
                \item Kontrolle des Fernzugriff auf Netz- und Informationssysteme festlegen (ebenfalls für Dienstleister)
                \item Nicht benötigte Verbindungen und Dienste verbieten beziehungsweise deaktivieren
                \item Im angemessenen Rahmen den Zugang zu Netz- und Informationssystemen nur mit genehmigten Geräte erlauben
                \item Verbindungen von externen (beispielsweise Dienstleister) nur nach einem Genehmigungsverfahren und nur für einen definierten Zeitraum gewährleisten
                \item Kommunikation zwischen Systemen nur über vertrauenswürdige Kanäle (logische, kryptografische oder physikalische Trennung von anderen Kommunikationskanälen) ermöglcihen
                \item Durchführungsplan für das sichere, angemessene und schrittweise Einführung der neuesten Generation von Kommunikationsprotokollen
                \item Durchführungsplan für die Einführung international vereinbarter und interoperabler moderner E-Mail-Kommunikationsnormen
                \item Verfahren für die Sicherheit des \gls{dns} sowie für die Sicherheit und Hygiene des Internet-Routings bei für das Netz bestimmten Datenverkehr
            \end{itemize} Daraus kann man weiterführend die Anforderung der Segmentierung von Systeme in Netze oder Zonen, sowie die eigenen von den System Dritter zu trennen, ableiten. Diese Maßnahmen umfassen die funktionale, logische und physische Beziehung zwischen den eigenen und der Systeme dritter. Der Zugang zu einem Netz oder einer Zone auf Grundlage einer Bewertung der Sicherheitsanforderung gewährt. Systeme, die für den Betrieb oder für die Sicherheit unverzichtbar sind in gesondert gesicherten Zonen unterzubringen. Für sowohl ein- als auch ausgehenden Datenverkehr ist eine demilitarisierte Zone zu errichten. Allgemein ist die Kommunikation zwischen und innerhalb der Zonen auf das Notwendige zu beschränken. Ebenfalls ist eine strikt Trennung von operativen zu Netzen, welchem dem Zweck der Verwaltung dienen, notwendig. Netz- und Informationssysteme, genutzt als Produktionssysteme oder Systemen genutzt für die Entwicklung, Tests und Sicherung müssen ebenfalls voneinander Segmentiert werden. Software, welche auf den Netz- und Informationssysteme betrieben wird, erfordert ebenfalls gesonderten Schutzt. Withcit ist das die Netz- und Informationssysteme vor Schad- und nicht genehmigter Software geschützt werden. Hierbei sollen Maßnahmen, welche sowohl die Erkennung als auch die Vermeidung der Verwendung von Schad- und nicht genehmigter Software definiert und umgesetzt werden. Sofern angemessen ist eine Erkennungs- und Reaktions­software auf den Netz- und Informationssysteme zu betreiben.\footnote{
                \footcite[Vgl. Nummer 6.7, 6.8 \& 6.9,][, Anhang]{EU2024-2690}
                \footcite[Vgl. §31 Absatz 2,][]{NIS2UmsuCG} % automatische Angrifsserkennung anhand geeignete Parameter und Merkmale
                \footcite[Vgl. §53 Absatz 1,][]{NIS2UmsuCG} % Konformitätserklärung zu IKT Diensten und Produkten
            }
            
            % Nummer 11: Zugriffskontrolle | PIAM, Audit Logging, Jumphost, Gruppen und Berechtigungen (ACL/ACE), Passwort Management, Zentrales Identity Management
            % \emph{Zugriffskontrolle}
            \subsubsection{Zugriffskontrolle}
            Die Mensch-Maschine- oder Maschine-Maschine-Interaktion muss Maßnahmen der \emph{Zugriffskontrolle} unterliegen. Hierfür wird ein Konzepte für die logische und physische Kontrolle des Zugangs zu Netz- und Informationssystemen auf Basis von geschäftlichen Anforderungen sowie Anforderungen an die Sicherheit von Netz- und Informationssystemen gefordert. Diese Konzept betrifft den Bereich des Zugang zu Netz- und Informationssystemen und bezieht sich auf das Personal, die Besuchern und externen Einrichtungen. Diese müssen in angemessen weise authentifiziert und durch einschlägige Personen genehmigt werden bevor der Zugang zu den Netz- und Informationssystemen gewährt wird. Umfang und Dauer sind soweit möglich zu Beschränken. Allgemein sind die Zugangs- und Zugriffsrechte für die Netz- und Informationssysteme zu Dokumentiert sowie in einem Register zu führen und die Grundsätze in From von \emph{Need-to-know}\footnote{Informationen sind nur für Subjekte zugänglich, welche die Informationen benötigen.}, \emph{Need-to-use}\footnote{Informationen sind nur für Subjekte zugänglich, welche die Informationen verwenden müssen.} und der \emph{Aufgabentrennung}\footnote{Informationen werden durch verschiedene Identitäten einem Benutzer Kontextabhängig zugänglich gemacht} umzusetzen. Bei Beendigung oder Änderung des Beschäftigungsverhältnisses müssen diese entsprechend angepasst werden. Anpassungen und andere Verwaltungstätigkeiten hinsichtlich der Zugangs- und Zugriffsrechte sind zu Protokollieren. Nach dem Grundsatz der \emph{Aufgabentrennung} sind dedizierte Systemverwaltungskonten für Systemverwaltungsvorgänge (Installation, Konfiguration, Verwaltung oder Wartung) und privilegierte Konten mit entsprechenden Grundsätze der Verwaltung erforderlich. Die Nutzung ist mittels starker Verfahren zur Identifizierung und Authentifizierung abzusichern sowie die Verwendung vorher zu genehmigen. Ein Systemverwaltungskonten darf ausschließlich mit Systemverwaltungssystemen verwendet werden. Ein Systemverwaltungssystemen, welches zur Verwaltung von Systemen verwendet wird, ist ausschließlich für diesen Zweck vorgesehen und von Anwendungssoftware zu trennen. Diese sind durch Authentifizierung und Verschlüsselung geschützt. Ein Wichtiger Aspekt von Zugangs- und Zugriffsrechte sind die Identitäten, mit welchen diese in Verbindung stehen. Die Identitäten eines Netz- und Informationssysteme und der Nutzer wird über dessen gesamten Lebenszyklus verwaltet. Hierfür gibt es eindeutige Kennungen, 1-zu-1 Beziehung von Benutzerkennung zu Nutzer (logische zu natürlicher Person), die Überwachung der Kennung und die Protokollierung der Verwaltung der Kennungen. Ist eine direkte Beziehung von Nutzer zu Kennung aus geschäftlichen oder operativen Gründen nicht möglich muss dies vorher genehmigt werden um zulässig verwendbar zu sein. Das Verfahren zur Genehmigung von geteilten Kennungen ist zu Dokumentieren sowie diese Kennungen im Rahmen des Risikomanagement im Bereich der Cybersicherheit gesondert zu betrachtet. Die von den Kennungen verwendeten Authentifizierungsverfahren und -techniken müssen für der Klassifizierung der Anlage bzw. des Werts angemessen sein. Auch das Personal muss angemessenen mit Authentifizierungsinformationen umgehen, was die Verwaltung geheimer Authentifizierungsinformationen und die Zuweisung an Nutzer unter Gewährleistung der Vertraulichkeit der Informationen beinhaltet.Die Änderung der Authentifizierungsdaten zu Beginn, in festgelegten Zeitabständen und bei beeinträchtigten der Authentifizierungsdaten ist erforderlich. Ebenfalls ist das Sperren der Kennungen bei gewisser Anzahl an Fehlgeschlagener Anmeldeversuche sowie das beenden inaktiver Sitzungen erforderlich. Mit gesonderte Authentifizierungsdaten sind Systemverwaltungskonten und privilegierte Konten zu schützen und nach Möglichkeit und wenn es angemessen ist Multifaktor-Authentifizierung zu verwenden. \footnote{
                \footcite[Vgl. Nummer 11,][, Anhang]{EU2024-2690}
                \footcite[Vgl. §30 Absatz 2, Nummer 9 und 10,][]{NIS2UmsuCG} % Erwerb, Entwicklung und Wartung 
            }

            % Nummer 9: Kryptografie Schlüsselmanagement (Key Management), PKI
            % \emph{Kryptografie} 
            \subsubsection{Kryptografie}
            Authentifizierung steht immer in einem Kontext mit \emph{Kryptografie}. Gemäß der Risikobewertung und der Anlagen- und Werteklassifizierung ist mittels eines Konzept und Verfahren die angemessene und wirksame Nutzung von Kryptografie zur Gewährleistung der Vertraulichkeit, Authentizität und Integrität der Daten zu erstellen und anzuwenden. Das Konzept umfasst Art, Stärke und Qualität der kryptografischen Maßnahmen im Angemessenen Umfang hinsichtlich der Einstufung der Anlagen und Werte nach dem Krypto-Agilitätsansatz. Die Protokolle oder Protokollfamilien, kryptografische Algorithmen, Kryptierungsstärke, kryptografische Lösungen und Nutzungsverfahren müssen vor Verwendung genehmigt werden. Zusätzlich Umfasst das Konzept das Schlüsselmanagement inklusive der Methoden für:
            \begin{itemize}
                \item Generierung von Schlüssel für kryptografische Systeme und Anwendungen
                \item Ausstellen von \gls{pki}-Zertifikaten
                \item Verteilung und Aktivierung von Schlüsseln
                \item Speicherung der Schlüssel sowie erhalte für Autorisierte Personen
                \item Änderung oder Aktualisierung von Schlüsseln
                \item Prozess der Änderung oder Aktualisierung von Schlüsseln
                \item Umgang mit beeinträchtigten Schlüsseln
                \item Wiederruf von Schlüsseln inklusive dem zurückziehen und deaktivieren
                \item Sicherung und Archivierung von Schlüsseln
                \item Vernichtung von Schlüsseln
                \item Protokollierung und Prüfung von Verwaltungstätigkeiten im Zusammenhang mit Schlüsseln
                \item Aktivierungs- und Deaktivierungsfristen für die Beschränkung des Nutzungszeitraumes gemäß eigen erstellter Vorgaben
            \end{itemize}Sämtliche kryptografischen Maßnahmen beziehen sich auf sowohl gespeichert oder gerade übermittelte Daten.\footnote{
                \footcite[Vgl. Nummer 9,][, Anhang]{EU2024-2690}
                \footcite[Vgl. §30 Absatz 2, Nummer 8,][]{NIS2UmsuCG} % Erwerb, Entwicklung und Wartung 
            }
            
            % Nummer 13: Sicherheit des Umfelds und physische Sicherheit 
            % \emph{Sicherheit des Umfelds und physische Sicherheit}
            \subsubsection{Sicherheit des Umfelds und physische Sicherheit}
            Neben all den logischen Komponenten der Netz- und Informationssystemen ist auch die \emph{Sicherheit des Umfelds und physische Sicherheit} ein wichtiger Aspekt des \gls{nis2umsucg}. Diese soll Verluste, Schäden oder Beeinträchtigungen von Netz- und Informationssystemen oder Unterbrechungen ihres Betriebs durch Störungen in den Versorgungsleistungen verhindern. Hierfür soll die Einrichtung Maßnahmen ergreifen, welche vor Stromausfall und anderen Störungen schützen sowie Telekommunikationsdienste vor Abhörung und Beschädigung bewahren. Anhand bestimmter Mindest- und Höchstkontrollwerte ergibt sich ob ein Ereignisse die Versorgungsleistungen von Strom und Telekommunikationsdienste betreffend den zuständigen in- oder externen Stellen zu melden ist. Nach Ermessen und sofern angemessen sind Verträge für Notversorgung der Versorgungsleistungen abzuschließen. Die Wirksamkeit, Überwachung, Wartung und Erprobung der Netz- und Informationssysteme die sich auf von der Einrichtung angebotene dienste beziehen, insbesondere Strom, Temperatur- und Feuchtigkeitsregelung, Telekommunikation und Internetverbindung, ist kontinuierlich zu Erproben. Allgemein müssen  Physische Bedrohungen und Bedrohungen des Umfelds verhindert oder verringert werden indem Schutzmaßnahmen konzipiert und umgesetzt Mindest- und Höchstkontrollwerte bestimmt werden. Ergänzend sind Umgebungsparameter zu überwachen und den zuständigen internen oder externen stellen zu melden, wenn die Mindest- oder Höchstkontrollwerteliegen unter- oder überschreiten werden. Mittels Sicherheitsperimeter soll Bereiche der Netz- und Informationssysteme und andere zugehörige Anlagen, gesichert durch Zutrittskontrollen und Zugangspunkte, geschützt werden. Diese werden auf Basis Risikobewertung festgelegt. Ebenfalls gilt es Maßnamen zur physischen Sicherheit von Büros, Räumen und Betriebsstätten zu konzipieren und den unbefugten physikalischen Zugriff zu Überwachen.\footnote{
                \footcite[Vgl. Nummer 13][, Anhang]{EU2024-2690}
            }

            % Nummer 8: Grundlegende Verfahren im Bereich der Cyberhygiene und Schulungen im Bereich der Cybersicherheit 
            %\emph{Grundlegende Verfahren im Bereich der Cyberhygiene und Schulungen im Bereich der Cybersicherheit}
            \subsubsection{Grundlegende Verfahren im Bereich der Cyberhygiene und Schulungen im Bereich der Cybersicherheit}
            Im Rahmen der \emph{Durchführungsverordnung (EU) 2024/2690} sind Personen an diversen stellen betroffen. \emph{Grundlegende Verfahren im Bereich der Cyberhygiene und Schulungen im Bereich der Cybersicherheit} für Mitarbeiter, Leitungsorgane und direkte Anbieter und Dienstanbieter ist ein weiter wichtiger Bestandteil der Anforderungen an die \emph{Durchführungsverordnung (EU) 2024/2690}. Mitarbeiter, Leitungsorgane und direkte Anbieter und Dienstanbieter müssen sich der Risiken bewusst und über die Bedeutung der Cybersicherheit informiert werden. Mitarbeiter, Leitungsorgane und direkte Anbieter und Dienstanbieter müssen sich der Risiken bewusst und über die Bedeutung der Cybersicherheit informiert werden. Um dies praktisch umzusetzen wird den betroffenen Personen ein repetitive Sensibilisierungsprogramm angeboten das:
            \begin{itemize}
                \item Im Einklang mit dem Konzept und den Verfahren für die Sicherheit von Netz- und Informationssystemen steht
                \item Einschlägige Cyberbedrohungen beinhaltet
                \item Kontaktstellen und Ressourcen für zusätzliche Informationen beinhaltet
                \item Kontaktstellen für Beratung zu Cybersicherheitsfragen beinhaltet
                \item Verfahren im Bereich der Cyberhygiene für Nutzer beinhaltet
            \end{itemize} Zusätzlich müssen sicherheitsrelevanten Rollen die zusätzlich Sensibilisierung erfordern identifiziert werden. Ergänzend dem Sensibilisierungsprogramm ist ein Schulungsprogramm einzuführen. Der Schulungsbedarf für bestimmte Rollen und Positionen auf Basis von Kriterien er ermittelt und beinhaltet zumindest Anweisungen für die sichere Konfiguration und den sicheren Betrieb der Netz- und Informationssysteme, einschließlich mobiler Geräte sowie die Unterrichtung über bekannte Cyberbedrohungen und Schulung in Bezug auf das Verhalten bei sicherheitsrelevanten Ereignissen. Schulungsprogramm sowie der Schulungsbedarf werden regelmäßige aktualisiert.\footnote{
                \footcite[Vgl. Nummer 8,][, Anhang]{EU2024-2690}
                \footcite[Vgl. §30 Absatz 2, Nummer 7,][]{NIS2UmsuCG} % 
            }

            % Nummer 10: Sicherheit des Personals
            %\emph{Sicherheit des Personals}
            \subsubsection{Sicherheit des Personals}
            \emph{Grundlegende Verfahren im Bereich der Cyberhygiene und Schulungen im Bereich der Cybersicherheit} stehen im Einklang mit der \emph{Sicherheit des Personals}. Für die \emph{Sicherheit des Personals} ist es wichtig das sowohl Mitarbeiter als auch direkten Anbieter und Diensteanbieter ihre Verantwortlichkeiten im Bereich der Sicherheit verstehen und sich zu ihrer Einhaltung verpflichten. Zu Gewährleistung gibt es Mechanismen die sicherstellen das die betroffenen Personen die Standardverfahren der Cyberhygiene verstehen und befolgen. Privilegierte oder administrative Rollen sich ihrer Verantwortlichkeiten und Weisungsbefugnisse bewusst und handeln entsprechend. Im Bezug auf die Sicherheit von Netz- und Informationssystemen verstehen und handeln die Leitungsorgane entsprechend ihre Rolle, Verantwortlichkeit und Weisungsbefugnisse. Bei der Personalakquise ist zu Gewährleisten das die Person für die Position qualifiziert ist. Dies kann mittels diverser verfahren, wie beispielsweise die Überprüfung der   Referenzen, Überprüfungsverfahren, Validierung von Zeugnissen oder schriftliche Prüfungen, erfolgen. Allgemein sind Kriterien Festzulegen, anhand welcher Rollen, Verantwortlichkeiten und Weisungsbefugnisse ermittelt werden, bei welchen es eine Zuverlässigkeitsüberprüfungen bedarf. Diese Rollen, Verantwortlichkeiten und Weisungsbefugnisse dürfen nur von Personen nach einer Zuverlässigkeitsüberprüfungen wahrgenommen werden. Unabhängig des Beschäftigungsverhältnisses, also auch nach Beendigung dieses, ist sicherzustellen das die Sicherheit von Netz- und Informationssystemen gewahrt ist. Generell sind Disziplinarverfahren für den Umgang mit Verstößen zu implementieren. Dies setzt die \emph{Konzept für die Sicherheit von Netz- und Informationssystemen} in den Fokus. Die Disziplinarverfahren werden bekannt gemacht und aufrechterhalten.\footnote{
                \footcite[Vgl. Nummer 8,][, Anhang]{EU2024-2690}
                \footcite[Vgl. §30 Absatz 2, Nummer 7 und 9,][]{NIS2UmsuCG} % 
            }\medbreak

            In Summe beschreiben die zuvor genannten Konzepte, Verfahren, Maßnahmen und Anforderungen den im Kontext von \gls{nis2umsucg} und damit auch der \emph{Durchführungsverordnung (EU) 2024/2690} umzusetzenden Rahmen. Für betreiber Kritische Anlagen gilt ein Besonderes Maß der Nachweispflicht. Diese müssen, beispeilsweise durch Audits oder Zertifizierungen, die Umsetzung aller Konzepte, Verfahren, Maßnahmen und Anforderungen nachweisen.\footnote{\footcite[Vgl. §39,][]{NIS2UmsuCG}} Konkrete Handlungsempfehlungen sind in Abschnitt \ref{sec:ParktischeMaßnahmen} beschrieben und ergänzen diese theoretische Grundlage.
            
        \subsection{Registrierungs- und Meldewesen}
        Betroffene Einrichtungen müssen sich bei einer vom \gls{bsi} und \gls{bbk} gemeinsam eingerichteten Registrierungsmöglichkeit registrieren. Dies muss binnen drei Monaten, nachdem diese erstmals oder erneut als solche Einrichtung gelten, erfolgen\footnote{\footcite[Vgl. §33, Absatz 1,][]{NIS2UmsuCG}}. Prinzipiell sind die Einzelheiten des Registrierungsverfahrens durch das \gls{bsi} im einvernehmen mit dem \gls{bbk} festzulegen und über dessen Internetseite zu veröffentlichen\footnote{\footcite[Vgl. §33, Absatz 6,][]{NIS2UmsuCG}}. Im Rahmen des Registrierungsverfahrens sind diverse angaben zu Tätigen. Neben dem Namen, der Rechtsform und, sofern zutreffend, der Handelsregisternummer sind ebenfalls Anschrift sowie aktuelle Kontaktdaten inklusive der E-Mail-Adresse und öffentlichen IP-Adressbereiche anzugeben. Zusätzlich dazu muss die Brache oder, sofern einschlägig, der Sektor und eine Auflistung der EU Mitgliedstaaten, in welche die entsprechenden Dienste erbracht werden, erfolgen. Die Angabe der für die Einrichtung zuständige Aufsichtsbehörden des Bundes sowie der Länder ist ebenfalls notwendig\footnote{\footcite[Vgl. §33, Absatz 1,][]{NIS2UmsuCG}}. Betreiber kritischer Anlagen müssen zusätzlich ihre Anlagenkategorie nach BSI-KritisV sowie die ermittelten Versorgungskennzahlen gemäß der zu bestimmenden Rechtsverordnung des NIS2UmsuCG mitteilen. Der Standort sowie eine Kontaktstelle, welche stets verfügbar ist, muss ebenfalls angegeben werden. Darüberhinaus ist die kritische Dienstleistung sowie die öffentlichen IP-Adressbereiche der damit verbundenen Anlagen Bestandteil der Registrierung eines Betreibers kritischer Anlagen\footnote{\footcite[Vgl. §33, Absatz 1,][]{NIS2UmsuCG}}. Das \gls{bsi} kann in einvernehmen mit der jeweilig zuständigen Aufsichtsbehörde besonders wichtigen und wichtigen Einrichtungen, sowie Domain-Name-Registry-Diensteanbieter, eigenständig Registrieren, sofern diese ihrer Pflicht nicht nachkommen\footnote{\footcite[Vgl. §33, Absatz 3,][]{NIS2UmsuCG}} Alternativ kann das \gls{bsi} die aus ihrer Sicht notwendigen Aufzeichnungen, Schriftstücke und sonstigen Unterlagen Anfordern um den Umstand der Registrierungspflicht zu bewerten. Dies unter der Voraussetzung das aus sich der Einrichtung Geheimschutzinteressen oder überwiegende Sicherheitsinteressen vorliegt\footnote{\footcite[Vgl. §33, Absatz 4,][]{NIS2UmsuCG}}. Sollten sich Änderungen an den zu Übermittelnden Informationen ergeben müssen diese binnen zwei Wochen mitgeteilt werden\footnote{\footcite[Vgl. §33, Absatz 5,][]{NIS2UmsuCG}}. Bei Einrichtungen der Art DNS-Diensteanbieter, Top Level Domain Name Registries,Domain-Name-Registry-Dienstleister, Anbieter von Cloud-Computing-Diensten, Anbieter von Rechenzentrumsdiensten, Betreiber von Content Delivery Networks, Managed Service Provider, Managed Security Service Provider sowie für Anbieter von Online-Marktplätzen, Online-Suchmaschinen sowie Plattformen für Dienste sozialer Netzwerke ist zusätzlich die Hauptniederlassung sowie sonstigen Niederlassungen in der Europäischen Union anzugeben. Bei Änderungen haben diese Einrichtung eine first von drei Monate anstelle der zwei Wochen\footnote{\footcite[Vgl. §34,][]{NIS2UmsuCG}}.\medbreak

        Erhebliche Sicherheitsvorfälle sind einer vom \gls{bsi} und \gls{bbk} eingerichtete gemeinsame Meldestelle anzuzeigen. Die Allgemeine Ausgestaltung des Verfahrens des melden von Sicherheitsvorfällen sowie dessen Inhalt ist durch das \gls{bsi} und \gls{bbk} festzulegen. Im Rahmen der Festlegung der Durchführungsrechtsakte sollen betroffene Betreiber und Wirtschaftsverbände angehört werden\footnote{\footcite[Vgl. §32, Absatz 4,][]{NIS2UmsuCG}}. Betroffen hiervon sind besonders wichtige oder wichtige Einrichtungen. Erlangt eine besonders wichtige oder wichtige Einrichtungen Kenntnis über einen erheblichen Sicherheitsvorfall müssen diese spätestens nach 24 Stunden eine Erstmeldung abgeben. Die Erstmeldung soll widerspiegeln ob der Verdacht besteht, dass der erhebliche Sicherheitsvorfall auf rechtswidrige oder böswillige Handlungen zurückzuführen ist oder grenzüberschreitende Auswirkungen haben könnte\footnote{\footcite[Vgl. §32, Absatz 1,][]{NIS2UmsuCG}}. Handelt es sich um einen Betreiber kritischer Anlagen und der Sicherheitsvorfall könnte sich auf diese Auswirken, oder hat tatsächlich Auswirkungen, sind zusätzlich Angaben zu der Art der Betroffenen kritischen Anlage sowie der kritischen Dienstleistung und der entsprechenden Auswirkung zu tätigen\footnote{\footcite[Vgl. §32, Absatz 3,][]{NIS2UmsuCG}}. Binnen 72 Stunden nach Kenntnisnahme eines erheblichen Sicherheitsvorfalles ist die Erstmeldung durch eine Meldung über den Sicherheitsvorfall abzulösen. Hierbei sollen die information zum einen aktualisiert aber auch verifiziert oder falsifiziert werden. Wichtiger Bestandteil der Meldung ist eine erste Bewertung des erheblichen Sicherheitsvorfalles inklusive Schweregrad und Auswirkung sowie die Indikatoren der Komprimierung. Zusätzlich kann das \gls{bsi} eine Zwischenmeldung einfordern, welche relevante Statusaktualisierungen beinhaltet\footnote{\footcite[Vgl. §32, Absatz 1, Nummer 2 und 3,][]{NIS2UmsuCG}}.  Spätestens einen Monat nach Übermittlung der Meldung wird eine Abschlussmeldung gegeben. Die Abschlussmeldung ist eine ausführliche  Beschreibung des Sicherheitsvorfalls, inklusive seines Schweregrads und dessen Auswirkungen sowie Angaben zur Art und Ursache. Relevant sich auch die getroffenen und fortwährenden Abhilfemaßnahmen. Wenn die Auswirkungen grenzüberschreitend sind diese Auswirkungen ebenfalls zu beschreiben\footnote{\footcite[Vgl. §32, Absatz 4,][]{NIS2UmsuCG}}. Sollte der Sicherheitsvorfalls nach einem Monat weiterhin anhalten ist eine Fortschrittsmeldung zu verrichten und die Abschlussmeldung wird erst nach Abschließender Bearbeitung vorgelegt.\footnote{\footcite[Vgl. §32, Absatz 2,][]{NIS2UmsuCG}} Der erhalt der getätigten Meldungen werden unverzüglich und, sofern möglich, spätestens nach 24 Stunde vom \gls{bsi} Bestätigt. Die betroffene Einrichtung kann ebenfalls Orientierungshilfen, operative Beratung oder technische Unterstützung zu Abhilfemaßnahmen durch das \gls{bsi} erhalten\footnote{\footcite[Vgl. §36, Absatz 1,][]{NIS2UmsuCG}}. Nicht nur eine Meldung an das \gls{bsi} kann bei einem erheblichen Sicherheitsvorfall notwendig sein. Besonders wichtige Einrichtungen und wichtigen Einrichtungen können vom \gls{bsi} aufgefordert werden die Empfänger ihrer Dienste von einem erheblichen Sicherheitsvorfall zu unterrichten, wenn die entsprechenden Dienste beinträchtig sind. Hier kann eine Publikation auf der eigenen Internetseite der Einrichtung hinreichend sein\footnote{\footcite[Vgl. §35, Absatz 1,][]{NIS2UmsuCG}}. Ist die besonders wichtige oder wichtigen Einrichtungen dem Sektor Finanzwesen, Sozialversicherungsträger sowie Grundsicherung für Arbeitssuchende, digitale Infrastruktur, Verwaltung von IKT-Diensten oder Digitale Dienste zuzuordnen sind bereits bei einer erheblichen Cyberbedrohung die Empfänger der entsprechenden Dienste sowie das \gls{bsi} zu Informieren. Hierbei muss die Einrichtung unverzüglich alle Maßnahmen oder Abhilfemaßnahmen mitteilen, welche die Empfänger als Reaktion auf diese Bedrohung ergreifen können. Ebenfalls ist über die erhebliche Cyberbedrohung selbst zu informieren. Diese pflichten gelten ausschließlich wenn die Interessen des Empfänger die der Einrichtung überwiegen\footnote{\footcite[Vgl. §35, Absatz 2,][]{NIS2UmsuCG}}.
        \subsection{Risiken und Sanktionen}
        Für Einrichtungen, welche von \gls{nis2umsucg} und der \emph{Durchführungsverordnung (EU) 2024/2690} betroffen sind bestehen keine unmittelbaren neuen Risiken, Abseits der definierten Strafen. \gls{nis2} verfolgt das Ziel das Cyber­sicherheits­niveau zu erhöhen und somit Cybersicherheitsrisiken zu minimieren. Risiken entstehen für Lieferanten und Hersteller, welche eine Geschäftsbeziehung mit einer betroffenen Einrichtung pflegen. Hier gelten dieselben Hintergründe wie zuvor einleitend in Abschnitt \ref{sec:NIS2UmsuCG} beschrieben.\footnote{
            \footcite[Vgl. §30 Absatz 6,][]{NIS2UmsuCG}
            \footcite[Vgl. §56 Absatz 3,][]{NIS2UmsuCG}
            \footcite[Vgl. Nummer 5.1.1. \& 5.1.2.,][, Anhang]{EU2024-2690}
            \footcite[Vgl. Nummer 5.1.4. \& 5.1.5.,][, Anhang]{EU2024-2690}
            \footcite[Vgl. Nummer 6.1.1. \& 6.1.2.,][, Anhang]{EU2024-2690}
            \footcite[Vgl. Nummer 6.1.1. \& 6.1.2.,][, Anhang]{EU2024-2690}
            \footcite[Vgl. Nummer 6.2.2 \& 6.2.3.,][, Anhang]{EU2024-2690}
        } Darüberhinaus kann der Einsatz einer kritischen Komponente unter bestimmten Bedingungen untersagt werden. Abseits davon kann auch der generelle Einsatz von kritischen Komponenten eines Herstellers untersagt werden. Das stellt entsprechend für Hersteller ein Risiko dar Produkte an bestimmte Einrichtungen nicht mehr vertreiben zu können. Zusätzlich kann durch Rechtsverordnung bestimmt werden das IKT-Produkte, IKT-Dienste und IKT-Prozesse nur eingesetzt werden dürfen, wenn diese über eine Cybersicherheitszertifizierung gemäß Artikel 49 der Verordnung \gls{eu} 2019/881 verfügen.\footnote{
            \footcite[Vgl. §30, Absatz 6,][]{NIS2UmsuCG}
            \footcite[Vgl. §41,][]{NIS2UmsuCG}
        }\medbreak
        Sanktioniert werden verschiedene Verstöße in Form von Geldstrafen, welche sich Teilweise auf Jahresumsätze beziehen. Erzielt eine Einrichtung ein Jahresumsatz von mehr als 500 Millionen Euro werden die Strafen im Bereich von 1,4\% bis 2,0\% anhand des Jahresumsatz bemessen. Lässt man hier die Fehlende Obergrenze außer Betracht kann man unter Anblick der verbleibenden Definitionen für Strafen verallgemeinert den Bereich 100 000 Euro bis 10 000 000 Euro festlegen. Diese Strafen werden unter anderem verhängt wenn gegen Registrierungs-, Melde- oder Nachweispflichten verstoßen wird, aber auch die nicht richtige oder vollständig Umsetzung der Risikomanagementmaßnahmen.\footnote{
            \footcite[Vgl. §65,][]{NIS2UmsuCG}
        }

    %:: Abschnitt: Herleitung des Fragenkatalogs
    \newpage
    \section{Praktische Maßnahmen}\label{sec:ParktischeMaßnahmen}
    Der folgende Abschnitt beschreibt exemplarisch organisatorische oder technische Maßnahmen, welche die Erfüllung der Anforderung an \gls{nis2umsucg}, der \emph{Durchführungsverordnung (EU) 2024/2690} und \gls{kritis-dachg}. Aus den Rechtlichen Anforderungen können Allgemeine Erfordernisse für die Netz- und Informationssysteme abgeleitet werden. So ist die zentrale Verwaltung Identitäten Vorteilhaft, jedoch erfordert dies die Unterstützung der Verwendung von externen Identitätsanbieter\footcite[Vgl. Nummer 3.2.3., 11.2.2.,][]{EU2024-2690} in den Netz- und Informationssystemen. Ebenfalls ist ein für die zu verwendenden Identitäten ein Konzept für Rollen- und Berechtigung\footcite[Vgl. Nummer 11.2.2.,][]{EU2024-2690} erforderlich. Die in dem betroffenen Netz- und Informationssystem anfallenden Protokolle sollten ebenfalls in einer Art verfügbar sein, sodass diese Zentralisiert verarbeitet werden können.\footcite[Vgl. Nummer 3.2.2., 3.2.3.][]{EU2024-2690}. Für die Integrität der Daten ist sowohl die Kommunikations- als auch Datenverschlüsselung\footcite[Vgl. Nummer 9.2][]{EU2024-2690} ein Zentraler Bestandteil. Zusätzlich sollten die Netz- und Informationssysteme in einer hinreichenden Art interoperable sein, sodass information wechselseitig verarbeitet werden können. 
        \subsection{Personal, Dienstleister und Lieferanten}
        Im Folgenden werden die generellen und wichtigsten Erfordernisse und Maßnahmen hinsichtlich Subjekten in einem Unternehmenskontext, bezogen auf die durch \gls{nis2umsucg} und \gls{kritis-dachg} definierte Anforderungen, beschrieben.
            \subsubsection{Personal}
            % Nummer: 8.1. und 8.2. Schulung und unterweisung
            % On- Change- und Offboardingprozess (Mitarbeiterlebenszyklus)
            %   Nummer 10.1.2. Buchstabe d Polizeiliches Führungszeugnis und einstellungstest
            %   Nummer 8.2. und 10.1.2. Buchstabe a und b Unterweisung bei Onboarding (wie unser IT-Onboarding)
            %   Nummer 10.3.2. Verschwigenheitsvereinbahrung
            %   Nummer 12.2.2. Buchstabe a und 12.5. (Handhanbung IT Assets eines MA)
            Ein Zentraler Bestandteil eines Unternehmens ist das Personal. Hierbei gibt es in den verschiedenen Phasen des Lebenszyklus von Personal, von Einstellung bis hin zu Offboarding, diverse Themen. Im Rahmen des \gls{nis2umsucg} ist vor Einstellung sowohl die Identität als auch die Eignung zu prüfen.\footcite[Vgl. Nummer 10.1.2. Buchstabe d][]{EU2024-2690} Die Identität selbst kann über den jeweiligen Personalausweis im Rahmen eines persönlichen kennenlernes erfolgen. Personaldokumente, wie der Personalausweis der \gls{brd}, haben Sicherheitskennzeichen um dessen Authentizität zu gewährleisten.\footcite[Sicherheitskennzeichen Personalausweis][]{MISSING} Darüberhinaus kann ein Auszug aus dem Bundeszentralregister (Führungszeugnis) durch die Person bei der für Sie zuständige Meldebehörde beantragen werden. Somit wird unabhängig der eigenen Kenntnisse im Rahmen der Beantragung die Identität der Person durch die Meldebehörde implizit bestätigt.\footcite[Vgl. §30 Absatz 2][]{bzrg} Neben der Identität muss die Kompetenz gewährleistet werden. Eine entsprechende Arbeitsprobe basierend auf realen Herausforderungen aus dem Unternehmen bietet eine mittlere prognostische Validität für spätere Leistung und lassen somit indirekt Rückschlüsse auf die Kompetenzen zu. Zusätzlich gibt es einen moderaten Zusammenhang zwischen Ergebnis der Arbeitsprobe sowie kognitiven Fähigkeiten der Person.\footnote{\footcite[Vgl. S. 1020 Tabelle 1][]{69ec61}\footcite[Vgl. S. 1025 Tabelle 4][]{69ec61}} Während des Beschäftigungsverhältnisses und zu Beginn dessen sind Unterweisungs- und Schulungsmaßnahmen durchzuführen. Das Personal muss informiert werden wie mit dem Thema Informationssicherheit innerhalb des Unternehmens umgegangen wird. Hierzu ist eine auf den von dem Unternehmen aufgestellten Konzepten basierende Unterweisung zu erstellen.\footcite[Vgl. Nummer 10.1.2. Buchstabe a und b][]{EU2024-2690}. Darüberhinaus ist das Personal zu Schulen und ein entsprechendes Angebot gemäß der individuellen Risiken anzubieten. In allen Fällen soll Anwendergerecht über das Thema Cyber Security Awareness in Form von Themen wie Social Engineering\footnote{Eine gezielte psychologische Manipulation von Menschen – und nicht von technischen Systemen –, um sie dazu zu bringen, sensible Informationen preiszugeben, unbefugten Zugriff zu gewähren oder Handlungen auszuführen, die die Sicherheit gefährden} und Phishing\footnote{Eine Social-Engineering-Attacke, bei der sich ein Angreifer als vertrauenswürdige Instanz ausgibt – in der Regel per E-Mail, SMS, Telefonanruf oder gefälschter Website –, um Opfer dazu zu verleiten, sensible Daten preiszugeben, Malware zu installieren oder unbefugten Zugriff zu gewähren.}, aber auch Themen wie Sicherheitskonfiguration und Datensicherheit\footcite[Vgl. S. 4 und 5][BSI – Leitfaden zu Schulungsinhalten im Bereich der Cyber-Luftsicherheit]{MISSING} haben im Kontext von \gls{nis2umsucg} und dessen Anforderungen eine Allgemeine Relevanz.\footcite[Vgl. Nummer 8.1. und 8.2.][]{EU2024-2690} Darüberhinaus müssen die Mitarbeiter im Umgang mit den von ihnen verwendeten Systemen, wie beispielsweise das persönliche Arbeitsmittel, geschult sein. Dies Umfass Aspekte wie den Umgang in Form von Verwendung und Transport aber auch Bereiche wie Rückgabe bei Beendigung des Beschäftigungsverhältnis.\footnote{\footcite[Vgl. S. 6][OPS.1.1.1.A6]{MISSING}\footcite[Vgl. S. 8 - 9][Konkretisierung der Anforderungen an die gemäß § 8a Absatz 1 und Absatz 1a BSIG umzusetzenden Maßnahmen]{MISSING}\footcite[Vgl. Nummer 12.2.2. Buchstabe a][]{EU2024-2690}\footcite[Vgl. Nummer 12.5.][]{EU2024-2690}}Ebenfalls nach Beendigung des Beschäftigungsverhältnisses ist die Verschwiegenheit über unternehensrelevanten Informationen zu wahren. Hierzu sind Arbeitsvertragliche Regelungen erfolgreich, sofern es Aspekte im Kontext der Beschäftigung gibt, welche nicht durch das \gls{geschgehg} abgedeckt sind. Zusätzlich muss darauf geachtet werden das durch etwaige Formulierungen einer Verschwiegenheitsvereinbarung keine unangemessen benachteiligen entstehen, da dies eine Verschwiegenheitsvereinbarung unwirksam macht.\footnote{\footcite[Vgl. §307, Abs. 1][postnote]{BGB}\footcite[Vgl. §3 und §4][postnote]{GeschGehG}}
            \subsubsection{Sanktionen und disziplinarische Konsequenzen}
            % Nummer 10.4.1. Erforderliche Schulung -> Gespräche -> Abmahnung
            Für Verstöße gegen Konzepte betreffend der Netz- und Informationssysteme sind disziplinarische Konsequenzen einzuführen, zu kommunizieren und aufrecht zu erhalten.\footcite[Vgl. Nummer 10.4.1.][]{EU2024-2690} Prinzipiell besteht die Möglichkeit im Rahmen des Beschäftigungsverhältnisses Vertragsstrafen zu definieren, sofern ein Schaden aufgrund zuwiderhaltung des definierten Rahmens entsteht.\footcite[Vgl. §339][]{BGB}. In allen Fällen kann bei vertragswidrigen Verhalten das Beschäftigungsverhältnis mit vorhergehender Abmahnung außerordentlich Beendet werden.\footnote{\footcite[Vgl. §314 und §323][]{BGB}\footcite[Vgl. S 543][]{9783648174258}} Ziel einer Abmahnung muss die Beschreibung des Verstoßes, die Klarstellung der pflichten und die Konsequenzen bei zuwiederhaltung, als Warnfunktion, sein.\footcite[Vgl. S 423][]{9783648174258} Konsequenzen sind wichtig um eine Änderung der Verhaltens beizuführen, jedoch sollte initial, bis zu einem gewissen Punkt, entsprechende Problemgesprächte geführt werden. Bestandteil hiervon sollen Informationsabgleich zu Situation und Bedarf, Abgleich von Positionen und Interessen, Lösungsoptionen entwickeln und zusammentragen sowie eine entscheidung über die Lösung und ein abschließender Aktionsplan.\footnote{
                \footcite[Vgl. S. 143][]{9783658023638}
                \footcite[S. 129 - 132][]{9783658023638}
            }

            \subsubsection{Dienstleister und Lieferanten} 
            % On- und Offboarding von Dienstleistern
            %   Nummer 8.1.2., 10.1.2. Buchstabe a und b Unterweisung bei Onboarding
            % Nummer 5.1.2. -> Richtlinie an Anforderung für Lieferanten
            % Nummer 5.1.4. -> Richtlinie an Einhaltung der Anforderungen an Lieferanten            
            Dienstleister und Lieferanten erfordern als Risikoquelle ebenfalls ein gewisses Maß an Informationssicherheit. Dies obliegt nicht alleine den Dienstleister und Lieferanten, vielmehr sind Schulungsmaßnahmen, wie mit dem Thema Informationssicherheit innerhalb des Unternehmens umgegangen wird, erforderlich. Dies ist Analog den Anforderungen an Maßnahmen für das Personal, weswegen es sich anbietet bei überschneidenden Aspekten einheitliche Maßnahmen zu etablieren.
            Neben den Personellen Anforderungen ist ein Verzeichnis für direkte Anbieter und Diensteanbieter zu errichten. dies kann beispielsweise über ein \gls{erp} System erfolgen. Aus diesem müssen Kontaktstellen sowie eine Liste der \gls{ikt-produkt}, -Dienste und -Prozesse abrufbar sein. Bevor ein Dienstleister oder Lieferanten in dieses Verzeichnis aufgenommen werden kann muss dieser entsprechende Anforderungen erfüllen. Diese Anforderungen können im Rahmen einer Richtlinie Unternehmensweit Anwendung finden. Vorgelagert stehen Kriterien für die Auswahl eines Dienstleister oder Lieferanten im Vordergrund. Bestandteil müssen die Cybersicherheitsverfahren sowie die Sicherheit derer Entwicklungsprozesse und die Fähigkeit die eigens Festgelegten Cybersicherheitsspezifikationen zu erfüllen sein. Nicht nur der Auswahlprozess ist eine Zentrale Rolle, auch die Durchführung und Einhaltung auf Seiten des Dienstleister oder Lieferanten. Vertraglich muss im Rahmen der Leistungsvereinbarung mit dem Anbieter geregelt sein das die Cybersicherheitsanforderungen eingehalten, Sensibilisierungs-, Qualifikations- und Ausbildungsanforderungen und Anforderungen an die Zuverlässigkeitsüberprüfungen an die Mitarbeitenden der Anbieter oder Diensteanbieter gestellt werden. Darüberhinaus muss der Anbieter verpflichtet werden betreffende und relevante Prüfberichte auszuhändigen, Sicherheitsvorfälle die ein Risiko darstellen unverzüglich zu melden und Schwachstellen zu beheben. Ebenfalls muss für die Unterauftragsvergabe durch den Anbieter oder Diensteanbieter entsprechende Cybersicherheitsanforderungen an den Unterauftragnehmer gestellt werden. Für die Erstellung der Richtlinien kann man sich der \emph{Best-Practice-Empfehlungen für Anforderungen an Lieferanten zur Gewährleistung der Informationssicherheit in Kritischen Infrastrukturen} des \gls{bsi} bedienen.\footnote{
                \footcite[Vgl. Nummer 5.2.][]{EU2024-2690}
                \footcite[Vgl. Nummer 8.2.1.][]{EU2024-2690}
                \footcite[Vgl. Nummer 10.1.2. Buchstabe a und b][]{EU2024-2690}
                \footcite[Vgl. S. 5][Best-Practice-Empfehlungen für Anforderungen an Lieferanten zur Gewährleistung der Informationssicherheit in Kritischen Infrastrukturen]{MISSING}\footcite[Vgl. Nummer 5.1.2.][]{EU2024-2690}
                \footcite[Vgl. Nummer 5.1.4.][]{EU2024-2690}
            }

        \subsection{Verwaltung der Systeme}
        % TODO: Einleitung schreiben
            \subsubsection{Software- und Sicherheitspatch-Management}
            % Nummer 6.6., 6.9.
            % https://learn.microsoft.com/en-us/intune/intune-service/fundamentals/what-is-intune
            % https://www.statista.com/statistics/218089/global-market-share-of-windows-7/
            % https://learn.microsoft.com/en-us/intune/intune-service/fundamentals/supported-devices-browsers
            % https://www.bsi.bund.de/SharedDocs/Downloads/DE/BSI/Grundschutz/IT-GS-Kompendium_Einzel_PDFs_2022/07_SYS_IT_Systeme/SYS_3_2_2_Mobile_Device_Management_Edition_2022.pdf?__blob=publicationFile&v=3
            Ein wichtiger Bestandteil der Verwaltung von Systemen ist die Verwaltung der darauf verwendeten Applikationen sowie dessen Aktualität, aber auch die Aktualität des Betriebssystems. Allgemein sind Verfahren für das Sicherheitspatch-Management sowie der Schutz gegen Schad- und nicht genehmigte Software. Sicherzustellen ist das die die Verwendung von Schadsoftware oder nicht genehmigter Software aufgedeckt oder verhindert wird. Ebenfalls sind Sicherheitspatches innerhalb einer angemessenen Frist anzuwenden. Beide Aspekte können mittels \gls{mdm} und \gls{mam} realisiert werden.\footcite[Vgl. S. 21 und 118][]{9781509301331}. Für beide Lösungen gibt es diverse Anbieter\footcite[Diverse MDM/MAM anbieter][]{MISSING}, wie beispielsweise Microsoft Intune. Microsoft Intune bietet erfüllt die durch das \gls{nis2umsucg} Beschriebenen Anforderungen und ist für gängige Betriebssysteme verfügbar\footcite[Intune Supported os, Feature Matrix][]{MISSING}. Weiterhin kann Sicherheitspatch-Management für Systeme, basierend auf Windows oder Linux, durch Open-Source Lösungen wie Ansible, Winget-AutoUpdate und dem Windows Paket-Manager winget Ergänzend oder ausschließlich eingesetzt werden.\footcite[][winget etc.]{MISSING}
        \subsection{Dokumentation}
        % TODO: Einleitung schreiben
            \subsubsection{Anlagen- und Werte, Betriebsabläufe und Konfigurations-Management}\label{subsubsec:AuWBuKM}
            % Nummer 6.7.2. Buchstabe a
            % Nummer 6.3.2.
            % Nummer 12.4.2.
            % https://www.omg.org/spec/BPMN/2.0/PDF
            Ein Informationssystem, welches mehrere Anforderungen des \gls{nis2umsucg} erfüllen kann, sind \gls{cmdb}-Systeme. Mit diesen können die essenziellen Komponenten der Dokumentation der  Konfigurationen für die Lebensdauer von Hardware, Software, Dienste und Netze gewährleistet werden. Ergänzend hierzu müssen entsprechende Prozesse von Bereitstellung bis Dekommissionierung etabliert werden, sodass die Durchsetzung der festgelegten sicheren Konfigurationen entsprechend forciert wird.
             Die Informationen der \gls{it} betreffenden Anlagen- und Werte ergänzen die Notwendige pflege vollständigen, genauen, aktuellen und kohärenten Inventars. In der Form ergänz es das notwendige Anlagenverzeichnis sowie ein zu errichtendes \gls{bpm}-System zur pflege der Betriebsabläufe. Alternativ kann zu dem Informationssystem in Form eines \gls{bpm}-System die pflege mittels einer Visualisierungsprogramm, wie \emph{bpmn.io}, und der \gls{bpmn}. \gls{bpmn} ist ein Standard für die einheitliche und verständliche Dokumentation von Geschäftsprozessen.\footnote{
                \footcite[S. 1][BPMN 2.0]{MISSING}
                \footcite[Vgl. §284 Abs. 3][]{HGB}
                \footcite[Vgl. §4 Abs. 3 Satz 5 ][]{EStG}
                \footcite[Vgl. 11 - 12][]{9780128012659}
                \footcite[Vgl. S. 1][A Framework for BPM Software Selection in Relation to Digital Transformation Drivers.]{MISSING}
            }
            \subsection{Veränderungsmanagement}
            % Nummer 6.4.1., 6.4.2. und 6.4.3.
            Änderungen an Netz- und Informationssystemen müssen Verfahren unterliegen, welche die Freigaben, Änderungen und Notfalländerungen an in Betrieb befindlicher Software und Hardware sowie für Änderungen der Konfiguration umfassen. Veränderungsmanagement bezieht sich nicht ausschließlich auf den Kontext von Netz- und Informationssystemen, sondern es betrifft das ganze Unternehmen. Wichtig ist das im Rahmen der Anforderungen die Maßnahmen zum Veränderungsmanagement der cx in Einklang mit den Allgemeinen verfahren des Veränderungsmanagement stehen. Grundlegend sollen diese beiden Arten ohnehin voneinander getrennt sein. Wichtig ist das Veränderungen immer einen Mehrwert liefern und durch Personen bewertet werden die den zu generierenden Mehrwert und die Risiken einschätzen können. Ebenfalls ist jede Änderung durch eine dafür autorisierte Person freizugeben. Das Veränderungsmanagement für Netz- und Informationssystemen kann in die drei Typen der \emph{standard Änderungen}, \emph{normalen Änderungen} und \emph{Notfalländerungen} unterschieden werden. \emph{Stand Änderungen} stellen alle vorautorisiert Änderungen dar, welche mit einem geringen Risiko behaftet sind. Dies sind gut Dokumentierte und bekannte Arbeitsabläufe und Änderungen, welche während des Tagesgeschäfts durchgeführt werden. Die \emph{normalen Änderungen} unterliegen in aller Regel ebenfalls einem geringen Risiko, stellen aber größere Änderungen dar. Diese sind gesondert zu bewerten, zu genehmigen und zu planen. \emph{Notfalländerungen} hingegen sind alle Änderungen die unmittelbar durchgeführt werden müssen um Akute Probleme, wie beispielsweise das schließen einer Schwachstelle, zu beheben. Diese Änderungen werden trotz alledem vorab Bewertet und Genehmigt, jedoch ist dieser Prozess beschleunigt. Im Falle einer \emph{Notfalländerungen} kann akzeptiert werden die Dokumentation nachträglich durchzuführen. Demnach ist ein Änderungsprozess sowie ein Prozess für die Definition von \emph{standard Änderungen} zu etablieren. Hierbei kann \gls{itsm}, respektive entsprechende Informationssysteme, unterstützen. Open-Source Lösungen wie iTop (IT Operational Portal) könne diesen Prozess digital unterstützen und gleichzeitig die Grundlage für die in Abschnitt \ref{subsubsec:AuWBuKM} beschriebenen Anforderungen an das \emph{Konfigurations-Management} bedienen. 
            \footnote{
                \footcite[Vgl. S. 6 - 7][]{9783642301056}   
                \footcite[Vgl. S. 159 - 162][postnote]{9789925600083}
                \footcite[Vgl. S. 34 - 36][]{9781780176079}
                \footcite[Vgl. S. 1][]{MISSING} % https://www.itophub.io/wiki/page
            }
        \subsection{Sicherheits- und Identitätsverwaltung}
        % TODO: Einleitung schreiben
            \subsubsection{Managementsystem für Informationssicherheit}
            % 3.1.1. & 3.2.2. -> Sicherheitsvorfälle erkennen, behandeln, etc.
            % 1.1.1. -> Allgemeines Konzept für Sicherheit
            % 3.5.2. & 3.6.1., 3.6.2. -> Reaktion, Bewältigung und nachträgliche Prüfung sowie Verbesserung von Sicherheitsvorfällen
            % 3.4.2. Vorfälle Klassifizieren und Priorisieren
            Allgemein muss ein Konzept für die Sicherheit der Netz- und Informationssysteme existieren. Zentraler Bestandteile ist das erkennen, klassifizieren und bewältigen von Vorfällen. Hierfür kann ein \gls{isms} errichtet werden. Das \gls{isms} zentralisiert die Sicherheitsstrategie, welche sich den Hilfsmitteln zur Umsetzung und der Dokumentation bedient. Die Hilfsmittel sind die Sicherheitsorganisation und das Sicherheitskonzept. Die Sicherheitsorganisation besteht aus den Regeln, Anweisungen, Prozessen, Abläufen und Strukturen, wohingegen das Sicherheitskonzept die Beschreibung des Geltungsbereiches und die Risikobewertung sowie -behandlung. Die Dokumentation hingegen ist die Leitlinie der Informationssicherheit, welche die Kernaspekte der Sicherheitsstrategie zur Informationssicherheit dokumentiert. Ein weiterer wichtiger Bestandteil des \gls{isms} ist die Erfolgskontrolle sowie die Kontinuierliche Verbesserung des Sicherheitsprozesses.
            \footnote{
                \footcite[Vgl. S. 15 - 17, 18, 24 - 15, 33 - 34][]{bsi-200-1}
                \footcite[Vgl. S. 9 - 10][]{iso27001-2022}
                \footcite[Vgl. S. 47 - 50, 53 - 54][]{9781780175188}
            }
            \subsubsection{Sicherer E-Mail Verkehr}
            % Nummer 6.7.2. Buchstabe k
            % SPF in RFC 7208
            % DKIM in RFC 6376
            % DMARC in RFC 7489 zum Monitoring
            % Secure Email Gateway (SEG)
            Damit ein Sicherer Verkehr zwischen E-Mail Servern erfolgen kann gibt es grundlegende Mechanismen wie \gls{spf}, \gls{dkim} und \gls{dmarc}. Das \gls{spf} soll empfangene E-Mail Server befähigen zu prüfen ob ein Absendender Server berechtigt ist von der in der E-Mail angegebenen Domäne zu senden und wie mit E-Mails verfahren wird, welche nicht der konfigurierten Richtlinie entsprechen. Für die Konfiguration ist ein \emph{TXT} Eintrag für diese Domäne im \gls{dns} erforderlich. \gls{dkim} hingegen soll mittels einer kryptografischen Signatur die Authentizität der E-Mail gewährleisten. Hierbei wird die Authentizität des Signierers und nicht des Verfassers der Nachricht gewährleistet. Unter Verwendung eines asymmetrischen kryptografischen Verfahrens wird die E-Mail mittles eines privaten Schlüssels Signiert, sodass der Empfänger diese über den öffentlichen Schlüssel verifizieren kann. Dieser öffentliche Schlüssel wird als Eintrag der Domäne im \gls{dns} in Form eines \emph{TXT} Eintrages veröffentlicht. Beide Mechanismen können mittels \gls{dmarc} kontrolliert werden. Dies ermöglicht sowohl eine Rückmeldung zum Nachrichtenfluss als auch die Durchsetzung von Richtlinien gegen nicht authentifizierte E-Mails. Domänen-Besitzer und Empfänger stellen diese Informationen mittels eines \emph{TXT} Eintrag in dem für ihre Domäne zuständige \gls{dns} bereits. Es existieren weitere Sicherheitsmechanismen, wie beispielsweise \gls{smime}, jedoch bieten \gls{spf}, \gls{dkim} und \gls{dmarc} durch ihre Zentrale und Domänenweite Konfiguration die Möglichkeit mit dem geringsten Aufwand den E-Mail verkehr sicherer zu gestalten. Unabhängig davon sollte die erzwungene Transportverschlüsselung zwischen den Transportservern des E-Mail Datenverkehr forciert werden um zu gewährleisten das jegliche Kommunikation in diesem Kontext Verschlüsselt ist.\footnote{
                \footcite[][1. Introduction]{RFC7208}
                \footcite[][3. SPF Records]{RFC7208}
                \footcite[][3.1. DNS Resource Records]{RFC7208}
                \footcite[][1]{RFC6376}
                \footcite[][5.1]{RFC5585}
                \footcite[][4.1]{RFC5863}
                \footcite[][1]{RFC8551}
                \footcite[1][]{RFC8689}
            }
            \subsubsection{Netzwerke}
            % Nummer 6.8.2.
            % Segmente (aus erster Hausarbeit) inkl. Entwicklungsumgebung
            % Nummer 6.7.2. Buchstabe b,c,d,f,i,l; Nummer 6.2.2. Buchstabe b
            % Firewall sowie Topologie eines DNS Systems (OPNSense)
            % > ZTNA Lösung sowie das Blockieren unsicherer Verbindungen (HTTP) und nicht notwendiger über die Firewall
            Die Netzwerke sind zu Segmentieren und Maßnahmen für die Netzsicherheit der Netz- und Informationssysteme zu etablieren. Für die Segmentierung werden zunächst die Netzwerke in logische Sicherheitszonen unterteilt. Eine Sicherheitszone ist ein Teil des Netzwerks und wird durch die allgemeinen Sicherheitseigenschaften der darin befindlichen Systeme bestimmt. Als nicht vertrauenswürdige Zone gilt der Bereich, in dem sich aus Sicht des Netzwerks alle Systeme befinden, über die keine Kontrolle besteht. Dies betrifft hauptsächlich den Telekommunikationsdatenverkehr aus den öffentlich zugänglichen Telekomunikationsnetzen, also dem Internet. Die demilitarisierte Zone befindet sich hingegen unter dem Einfluss des Netzwerks und umfasst alle Systeme, die in irgendeiner Form den öffentlich zugänglichen Telekomunikationsnetzen teilweise oder vollständig verfügbar gemacht werden. Alle Dienste, die sowohl von der demilitarisierten Zone als auch von der nachfolgend beschriebenen eingeschränkten Zone benötigt werden, befinden sich in der vertrauenswürdigen Zone. Die vertrauenswürdige Zone enthält keine sensitiven Daten und dient als Zwischenschicht für einen gesicherten Informationsaustausch. Die eingeschränkte Zone, die das interne Netzwerk darstellt, umfasst alle Teilnehmenden, die in irgendeiner Form sensitive Daten verarbeiten; sie ist daher die restriktivste und schützenswerteste Zone. Damit die Teilnehmenden in den verschiedenen Zonen gesichert administriert werden können, gibt es eine Verwaltungszone. Diese ermöglicht den Administratorinnen und Administratoren eines Netzwerks einen unidirektionalen Zugriff auf die Teilnehmenden der verschiedenen Zonen. Alle Sicherheitszonen sind hinsichtlich ihrer Kommunikation unterschiedlich verknüpft; die Kommunikation zwischen den Zonen ist nicht immer unidirektional, sondern kann auch bidirektional sein. Der Datenverkehr zwischen den Zonen ist auf das erfolgreiche minimum zu beschränken. Dies kann mittels einer Firewall erreicht werden. Die Aufgabe von Firewalls besteht darin, durch Kontrolle und Filterung von Datenpaketen die Weiterleitung solcher Pakete zu unterbinden, die eine mögliche Bedrohung für die Daten und Komponenten eines Telekommunikationsnetzwerkes darstellen.\footnote{
                \footcite[Vgl. S. 113 - 114][]{9780128020425}
                \footcite[Vgl. S. 93][]{9780124201149}
                \footcite[Vgl. S. 714][postnote]{9783486721386}
            }
            \subsubsection{Identitätsmanagement}
            % Nummer 11.1.2.
            % Nummer 11.2.2. Buchstabe b - f; für c ein Prozess
            % Nummer 11.3.2. Gesonderte Administrationskonten - Organisatorische Anweisung hinsichtlich Nutzung
            % Nummer 11.5.2.
            % Nummer 11.5.3. Prozess zur Genehmigung
            % Nummer 11.6.2. Prozess zur sicheren mitteilung initialer Kennwörter mit initialer änderungsforcierung
            % Nummer 11.7.1. MFA erfordern wo möglich; durch IAM einfacher
            Der Zugang zu Netz- und Informationssysteme sowie zu dessen Daten und Informationen ist zu Kontrollieren. Zentraler Aspekt ist das die Prinzipien von \emph{Need-to-know} sowie \emph{Need-to-use} eingehalten werden. Sämtliche Personen und Geräte, wie Personal oder Dienstleister, Fallen unter die Notwendigkeit der Zugriffskontrolle. Dies kann mittels \gls{iam} erfolgen. Ziel von \gls{iam} ist das Verwalten von Identitäten über dessen Lebenszyklus hinweg inklusive der Protokollierung von Änderungen an einer Identität. Zusätzlich müssen Arbeitsabläufe etabliert werden, welche das erstellen einer Identität genehmigt und die sichere Kommunikation von initialen Zugangsdaten gewährleistet. Ebenfalls ist zu gewährleisten das Identitäten zu Verwaltung von Systemen separiert von denen des alltäglichen Arbeitens sind. Für die Erhöhung der Sicherheit der Zugänge, verknüpft mit den Identitäten, kann \gls{mfa} eingeführt werden. Hierfür und im Allgemeinen ist eine zentralisiertes \gls{iam} inklusive Föderation von Identitäten förderlich. Zu diesem Zweck können Lösungen wie \emph{Keycloak} eingesetzt werden. \emph{Keycloak} ist ein Open-Source-Tool für die Identitäts- und Zugriffsverwaltung. Durch die Delegierung der Authentifizierung an \emph{Keycloak} muss sich das Netz- und Informationssystem nicht mit verschiedenen Authentifizierungsmechanismen oder dem sicheren Speichern von Passwörtern befassen. \emph{Keycloak} basiert auf branchenüblichen Protokollen und unterstützt \gls{oauth} 2.0, OpenID Connect und \gls{saml} 2.0. Die Verwendung branchenüblicher Protokolle ist sowohl aus Sicherheitsgründen als auch im Hinblick auf die einfachere Integration in bestehende und neue Netz- und Informationssystem wichtig.
            \footnote{
                \footcite[Vgl. S. 3 - 6, 13 - 15][]{9781836797661} % https://www.google.de/books/edition/Study_Guide_to_Identity_and_Access_Manag/cGAsEQAAQBAJ?hl=de&gbpv=1&kptab=overview
                \footcite[Vgl. S. 150, 173 - 180][]{9789811926570}
                \footcite[Vgl. S. 4 - 5, 187, 197, 203 - 204][]{9781800562493}
            }
            \subsubsection{Verschlüsselung}
            % Nummer 9.2.
            Um die Anforderung an die Verschlüsselung von Daten und Informationen bei Speicherung und Übertragung zu erfüllen ist eine \gls{pki} erforderlich. Zunächst ist wichtig das nach einem Krypto-Agilitätsansatz verschiedene Protokolle oder Protokollfamilien sowie kryptografische Algorithmen, Kryptierungsstärke, kryptografische Lösungen und Nutzungsverfahren bereitgestellt werden. Dies kann im Rahmen einer \gls{pki} mittels entsprechender Richtlinien durch die \gla{ca} forciert werden. Ebenfalls können über diese entsprechende \gls{pki} Zertifikate ausgestellt und widerrufen werden. Wichtig ist das Schlüssel Archiviert werden können, jedoch in solch einer Art das diese ab Archivierung nicht mehr Produktiv genutzt werden können. Dies ist zur Validierung älterer Informationen, die erstellt wurden, als das Zertifikat gültig war. Empfehlenswert ist das implementieren einer mindestens zweistufigen \gls{pki} in Form von einer Stammzertifizierungsstelle und Subzertifizierungsstellen für die Anwenderzertifikate. Die Anzahl der Subzertifizierungsstellen kann von den Konkreten Anforderungen abhängig gemacht werden. Um dem Zweck der Netzwerksegmentierung nachzukommen ist die Einführung von entsprechenden Registrierungsstellen dienlich, da somit die Zertifikatsanfragenden Systeme von den Zertifikatsausstellenden Systemen hinsichtlich einer direkten Kommunikation getrennt werden. Zu diesen technischen mitteln sind Verfahren zu etablieren, sodass das Personal über dessen Existenz und Verwendung abgeklärt ist.\footnote{
                \footcite[Vgl. S. 216 - 218][]{9781498707480}
                \footcite[Vgl. S. 63 - 70][]{9781907117046}
                \footcite[Vgl. S. 23 - 28, 47 - 50][]{9780815396413}
                \footcite[Vgl. S. 293, 302][]{9780763791285}
            }
            \subsubsection{Schwachstellenmanagement}
            % Nummer 6.10. Scanner + Abo bei EU stelle wie CVEs
            Basis für das Schwachstellenmanagement können Programme wie das \gls{cve} oder \gls{euvd} sein. Ziel beider Programme ist es bekannt gewordene Schwachstellen öffentlich verfügbar zu machen. Hier muss entsprechendes Fachpersonal im Kontext der betroffenen Einrichtung Meldungen bewerten und Maßnahmen entsprechend der definierten Konzepte der Informationssicherheit ergreifen. Teil des Schwachstellenmanagement ist die zuvor beschriebene Erkennung und Beseitigung. Zunächst muss man sich bewusst werden welche Aspekte von Schwachstellen Betroffen sein können. Letztendlich sind es nicht zwangsläufig die Unmittelbaren Netz- und Informationssysteme oder Anwendungen, sondern auch Software von Drittanbietern und Open-Source-Software die in diesen Komponenten verwendet werden. Das Schwachstellenmanagement an sich ist ein kontinuierlicher Prozess, welcher mittels Automatismen gestützt werden sollte. Hierfür können offene Lösungen wie \gls{openvas} verwendet werden, welche ein Rahmenwerk, bestehend aus mehreren Diensten und Tools, sind, dass eine umfassende und leistungsstarke Lösung für das Scannen und Verwalten von Schwachstellen bietet.\footnote{
                \footcite[][EU Vulnerability Database (EUVD)]{MISSING}
                \footcite[Vgl. S. 6 - 8][]{978-1-63081-938-5}
                \footcite[Vgl. S. 1 - 9, 14][]{9781394221226}
                \footcite[Vgl. S. 48][]{9781484242698}
            }
        \subsection{Daten, Informationen und Systeme}
        % TODO: Einleitung schreiben
            \subsubsection{Netz- und Informationssysteme sowie Applikationen}
            % Allgemeine Anforderung an zu beschaffende Netz- und Informationssysteme sowie Applikationen
            %           Nummer 3.2.1 -> Protokollierung muss gegeben sein
            %           Nummer 6.1.2. -> Sicherheitsmaßnahmen beim Erwerb von IKT-Diensten oder IKT-Produkten
            %           Nummer 6.2.1. und 6.2.2. -> Sicherer Entwicklungszyklus
            % Nummer 12.2.2. Handhabung Assets für Erwerb, Verwendung, Speicherung, Transport und Entsorgung
            % Nummer 12.2.1. Klassifizierung für Anforderunge an Vertraulichkeits-, Integritäts-, Authentizitäts-  und  Verfügbarkeitsanforderungen 
            % Nummer 4.2.2.
            Nicht nur der Erwerb von Netz- und Informationssysteme sowie Applikationen hinsichtlich der Lieferanten sondern auch die Lösungen an sich müssen bestimmte Anforderungen erfüllen. Hierbei ist unerheblich ob ein Netz- und Informationssysteme sowie Applikationen eigenentwickelt oder beschafft wurde. Wenn man die gesamtheitlichen Anforderungen an die \emph{Durchführungsverordnung (EU) 2024/2690} betrachtet ergeben sich diverse implizite Anforderungen. So muss gegeben sein das ein externer Identitätsanbieter verwendet werden kann. Zusätzlich muss Protokollierung des System, zumindest lokal, existieren. Zusätzlich solle vor Erwerb die Klassifizierung für Anforderungen an Vertraulichkeits-, Integritäts-, Authentizitäts-  und  Verfügbarkeitsanforderungen bekannt sein, sodass eine objektive Auswahl unter Anbetracht der existierenden Gegebenheiten möglich ist. Neben diesen funktionalen Anforderungen die sich aus der Gesamtheit ergeben existieren ebenfalls explizitere Anforderungen. Es müssen Sicherheitsanforderungen und -verfahren für den Entwicklungszyklus sowie die Entwicklungsumgebungen erstellt werden. Ebenfalls ist eine Analyse der Sicherheitsanforderungen in der Spezifikations- und Entwurfsphase jedes Entwicklungs- oder Beschaffungsvorhaben erforderlich. Um dem gerecht zu werden sind Vorschriften für die in- oder externe Entwicklung von xxx in den Phasen der Spezifikation, Konzeption, Entwicklung, Umsetzung sowie Tests festzulegen. Für das erstellen dieser Vorschrift kann man sich Publikationen des \gls{bsi} und der \gls{iso}. So sollten die Vorschriften Prinzipien der sicheren Systemarchitektur und technische Grundsätze umfassen, aber auch Anforderungen an die Dokumentation sind zu stellen. Allgemein sollte die Vorschrift die Themengebiete des Projektmanagement (Allgemeine funktionale und nicht funktionale Anforderungen, Beschaffung, IT-Betrieb, Personal), der Dokumentation, des Test und der Freigabe, der Installation, das Patch- und Änderungsmanagement sowie die Außerbetriebnahme umfassen um den gesamten Lebenszyklus abzudecken. Ebenfalls sollten cloudbasierte Dienstleistungen dediziert betrachtet werden, da hier der Einfluss auf beispielsweise Vertraulichkeits-, Integritäts-, Authentizitäts-  und  Verfügbarkeitsanforderungen, in Abhängigkeit des Servicemodell, eingeschränkt ist. Ergänzend und abschließend können in dem Rahmen ebenfalls die Handhabung bei Erwerb, Verwendung, Speicherung, Transport und Entsorgung festgelet werden um den Anforderungen an die Anlagen- und Werte gerecht zu werden.
            \footnote{
                \footcite[][Anforderungsarten]{MISSING}
                \footcite[][S. 10]{TR-03185}
                \footcite[][S. 10 - 11, 13]{Empfehlungen zu Entwicklung und Bereitstellung von in Kritischen Infrastrukturen eingesetzten Produkten}
                \footcite[][8.26, 8.27]{iso27002-2022}
                \footcite[prenote][S. 6, 9 - 12]{Nutzung von cloudbasierten Dienstleistungen in Kritischen Infrastrukturen – eine Hilfestellung des UP KRITIS}
            }
            \subsubsection{Sicherung, Wiederherstellung und Betriebskonti­nuität}
            % NOTE: Nummer 4.1. und 4.2.
            % NOTE: Nummer 3.2.5. Archivierung von Protokllen
            % NOTE: Nummer 13.1.2. Buchstabe a und b -> USV, Fallback Internet
            % NOTE: Nummer 13.3.2. ZuKo & CCTV & EMA
            \subsubsection{Systemverwaltungssysteme}
            % NOTE: Nummer 11.3.2. Buchstabe d; 11.4.2.
            % NOTE: Nummer 12.3.2. Autorun und nicht autorisierte Wechseldatenträger verhindern
            \subsubsection{Protokolle und Protokollierung}
            % NOTE: Nummer 3.2.1. und 3.2.3.; Windows und Linux Audit Logs + SaaS Protokolle
            % NOTE: Nummer 3.2.2. und 3.2.4. automatische und manuelle Analyse der Protokolle
            \footnote{
                \footcite[Vgl. S. 106 - 107][]{iso27002-2022}
            }
        \subsection{Risikomanagement}
    \section{Fragenkatalog}\label{sec:HerleitungDesFragenkatalog}
        \emph{In diesem Kapitel wird mit adäquaten Unterabschnitten, welche sich im Rahmen der Recherche ergeben werden, der Fragenkatalog an sich und dessen Herleitung, auf Basis der diversen Werke, beschrieben}
        
        \subsection{Fallstudie}


    %:: Abschnitt: Schlussbetrachtung
    \newpage
    \section{Schlussbetrachtung}
        \emph{Die Schlussbetrachtung soll alle abschließend wichtigen Punkte aufgreifen.}
        \subsection{Limitation}
        \emph{Im Rahmen der Limitation soll eingegrenzt werden was die Arbeit Limitiert hat. Unter anderem ist das Ergebnis auf den IST Stand der aktuellen Gesetzesentwürfe beschränkt.}
        \subsection{Ausblick}
        \emph{Aus der Arbeit können sich weitere Forschungsfragen ergeben. Nicht zuletzt der weitere Forschungsbedarf bzw. die erneute Validierung wenn aus den Entwürfen beschlossene Gesetze geworden sind.}
        \subsection{Fazit}
        \emph{Abschließend soll im Rahmen der Arbeit noch ein selbstkritisches Fazit gezogen werden.}

    %:: Abschnitt: Literatur
    \newpage
    \printbibliography
\end{document}