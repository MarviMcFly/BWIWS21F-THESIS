\documentclass[a4paper,11pt,twoside]{article} % twoside optional
\usepackage[margin=25mm]{geometry}
\usepackage[utf8]{inputenc}                                 % Eingabecodierung UTF-8 (für Umlaute etc.)
\usepackage[scaled]{helvet}                                 % Serifenlose Schrift Helvetica, skaliert
\usepackage{lmodern}    % schöne Schrift
\usepackage[T1]{fontenc}
\usepackage[ngerman]{babel}
\usepackage{microtype}

\usepackage{fancyhdr}
\setlength{\headheight}{15pt} % genug Platz für Header

% ---- Version hier zentral setzen ----
\newcommand{\docversion}{v1.0.0} % <=== HIER anpassen
\renewcommand{\familydefault}{\sfdefault}
% ---- fancyhdr-Setup ----
\pagestyle{fancy}
\fancyhf{} % alles leeren

% Links: Version (auf geraden & ungeraden Seiten)
\fancyhead[LE,LO]{\docversion}
% Rechts: Seitenzahl (auf geraden & ungeraden Seiten)
\fancyhead[RE,RO]{\thepage}

% dünne Linie unter dem Header (optional)
\renewcommand{\headrulewidth}{0.4pt}
% keine Fußzeile
\renewcommand{\footrulewidth}{0pt}

% Auch "plain"-Seiten (z. B. nach \maketitle / \section* etc.) anpassen:
\fancypagestyle{plain}{%
  \fancyhf{}
  \fancyhead[LE,LO]{\docversion}
  \fancyhead[RE,RO]{\thepage}
  \renewcommand{\headrulewidth}{0.4pt}
}

\title{Fragebogen zur Selbstbewertung für die NIS2 und KRITIS-Konformität}
\author{Marvin Künzel}
\date{\today}

\begin{document}
\maketitle
\thispagestyle{plain} % stellt sicher, dass auch die Titelseite den Header bekommt
\section{Einleitung}
Dieser Fragebogen wurde von \emph{Marvin Künzel} im Rahmen seiner \emph{Bachelor-Thesis} zur \emph{zur Erlangung des Grades eines Bachelor of Science (B.Sc.)} mit dem Thema \emph{Informationssicherheit in deutschen Unternehmen – Anforderungen und Umsetzungsoptionen im Kontext des KRITIS-Dachgesetzes und NIS2-Umsetzungsgesetzes} erarbeitet. Ziel ist es mittels des Fragebogen eine Selbsteinschätzung zum aktuellen Konformitätsstatus eines Unternehmens in Anbetracht des \emph{NIS-2-Umsetzungs- und Cybersicherheitsstärkungsgesetz} sowie \emph{KRITIS-Dachgesetz}.
\subsection{Haftungsausschluss}
Die Auswertung und die Ergebnisse dieses Fragebogens dienen ausschließlich der allgemeinen Information und dient Forschungszwecken. Sie stellen keine individuelle Beratung dar und begründen kein Vertrags- oder Beratungsverhältnis. Entscheidungen auf Basis der Fragebogen-Ergebnisse werden freiwillig und eigenverantwortlich getroffen. Ergebnisse dieses Fragebogens sind nicht bindend. Die Ergebnisse sind kritisch unter Anbetracht der angegebenen Referenzen zu prüfen. Es besteht keine Gewähr für die Richtigkeit, Vollständigkeit und Aktualität der bereitgestellten Informationen sowie dafür, dass die Auswertung in jedem Einzelfall zutreffend sind. Die dargestellten Ergebnisse, Analysen und Interpretationen sollen lediglich einen möglichen ersten Anhaltspunkt bieten und spiegeln die Sichtweisen des Forschenden wider und sind nicht als rechtliche, finanzielle oder hochwertige Handlungsempfehlungen zu verstehen. Der Forschende übernehmen keinerlei Haftung für Schäden, Verluste oder Nachteile, die direkt oder indirekt aus der Teilnahme an der Befragung oder der Nutzung der Ergebnisse entstehen könnten.
\subsection{Kontaktstelle}
Folgend finden Sie die Kontaktstellen im Kontext dieses Fragebogens.
\begin{table}[h!]
\centering
\begin{tabular}{lll}
\textbf{Art} & \textbf{Kontaktperson} & \textbf{Kontaktstelle(n)}   \\ \hline
Rückfragen   & Marvin Künzel          & questionaries@kuenzel-it.de \\ \hline
Resonanz     & Marvin Künzel          & questionaries@kuenzel-it.de \\
\end{tabular}
\end{table}
\subsection{Begriffsdefinitionen}

\subsection{Ergebnisermittlung}
\subsubsection{Gewichtung}
Die Fragen sind sind mit einer Gewichtung versehen. Sämtliche \emph{Kernfragen} haben das Gewicht \(w = 2\), wobei alle übrigen Fragen das Gewicht \(w = 1\) haben.  
\subsection{Ergebniseinschätzung}


\end{document}